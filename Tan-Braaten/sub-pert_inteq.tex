\documentclass[Note_on_Braatens.tex]{subfiles}


\begin{document}
\section{Perturbative view of integral equation}
If we're to have the integral equation only with perturbative counterterms, for a start we can write
\begin{align*}
	\begin{tikzpicture}[baseline=(o1.base)]
		\begin{feynman}
			\diagram[small,horizontal=p1 to p3]{
				p1 -- o1[blob, minimum size=30pt] -- p3;
				p2 -- o1 -- p4;
			};
			% \node[blob, minimum size=30pt] at (o1);
		\end{feynman}
	\end{tikzpicture} & =
	\begin{tikzpicture}[baseline=(o3.base)]
		\begin{feynman}
			\diagram[small,layered layout,horizontal=p1 to p3]{
				p1 -- o3 -- p3;
				p2 -- o3 -- p4;
			};
			\node at (o3);
		\end{feynman}
	\end{tikzpicture}+
	\begin{tikzpicture}[baseline=(o3.base)]
		\begin{feynman}
			\diagram[small,layered layout,horizontal=p1 to p3]{
				p1 -- o3 -- p3;
				p2 -- o3 -- p4;
			};
			\node[crossed dot] at (o3);
		\end{feynman}
	\end{tikzpicture}+
	\begin{tikzpicture}[baseline=(o1.base)]
		\begin{feynman}
			\diagram[small,horizontal=o1 to o4]{
			p1 -- o1 --[draw=none] o4 -- p3;
			p2 -- o1 --[draw=none] o4 -- p4;
			};
			\diagram*{
			(o1) --[half left,looseness=1.7] (o4);
			(o1) --[half right,looseness=1.7] (o4);
			};
		\end{feynman}
	\end{tikzpicture}
	\\&+
	\begin{tikzpicture}[baseline=(o3.base)]
		\begin{feynman}
			\diagram[small,horizontal=o3 to o4]{
			p1 -- o1 --[draw=none] o3 --[draw=none] o4 --[draw=none] o6 -- p3;
			p2 -- o1 --[draw=none] o3 --[draw=none] o4 --[draw=none] o6 -- p4;
			};
			\diagram*{
			(o1) --[half left,looseness=1.7] (o3);
			(o1) --[half right,looseness=1.7] (o3);
			(o3) --[half left,looseness=1.7] (o4);
			(o3) --[half right,looseness=1.7] (o4);
			};
			\node[blob, minimum size=25pt] at ($(o4)!0.5!(o6)$) (o5);
		\end{feynman}
	\end{tikzpicture}+
	\begin{tikzpicture}[baseline=(o3.base)]
		\begin{feynman}
			\diagram[small,horizontal=o3 to o4]{
			p1 -- o3[crossed dot, minimum size=8pt] --[draw=none] o4 --[draw=none] o6 -- p3;
			p2 -- o3 --[draw=none] o4 --[draw=none] o6 -- p4;
			};
			\diagram*{
			(o3) --[half left,looseness=1.7] (o4);
			(o3) --[half right,looseness=1.7] (o4);
			};
			\node[blob, minimum size=25pt] at ($(o4)!0.5!(o6)$) (o5);
		\end{feynman}
	\end{tikzpicture}
\end{align*}
with one-loop counterterm. 

The iterative form is
\begin{align}
	\begin{tikzpicture}[baseline=(o1.base)]
		\begin{feynman}
			\diagram[small,horizontal=p1 to p3]{
				p1 -- o1[blob, minimum size=30pt] -- p3;
				p2 -- o1 -- p4;
			};
			% \node[blob, minimum size=30pt] at (o1);
		\end{feynman}
	\end{tikzpicture} & =
	\begin{tikzpicture}[baseline=(o3.base)]
		\begin{feynman}
			\diagram[small,layered layout,horizontal=p1 to p3]{
				p1 -- o3 -- p3;
				p2 -- o3 -- p4;
			};
			\node at (o3);
		\end{feynman}
	\end{tikzpicture}+
	\begin{tikzpicture}[baseline=(o3.base)]
		\begin{feynman}
			\diagram[small,layered layout,horizontal=p1 to p3]{
				p1 -- o3 -- p3;
				p2 -- o3 -- p4;
			};
			\node[crossed dot] at (o3);
		\end{feynman}
	\end{tikzpicture}+
	\begin{tikzpicture}[baseline=(o1.base)]
		\begin{feynman}
			\diagram[small,horizontal=o1 to o4]{
			p1 -- o1 --[draw=none] o4 -- p3;
			p2 -- o1 --[draw=none] o4 -- p4;
			};
			\diagram*{
			(o1) --[half left,looseness=1.7] (o4);
			(o1) --[half right,looseness=1.7] (o4);
			};
		\end{feynman}
	\end{tikzpicture}
	\\&+
	\begin{tikzpicture}[baseline=(o3.base)]
		\begin{feynman}
			\diagram[small,horizontal=o3 to o4]{
			p1 -- o1 --[draw=none] o3 --[draw=none] o4 -- p3;
			p2 -- o1 --[draw=none] o3 --[draw=none] o4 -- p4;
			};
			\diagram*{
			(o1) --[half left,looseness=1.7] (o3);
			(o1) --[half right,looseness=1.7] (o3);
			(o3) --[half left,looseness=1.7] (o4);
			(o3) --[half right,looseness=1.7] (o4);
			};
		\end{feynman}
	\end{tikzpicture}+
	\begin{tikzpicture}[baseline=(o3.base)]
		\begin{feynman}
			\diagram[small,horizontal=o3 to o4]{
			p1 -- o1 --[draw=none] o3 --[draw=none] o4[crossed dot, minimum size=8pt] -- p3;
			p2 -- o1 --[draw=none] o3 --[draw=none] o4 -- p4;
			};
			\diagram*{
			(o1) --[half left,looseness=1.7] (o3);
			(o1) --[half right,looseness=1.7] (o3);
			(o3) --[half left,looseness=1.7] (o4);
			(o3) --[half right,looseness=1.7] (o4);
			};
		\end{feynman}
	\end{tikzpicture}+
	\begin{tikzpicture}[baseline=(o3.base)]
		\begin{feynman}
			\diagram[small,horizontal=o3 to o4]{
			p1 -- o1 --[draw=none] o3 --[draw=none] o4 --[draw=none] o6 -- p3;
			p2 -- o1 --[draw=none] o3 --[draw=none] o4 --[draw=none] o6 -- p4;
			};
			\diagram*{
			(o1) --[half left,looseness=1.7] (o3);
			(o1) --[half right,looseness=1.7] (o3);
			(o3) --[half left,looseness=1.7] (o4);
			(o3) --[half right,looseness=1.7] (o4);
			(o4) --[half left,looseness=1.7] (o6);
			(o4) --[half right,looseness=1.7] (o6);
			};
		\end{feynman}
	\end{tikzpicture}\notag
	\\&+
	\begin{tikzpicture}[baseline=(o3.base)]
		\begin{feynman}
			\diagram[small,horizontal=o3 to o4]{
			p1 -- o3[crossed dot, minimum size=8pt] --[draw=none] o4 -- p3;
			p2 -- o3 --[draw=none] o4 -- p4;
			};
			\diagram*{
			(o3) --[half left,looseness=1.7] (o4);
			(o3) --[half right,looseness=1.7] (o4);
			};
		\end{feynman}
	\end{tikzpicture}+
	\begin{tikzpicture}[baseline=(o3.base)]
		\begin{feynman}
			\diagram[small,horizontal=o3 to o4]{
			p1 -- o3[crossed dot, minimum size=8pt] --[draw=none] o4[crossed dot, minimum size=8pt] -- p3;
			p2 -- o3 --[draw=none] o4 -- p4;
			};
			\diagram*{
			(o3) --[half left,looseness=1.7] (o4);
			(o3) --[half right,looseness=1.7] (o4);
			};
		\end{feynman}
	\end{tikzpicture}+
	\begin{tikzpicture}[baseline=(o3.base)]
		\begin{feynman}
			\diagram[small,horizontal=o3 to o4]{
			p1 -- o3[crossed dot, minimum size=8pt] --[draw=none] o4 --[draw=none] o6 -- p3;
			p2 -- o3 --[draw=none] o4 --[draw=none] o6 -- p4;
			};
			\diagram*{
			(o3) --[half left,looseness=1.7] (o4);
			(o3) --[half right,looseness=1.7] (o4);
			(o4) --[half left,looseness=1.7] (o6);
			(o4) --[half right,looseness=1.7] (o6);
			};
		\end{feynman}
	\end{tikzpicture}
\end{align}
It'd appeared that we missed one diagram during the iteration: 
\begin{align*}
	\begin{tikzpicture}[baseline=(o3.base)]
		\begin{feynman}
			\diagram[small,horizontal=o3 to o4]{
			p1 -- o3 --[draw=none] o4[crossed dot, minimum size=8pt] -- p3;
			p2 -- o3 --[draw=none] o4 -- p4;
			};
			\diagram*{
			(o3) --[half left,looseness=1.7] (o4);
			(o3) --[half right,looseness=1.7] (o4);
			};
		\end{feynman}
	\end{tikzpicture}
\end{align*}
If we add this one, the r.h.s. of the integral equation won't be finite again. Thus, only the counterterm from one loop is not sufficient enough to cancel all the divergences. 
\end{document}