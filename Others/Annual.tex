    % !TEX program = lualatex
\documentclass[8pt]{beamer}

    \usetheme{Madrid}
    \usecolortheme{crane}
    \linespread{1.5}
    
    \usepackage{graphicx}
    %\usepackage{xcolor}
    \usepackage{amsmath}
    \usepackage{physics}
    \usepackage{hyperref}
    \usepackage{slashed}
    \usepackage{siunitx}
    \usepackage{graphicx}
    %\usepackage{enumitem}
    \usepackage{tikz-feynman}
    \usepackage{comment}
    
    \newcommand{\phanitem}{\phantom{\item}}


    \newcommand{\g}{\gamma}
    \renewcommand{\a}{\alpha}
    \renewcommand{\b}{\beta}
    \renewcommand{\t}{\theta}
    \newcommand{\la}{\lambda}
    \newcommand{\p}{\phi}
    \newcommand{\ka}{\kappa}
    \newcommand{\vp}{\varphi}
    \newcommand{\s}{\sigma}
    \newcommand{\G}{\Gamma}
    \newcommand{\lag}{\mathcal{L}}


\title{Meson-meson scattering in 1+1 Dimension% \\and \\Divergence of Klein-Gordon Hydrogen Atom Wave-function Near origin
}
\author{Yingsheng Huang}
\institute{Institute of High Energy Physics
%\\ \footnotesize (In collaboration with Yu Jia and Rui Yu)
}
\begin{document}
\maketitle
\begin{frame}
	\frametitle{Content}
	\tableofcontents


\end{frame}

\section{1+1-d QCD and 't Hooft model }
\begin{frame}
	\frametitle{\insertsectionhead}
	Lagrangian:
	\begin{eqnarray}
		\mathcal{L}=-\frac{1}{4}G_{\mu\nu}\ _{i}^{\ j}G^{\mu\nu}\ _{j}^{\
		i}+\bar q^{a i}(i\gamma^\mu D_\mu-m_a)q_i^a,
	\end{eqnarray}
	where
	\begin{eqnarray}
		G_{\mu\nu}\ _{i}^{\ j}&=&\partial_{\mu} A_{i}^{\ j}\
		_{\nu}-\partial_\nu A_{i}^{\ j}\ _{\mu}+i g[A_\mu,A_\nu]_{i}^{\
				j},\nonumber\\
		D_\mu q_i^a&=&\partial_\mu q_i^a+ig A_i^{\ j}\ _\mu q_j^a,\nonumber\\
		i,j&=&1,2,...,N_c, \ \ \ \ a=1,2,...,N_f.
	\end{eqnarray}
	Choose light-cone gauge condition
	\begin{equation}
		A_{-}=A^{+}=0,
	\end{equation}
	where
	$A_{-}=\frac{1}{\sqrt{2}}(A^0+A^1)=\frac{1}{\sqrt{2}}(A_0-A_1)$.

	Using Dyson-Schwinger equation and Bethe-Salpeter equation in large $N_c$ limit we obtain the famous 't Hooft equation
	\begin{equation}
		\mu^2
		\varphi(x)=(\frac{\alpha_{1}}{x}+\frac{\alpha_{2}}{1-x})\varphi(x)-P\int_0^1
		dy\frac{\varphi(y)}{(x-y)^2}.\label{teq}
	\end{equation}
\end{frame}

\section{Meson-meson scattering amplitude}
\begin{frame}
	\frametitle{Meson-meson scattering amplitude (Gou-ying Chen and Rui Yu)}
	The Bethe-Salpeter equation for the quark-antiquark scattering amplitude can be written as
	\begin{eqnarray}
		\mathcal{T}(p,p^\prime;r)=-\frac{ig^2}{(p_{-}-p_{-}^\prime)^2}
		+i4N_c
		g^2\int\frac{d^2k}{(2\pi)^2}\frac{1}{(k_{-}-p_{-})^2}\tilde{S}(k)\tilde{S}(k-r)\mathcal{T}(k,p^\prime;r),\label{quarkeq}
	\end{eqnarray}
	\begin{figure}[hbt]
		\begin{center}
			% Requires \usepackage{graphicx}
			\includegraphics[width=4cm]{BS.eps}\\
			\caption{The Bethe-Salpeter equation of the bound state. Arrow lines are dressed quark propagators.}\label{BSequation}
		\end{center}
	\end{figure}
    where $\tilde{S}(p)\gamma_+=S(p)$. This equation has been solved (Callan, Coote and Gross, 1975) and the result is
    \small
	\begin{eqnarray}
		\mathcal{T}(x,x^\prime;r)
		&=&-\frac{ig^2}{r_{-}^2(x-x^\prime)^2}+\sum_{n}\frac{i}{r^2-r_{n}^2}\left\{\varphi_{n}(x)\frac{g^2}{|r_{-}|}
		\sqrt{\frac{N_c}{\pi}}\left[\theta(x(1-x))\frac{2|r_{-}|}{\lambda}+\frac{\alpha_1}{x}+\frac{\alpha_2}{1-x}-\mu_{n}^2\right]\right\}\nonumber\\
		&&\times\left\{\varphi_n^{\ast}(x^\prime)\frac{g^2}{|r_{-}|}\sqrt{\frac{N_c}{\pi}}
		\left[\theta(x^\prime(1-x^\prime))\frac{2|r_{-}|}{\lambda}+\frac{\alpha_1}{x^\prime}+\frac{\alpha_2}{1-x^\prime}-\mu_n^2\right]\right\},\label{qqamp}
    \end{eqnarray}
    \normalsize
\end{frame}

\begin{frame}
	For process $A(q^a\bar q^b)+B(q^c\bar q^a)\rightarrow C(q^a\bar
		q^b)+D(q^c\bar q^a)$ (where $a,b,c$ are different flavor indexes), the amplitude reads
	\begin{align*}
		  & i\mathcal{M} =(1+\mathcal{C})i\mathcal{M}_0,\nonumber          \\
		  & i\mathcal{M}_0 =\theta(\omega_2-\omega_1)i4g^2\omega_1\int_0^1
		dx\int_0^1
		dy\frac{1}{(y\omega_1-\omega_2-x)^2}\varphi_A(\frac{\omega_2-\omega_1+x}{\omega_2-\omega_1+1})\varphi_B(y)
		\varphi_C(x)\varphi_D(\frac{y\omega_1}{\omega_2}),\nonumber
	\end{align*}
	where
	\begin{equation}
		\omega_1=\frac{r_{B-}}{r_{C-}},\ \ \ \omega_2=\frac{r_{D-}}{r_{C-}}.
	\end{equation}
	Here and in the following, we define the operation
	$(A\leftrightarrow C,\ \ B\leftrightarrow D,\ \ \omega_1\rightarrow
		\frac{\omega_2}{1+\omega_2-\omega_1},\ \ \omega_2\rightarrow
		\frac{\omega_1}{1+\omega_2-\omega_1})$ as $\mathcal{C}$. One can
	find that the final expression is infra-red safe, thus we postpone
	$\lambda\rightarrow 0$ in our final expression.
	$A(q^a \bar q^b)+B(q^b\bar
		q^a)\rightarrow C(q^a\bar q^b)+D(q^b\bar q^a) $ reads
	\begin{equation}
		i\mathcal{M}=(1+\mathcal{P})(1+\mathcal{C})i\mathcal{M}_{0}.
	\end{equation}
	where the operation $\mathcal{P}$ is defined as
	$\mathcal{P}=(A\leftrightarrow B,\ \ C\leftrightarrow D,\ \
		\omega_1\rightarrow \frac{1+\omega_2-\omega_1}{\omega_2},\ \
		\omega_2\rightarrow \frac{1}{\omega_2})$.
\end{frame}

\begin{frame}
	$A(q^a \bar
		q^a)+B(q^a\bar q^a)\rightarrow C(q^a\bar q^a)+D(q^a\bar q^a) $ reads
	\begin{eqnarray}
		i\mathcal{M}
		=(1+\mathcal{R})(1+\mathcal{P})(1+\mathcal{C})i\mathcal{M}_{0}+(1+\mathcal{R})i\mathcal{M}_{1},
	\end{eqnarray}
    where
    \small
	\begin{align*}
        i&\mathcal{M}_{1}\\
        &=-(1+\mathcal{Q})\theta(1-\omega_1)i4g^2\int_0^1
		dx P\int_0^1
		dy\frac{\omega_1\omega_2}{[(y-1)\omega_1+(1-x)\omega_2]^2}\varphi_A(\frac{x\omega_2}{1+\omega_2-\omega_1})\varphi_B(y)
		\varphi_C(y\omega_1)\varphi_D(x)\nonumber\\
		&-(1+\mathcal{C})\theta(\omega_2-\omega_1)i4g^2\int_0^1dx P\int_0^1
		dy\frac{\omega_1}{(y\omega_1-x)^2}\varphi_A(\frac{x+\omega_2-\omega_1}{1+\omega_2-\omega_1})\varphi_B(y)
		\varphi_C(x)\varphi_D(\frac{(y-1)\omega_1+\omega_2}{\omega_2})\nonumber\\
		&
		-(1+\mathcal{Q}+\mathcal{P}+\mathcal{C})\theta(\omega_2-\omega_1)\theta(\omega_1-1)i\frac{4\pi}{N_c}\int_0^1
		dx\left[2r_{C+}r_{C-}+2r_{D+}r_{C-}+\frac{M_a^2}{x-\omega_1}+\frac{M_a^2}{x-1}\right.\nonumber\\
		&\left.-\frac{M_a^2}{x-\omega_1+\omega_2} -\frac{M_a^2}{x}\right]
		\times\varphi_A(\frac{x-\omega_1+\omega_2}{1+\omega_2-\omega_1})\varphi_B(x/\omega_1)
		\varphi_C(x)\varphi_D(\frac{x-\omega_1+\omega_2}{\omega_2}),\nonumber
    \end{align*}
    \normalsize
	and
	\begin{align}
		&\mathcal{R}=(C\leftrightarrow D,\ \ \omega_1\rightarrow
		\frac{\omega_1}{\omega_2},\ \ \omega_2\rightarrow 1/\omega_2),\ \ \ \nonumber\\
		&\mathcal{Q}=(B\leftrightarrow C,\ \ A\leftrightarrow D,\ \
		\omega_1\rightarrow 1/\omega_1,\ \ \omega_2\rightarrow
		\frac{1+\omega_2-\omega_1}{\omega_1}).
	\end{align}
\end{frame}

\section{Numerical Calculation}
\begin{frame}
    \frametitle{Numerical Calculation (Yingsheng Huang)}
\end{frame}

\section{Conclusion}
\begin{frame}
    \frametitle{\insertsectionhead}
\end{frame}

\begin{comment}
\section{Divergence in Relativistic Quantum Mechanics}
\begin{frame}
	\frametitle{\insertsectionhead}
	\begin{minipage}{0.43\linewidth}
		Ground state Klein-Gordon Wave-function with Coulomb potential:
		\begin{align}
			\psi  =\frac{c}{\sqrt{4\pi}}e^{-kr}r^\lambda
		\end{align}
		where
		\begin{align*}
			  & \lambda=-\frac{1}{2}+\sqrt{\frac{1}{4}-Z^2\alpha^2},\; \\&  c=\sqrt{\frac{(2k)^{2(1+\sqrt{\frac{1}{4}-Z^2\alpha^2})}}{\Gamma(2+2\sqrt{\frac{1}{4}-Z^2\alpha^2)})}},\\&    k=\frac{m}{\sqrt{1+\frac{(\frac{1}{2}+\sqrt{\frac{1}{4}-Z^2\alpha^2})^2}{Z^2\alpha^2}}}
		\end{align*}
		expand over $Z\a$ we got logarithmic divergence.
		$$R(r)          \sim-(Z\alpha)^2\log(2m Z \a r)   $$
	\end{minipage}
	\begin{minipage}{0.5\linewidth}
		\begin{figure}
			\centering
			\includegraphics[width=2.8 in]{K-G-fig.pdf}
			\caption{Comparison between Klein-Gordon wavefunction and Schr\"odinger wavefunction, with parameters set at $Z=1$, $\a=0.2$, $m=1$. }
		\end{figure}
	\end{minipage}
\end{frame}

\begin{frame}
	\frametitle{Quantum Mechanics perturbation theory}
	This behavior can be reproduced by simple perturbation theory, with extra power divergence.
	\begin{align}
		H & =H_0+H_{int} ,\;\;H_{int}=H_{kin}+H_{Darwin}+\mathcal{O}(v^6)                                                                                          \nonumber \\
		H & _0=-\frac{\nabla^2}{2m}-\frac{Z\alpha}{r},\ \ \ H_{kin}=\frac{\nabla^4}{8m^3},\ \ \ H_{Darwin}=\frac{1}{32m^4}[-\nabla^2,[-\nabla^2,-\frac{Z\alpha}{r}]]
	\end{align}
	The NLO correrction to wave-function is
	\begin{align}
		\phi^{(1)}=\sum_{n\neq 1}a_{n1}\phi_{n00}^{(0)}+\int d\ka a_{\ka 1}\phi_{\ka00}^{(0)}
	\end{align}
	where $\kappa=\frac{\abs{\vb{k}}}{m Z \a}$ and $\phi_{nlm}^{(0)}$, $\phi_{\ka lm}^{(0)}$ are Schr\"odinger wave-functions in bound state and scattering state.

	The scattering part would cause a divergence when integrating over very-high momentum states, by introducing a hard cutoff $\Lambda$ we can regularize it (Darwin term won't contribute)
	\begin{align}
		R^{(1)}(0)_{kin} & =\int^\frac{\Lambda}{m}d\ka(Z\alpha)^2(\frac{1}{\pi}+\frac{1}{\ka}) \\
		                 & \sim(Z\alpha )^2(\frac{\Lambda}{\pi m}+\log(\frac{\Lambda}{m}))
	\end{align}
\end{frame}

\section{Scalar NRQED and HQET}
\begin{frame}
	\frametitle{\insertsectionhead}
	Consider this problem in QFT.

	Describe the nucleus with HQET and the scalar electron with NRQED.

	The Lagrangian is
	\begin{align}
		\lag_{SNRQED+HQET}= & \varphi^*\bqty{iD_0+\frac{\nabla^2}{2m}+\frac{\nabla^4}{8m^3}+\frac{\nabla^6}{16m^5}+\frac{e}{32m^4}\pqty{[\nabla^2,[A_0,\nabla^2]]+[A_0,\nabla^4]}+\mathcal{O}(v^7)}\varphi\nonumber \\&+N^*iD^0N
	\end{align}
	where
	\begin{align*}
		D_{\mu}\varphi=\partial_{\mu}\varphi+ieA_{\mu}\varphi
	\end{align*}
	and
	\begin{align*}
		D_{\mu}N=\partial_{\mu}N-iZeA_{\mu}N
	\end{align*}
	The Feynman rules are thus straight-forward.
\end{frame}

\section{Reproduce divergence in QFT}
\begin{frame}
	\frametitle{\insertsectionhead}
	When discussing heavy quarkonium problem, non-relativistic Coulomb gauge wavefunction can be defined as NRQCD Bethe-Salpeter $Q\bar Q$ wavefunction, evaluated at equal time (Bodwin, Braaten and Lepage, 1995). Given the similar approximation we made here, we could assume the same.
	\begin{align}
		R(\abs{\vb{x}})\propto\mel{0}{\varphi(\vb{x})N(0)}{{}^1H}.\label{wavefunction}
	\end{align}
	We think that the divergence of wavefunction comes from the non-local attribute of operator $\varphi(x)N(0)$ rather than the state $\ket{{}^1H}$ itself, so it can be analyzed by operator product expansion (OPE). We study
	\begin{align}
		\mel{0}{\varphi(\vb{x})N(0)\tilde\varphi(p)\tilde N(k)}{0}=\int\frac{\dd^4 p}{(2\pi)^4}\frac{\dd^4 k}{(2\pi)^4}e^{-ip\cdot y}e^{-ik\cdot z}\mel{0}{\varphi(\vb{x})N(0)\varphi(y)N(z)}{0}.\label{NLME}
	\end{align}
	The dependence of x in this non-local matrix element can also be taken as a regularization scheme and thus the result should be the same as local one (bare)
	$$\mel{0}{\varphi(0)N(0)\tilde\varphi(p)\tilde N(k)}{0}$$
\end{frame}

\begin{frame}
	\frametitle{Local operators: $\mel{0}{\varphi(0)N(0)\tilde\varphi(p)\tilde N(k)}{0}$}
	We do this with dimensional regularization so that all unphysical power divergence will be removed.

	LO is trivial.

	At NLO, there's no logarithmic divergence. (Gamma function has no pole in half-integer points. )
\end{frame}

\begin{frame}
	\frametitle{Non-local operators: $\mel{0}{\varphi(\vb{x})N(0)\tilde\varphi(p)\tilde N(k)}{0}$}
\end{frame}

\begin{frame}
	\frametitle{Renormalization Group Equation}
	We can resum all the logarithms and retain the $r^{Z^2\a^2}$ form. (TBD)

	This process can be done with renormalization group equation, which is the common way to resum large logarithms, i.e. in SCET. Or we can naively calculate and sum all the logarithms the hard way.
\end{frame}

\section{Conclusion}
\begin{frame}
	\frametitle{\insertsectionhead}
\end{frame}
\end{comment}

\end{document}