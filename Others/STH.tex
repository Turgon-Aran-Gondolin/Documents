%!mode::"Tex:UTF-8"
\PassOptionsToPackage{unicode}{hyperref}
\PassOptionsToPackage{naturalnames}{hyperref}
%\documentclass[cs4size]{ctexart}
\documentclass{article}
%\usepackage{luatexja-fontspec}
%\setmainjfont{SimSun}
\usepackage{fullpage}
%\usepackage{geometry}
\usepackage{parskip}
\usepackage{physics}
\usepackage{amsmath}
\usepackage{amssymb}
\usepackage{xcolor}
%\usepackage[colorlinks,urlcolor=red,citecolor=green,anchorcolor=blue]{hyperref}
\usepackage{array}
\usepackage{longtable}
\usepackage{multirow}
\usepackage{comment}
\usepackage{graphicx}
\usepackage{cite}
%\usepackage{slashbox}
\usepackage{titlesec}
%\usepackage{intent}
%\usepackage{fancyhdr}
\usepackage{float}
\usepackage{appendix}
\usepackage{indentfirst}
%\usepackage{changepage}
%\usepackage{pdfpages}
\usepackage{luatexja-fontspec}

\setmainjfont[BoldFont=FandolSong-Bold]{FandolSong-Regular}
\setsansjfont{FandolSong-Bold}
\setlength{\parindent}{2em}
\linespread{1.2}

\title{从物理学发展史看国家支持对其发展的影响}
\author{黄应生}
\date{序号:31\qquad 学号:201628000907002}
\begin{document}
\maketitle
提起物理学,人们首先会有一个误解:物理学家的工作就是找片芳草萋萋的绿地,朝着蓝天发会呆,然后潇洒地写下几行简洁优美公式。但是本质上来说,物理学是一门实验科学,无论理论有多么的漂亮、高深,我们仍然要将之与现实对应起来;某种意义上,Einstein、Bohr的时代一去不复返。既然说物理学是一门实验科学,势必需要实验仪器、地点、人员,也就是说,需要经费、资源。但是作为一门基础学科,物理学本身的产出实际上并不高(这点大概一直到近百年才有所改变,但也仅限于凝聚态物理、原子分子物理等研究尺度较大、与现实生活接轨较多的分支),也因此国家(政府、皇室、官方机构)的支持至关重要。

\section{Tycho时代的天文观测}

西方对物理学的研究始于天文观测。说起天文观测,大部分人的第一印象会与望远镜联系起来,但事实上望远镜的出现已经是17世纪的事情了(真正实用的天文望远镜还要等到17世纪末Sir Issac Newton发明第一具反射望远镜)。古老的天文观测,很大程度上依赖于不可靠的肉眼和一些极不精密的测量仪器。尽管测量条件十分简陋,天文学家们还是能够给出粗略的星历表。根据这些粗糙的数据,天文学家们开始对宇宙展开了研究。

公元前4世纪,Eudoxus提出了他的所谓的同心球学说,即以地球为中心,其它星体嵌套多层同心球绕地球运行;而后Aristotle又吸收了Euxodus的思想提出了著名的Aristotle体系。公元前3世纪的Aristarchus(古希腊天文学家、数学家,其贡献集中在圆锥曲线方面,由此可以想象他在天文上的成就)和公元前2世纪的Hipparkhos(Hipparkhos是第一个发明三角函数的人,同时也是第一个将天文仪器的圆周分成360°的希腊人,由此可见当时观测手段之粗陋)创造了所谓的本轮和均轮的描述方式,即行星/恒星本身存在圆周运动,在地球上看就是许多重圆周运动的嵌套,后者(Hipparkhos)甚至建立了著名的Hipparkhos模型,这个时期内,著名的安提凯希拉装置也出现了,这是一种利用齿轮运动来模拟天体运行的装置。公正地说,本轮与均轮的描述方式尽管计算繁复,但是却更加符合地球观测者的角度,并且,某种意义上说只是选取了一种不直观的参考系;而其计算工作也为之后的一些研究提供了帮助,比如理想的圆偏振引力波的时间演化某种意义上也符合本轮均轮模型。这种模型在Ptolemy的年代被发扬光大——Ptolemy体系,但是其繁重的计算终于使得哥白尼在约一千四百年后提出了日心说。

总的来说,这还是一个很美好的时代:由于原始的数学工具以及低下的观测精度,物理/天文学家们并不需要太多资源就可以进行研究(当然由于当时知识传播的成本较高,研究本身是有阶级限制的,并且诸如安提凯希拉装置这样的“精密”装置也并非轻易就能设计制造),但是尽管如此,当时建立的模型仍然远远达不到符合实验数据的要求。但是一千年过去了,Tycho Braho(1546 \textasciitilde 1601)终于对天文学界漫不经心的测量不满了,他决定改变这一现状。Tycho是丹麦贵族,其家庭是丹麦有数的几个主要贵族家庭之一。他的家庭对他的研究工作提供了很多助力:首先,Tycho早年接受了良好的教育,他先就读于哥本哈根大学,之后又相继在莱比锡大学、罗斯托克大学学习(正是在后者就读期间Tycho失去了他的鼻子);其次,他的家世使得丹麦国王腓特烈二世为他提供了一块领地——汶岛以及相应资金以供他建立天文台,Tycho说明的建立第一座天文台——天堡的费用是当时的17000磅,按生活费用换算相当于现在的200万美金。这笔钱对当时的丹麦来说并不轻松,并且这部分支出也并非一劳永逸,可以说丹麦当时一年的国民生产总值有很大一部分要供给Tycho的天文台;这是之后Tycho的薪金和地产被新任丹麦国王剥夺的一部分原因。Tycho同时拥有当时最精确的角度观测设备——大浑仪,精确度能达到月亮直径的百分之一,仅次于此后的望远镜。拥有这么庞大的资源,Tycho也做出了前人所未能及的成果:他为777颗恒星编制了高精度(当时最高精度)的星表,以2角分的精度对太阳、行星等进行了20多年的观测,建立了可信的蒙气差改正表(这是为了对大气折射进行修正)等等。他最大的贡献,在我看来,是树立了一套所谓的“长期、系统的精密观测”的实验观念,对后来者有很大启迪意义。

\section{体系化的物理学发展}

真正成体系的物理学发展,我个人认为要等到17世纪物理学革命的开始。总的来说,自Tycho对实验理念的变革之后,西方的物理学开始了真正的发展。这一过程始于Johannes Kepler。事实上,严格来说Kepler之前的天文学和物理学并没有交叉,一直到Kepler对星体运行机制的研究,才真正促成了天体物理学这一分支的产生,所以某种意义上说,Kepler可以说是第一位天体物理学家。当然Kepler本人的研究更趋向于数学、模型化的研究,这与他曾为之效力的Tycho有着明显的区别。同时,伽利略也提出了惯性定律,证明了加速度的存在。此后,承接于Kepler的行星运动三大定律,Newton的力学定律以及万有引力真正让天上的动力学和地上的力学\cite{1}联系了起来。在之后的几世纪里,物理学发现如井喷一般涌现。

在这个时代,科学的组织性也渐渐体现了出来\cite{1}。事实上,这个时期很少存在现在所谓的“大科学项目”,实验的成本相对较低。我们回头看中世纪的科学活动:由于中世纪的知识基本局限于(教会)大学,科学也被限制在大学的围墙之中;但是在受到大学精英教育裨益的同时,教会由于其本身立场某种意义上也限制了科学的发展方向,新潮的思想很可能会因为与教义相冲突或被解释为与教义相冲突而遭到扼制,典型的例子就是Copernicus的日心说(尽管那时已是文艺复兴末期,教会权威被极大程度地削弱,但是由此我们更可以看出真正中世纪教会权威下新思想萌芽的困难,但是这件事本身存在许多争议,也许存在一些内情,在此不再赘述细节)。到了17世纪,近代科学萌芽,而(教会)大学并不乐见其成。很自然地,科学家们开始另寻他路,类似Galileo这种成了“御用”数学家(托斯卡纳大公数学家)的不在少数。就这一点来说,Newton是例外,他一直在剑桥大学任教,尽管如此,剑桥并未给他的研究工作以任何支持。在当时,尽管几乎每位著名科学家都是大学培养出来的,但是科学却并未进入大学,大学课程仍然十分陈旧。相应的,科学界也开始组织建立各种学会,始于非官方组织(最早可以说是罗马的“山猫学会”),但是在60年代初,法国开始组建官方组织——皇家科学院。相比于英国同行——英国皇家学会,法国皇家科学院真正得到了法国政府的支持,也因此有了一些重要的成绩,比如对地球表面每分弧长度的测量,从而确定地球的大小,达到了前所未有的精度,对南美洲的考察有助于确定地球到火星的距离,并且间接测定了太阳系大小。这些项目可以说是当时的“大科学项目”了,绝不是单个科学家能够完成的。皇家科学院的缺点在于,尽管某种意义上它是在为后人铺路,但是作为一个政府部门,当时科学院基本充当了专利局的角色,过多的琐事最终严重影响了法国科学的发展,建立科学院10年后,欧洲科学界的领头羊便由法国成了英国。

\section{现代物理学的反思}

尽管人类历史悠悠几千年,现代物理学的发展史也有数百年(自Galileo始),但是物理学在近百年内的发展却足以令任何古人瞠目结舌。量子理论以及相对论(这里所说的不仅仅是量子力学或者任何一种相对论,而是泛指所有进行了量子化的理论、狭义相对论、广义相对论及它们的衍生理论)给物理学带来的不仅是理论上的突破,还有实验目的的改变以及实验手段的变革,但是相应地,也带来了实验成本的提高。举例来说,16世纪对力学的研究,或许只需要两个铁球就可以,但是20世纪初一根阴极射线管的成本就远远超过两个铁球了,Rutherford实验装置的造价远非之前的力学、热学实验所能比较,当然在今日由于工业化生产的原因这些设备的成本被大大压低,但是真正适应于我们这个时代的设备成本却远远不是现代工业水平所能降低的。在这种情况下,曾经举家之力就能完成研究的传奇已然不再,政府对物理学发展的支持日趋重要。

大概是有了法国皇家科学院的教训,现在的国立研究机构已经与专利局这种完全的政府机构脱离开来,但是由于20世纪之后科技在经济、军事领域的应用愈发重要,这也就使得国家难免对科学界的研究方向有所干预,这种干预主要体现在经费侧重上。举例来说,核能工业是20世纪中叶才兴起的新兴产业,但是在核能开发的初期,曼哈顿计划却是完全侧重于核武器开发的,所有民用原子能的计划全部需要让道\cite{3},但在宣告和平之后,美国海军也进入了核能开发行列,他们需要小型核反应堆以建造核动力推进系统,也因此核反应堆的研究受到美国海军的欢迎。核能工业是人类的必经之路,尚且有此一番曲折,这不禁让人寻思,究竟有多少研究,因为不是政府所必须而陷入缺乏经费的窘境,最终关门了事。

另一个问题是,真正功非当下、利在长远的项目需要有长远目光的政府的支持,这类项目很可能几十年都看不到直接成果,却是后来者研究的必然积淀。但是这个问题需要结合具体情况分析,若贸然决定,难免陷入法国皇家科学院一般的窘境。但有一点确定无疑:无论是何种科学,何种项目,真正需要进行决定的始终必须是少数的精英阶层,必须要对问题有真正深入全面的了解才有资格作出评判。

\begin{thebibliography}{}
  \bibitem{1}
理查德. S. 韦斯特福尔. 近代科学的建构: 机械论与力学[J], 2000, 119.
\bibitem{2}
彼得, 迈克尔, 哈曼, 等. 19世纪物理学概念的发展[J]. 19.
\bibitem{3}
乔治, 巴萨拉. 技术发展简史[J]. 2000.
\end{thebibliography}


\end{document}
