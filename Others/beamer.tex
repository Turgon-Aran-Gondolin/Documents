\documentclass[8pt]{beamer}

\usetheme{Madrid}
\usecolortheme{crane}
\linespread{1.5}

\usepackage{graphicx}
%\usepackage{xcolor}
\usepackage{amsmath}
\usepackage{physics}
\usepackage{hyperref}
\usepackage{slashed}
\usepackage{siunitx}
%\usepackage{enumitem}

\newcommand{\phanitem}{\phantom{\item}}

\title{Anthropic Principle in Modern Physics}
\author{Yingsheng Huang}
\institute{Institute of High Energy Physics}
\begin{document}
\maketitle
\begin{frame}
  \frametitle{Content}
  \tableofcontents
\end{frame}
\begin{frame}
  \frametitle{Questions}
  What would this world be like, naively speaking, if the following things are changed?


  \begin{itemize}
    \item Planck length
    \phanitem
    \item Fine-structure constant
    \phanitem
    \item Gravitational constant
    \phanitem
    \item Speed of light\\
    $\Longrightarrow$the permittivity and permeability of the vacuum
  \end{itemize}
\end{frame}
\section{Origin of anthtropic principle}
\begin{frame}
  \frametitle{Origin of anthropic principle: Cosmology}
  \begin{itemize}
    \item The anthropic principle (from Greek anthropos, meaning ``human") is the philosophical consideration that observations of the Universe must be compatible with the conscious and sapient life that observes it.
    \phanitem
    \item Brandon Carter first describe it in 1973 as the form of privileged observers. (Sould we give the observers -- us privileged position in the vast universe?)
    \phanitem
    \item ``The argument can be used to explain why the conditions happen to be just right for the existence of (intelligent) life on the Earth at the present time. For if they were not just right, then we should not have found ourselves to be here now, but somewhere else, at some other appropriate time." -- -- Roger Penrose
  \end{itemize}
\end{frame}

\section{Variants of anthropic principle}
\begin{frame}
  \frametitle{Variants of anthropic principle}
  \begin{itemize}
    \item Weak anthropic principle (WAP) (Barrow and Tipler): "The observed values of all physical and cosmological quantities are not equally probable but they take on values restricted by the requirement that there exist sites where carbon-based life can evolve and by the requirements that the universe be old enough for it to have already done so."
    \item Strong anthropic principle (SAP) (Barrow and Tipler): "The Universe must have those properties which allow life to develop within it at some stage in its history."
%The Latin tag ("I think, therefore the world is such [as it is]") makes it clear that "must" indicates a deduction from the fact of our existence; the statement is thus a truism.
    \item Modified anthropic principle (MAP) (Schmidhuber): The 'problem' of existence is only relevant to a species capable of formulating the question. Prior to Homo sapiens intellectual evolution to the point where the nature of the observed universe -- and humans' place within same-spawned deep inquiry into its origins, the 'problem' simply did not exist.
    \item Strong self-sampling assumption (SSSA) (Bostrom): "Each observer-moment should reason as if it were randomly selected from the class of all observer-moments in its reference class."
  \end{itemize}
\end{frame}

\section{Parametrized universe}
\begin{frame}
  \frametitle{Parametrized universe}
  \begin{itemize}
    \item The gravitational constant, the mass of the proton, the age of the universe, etc.
    \item In high energy physics: fine-structure constant, Planck constant (Planck units)
    \item Elegant theory: QED with only two parameters.
    $$\mathcal{L}=\bar\psi(i\slashed \partial-m)\psi-\frac{1}{4}F_{\mu\nu}F^{\mu\nu}-e\bar\psi\gamma^{\mu}\psi A_{\mu}$$
    \item Sometimes we call it rubbish: SUSY with more than 100 parameters.
    \item Distinguish: Physical parameters apart from parameters in effective theory.
    \item Fine-tuned universe.
  \end{itemize}
\end{frame}

\section{Possible physical interpretation}
\begin{frame}
  \frametitle{Possible physical interpretation}
  \begin{itemize}
    \item Multiverse (apparently the most favored one, by both amateurs and sci-fi writers)\\
    $\Longrightarrow$Some deep senses in Quantum Mechanics
    \phantom{\item }
    \item String theory (Well-known possible grand unify theory but yet unproven)\\
    $\Longrightarrow$Some said in additional dimensions there could be a different universe but I don't approve.\\
    $\Longrightarrow$Huge numbers of possible vacua is another more reasonable explaination.
    \phanitem
    \item The absurd universe/the unique universe/intelligence design/The fake universe/Top-down cosmology (Hawkings)
    \phanitem
    \item Most boring ones: \\
    The life principle: there is an underlying principle that constrains the Universe to evolve towards life and mind.\\The self-explaining universe: only universes with a capacity for consciousness can exist
  \end{itemize}
\end{frame}



\begin{frame}
  \Huge
  \begin{center}
    Thanks
  \end{center}
\end{frame}
\end{document}
