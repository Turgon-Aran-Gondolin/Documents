%!mode::"Tex:UTF-8"
\PassOptionsToPackage{unicode}{hyperref}
\PassOptionsToPackage{naturalnames}{hyperref}
%\documentclass[cs4size]{ctexart}
\documentclass{article}
%\usepackage{luatexja-fontspec}
%\setmainjfont{SimSun}
\usepackage{fullpage}
%\usepackage{geometry}
\usepackage{parskip}
\usepackage{physics}
\usepackage{amsmath}
\usepackage{amssymb}
\usepackage{xcolor}
%\usepackage[colorlinks,urlcolor=red,citecolor=green,anchorcolor=blue]{hyperref}
\usepackage{array}
\usepackage{longtable}
\usepackage{multirow}
\usepackage{comment}
\usepackage{graphicx}
\usepackage{cite}
%\usepackage{slashbox}
\usepackage{titlesec}
%\usepackage{intent}
%\usepackage{fancyhdr}
\usepackage{float}
\usepackage{appendix}
\usepackage{indentfirst}
%\usepackage{changepage}
%\usepackage{pdfpages}
\usepackage{luatexja-fontspec}

\setmainjfont[BoldFont=FandolSong-Bold]{FandolSong-Regular}
\setsansjfont{FandolSong-Bold}
\setlength{\parindent}{2em}
\linespread{1.2}

\title{西学东渐与中国近代科学}
\author{黄应生}
\begin{document}
\maketitle
明末清初是中国历史中的一个十分特殊的时期。这个时期,是中国史上历时最长的外族入侵之开端,是中国资本主义诞生之始,也是西学“入侵”中国,国学开始没落的一个时期。避开“科学”的问题不谈,中国古代显然是存在对自然现象以及数学的研究的。由于种种原因,中国在这方面自成体系。常有人觉得中国古代“玄之又玄”,亦或是儒学阻碍了科学发展,这类观点虽然也有道理,但却失之偏颇。以本课相关的天文为例,在公元前五世纪到公元十世纪这段漫长的时间内,天文现象几乎只有中国的记事可供利用,又比如中国的星官体系,与西方的星座体系相比各有千秋,但是星官体系能够与中国古代的宇宙观相容,可以说是一套自洽的体系\footnote{这一段多少是我个人的主观感受,西方的星座体系尽管也和希腊神话相联系,但是整体体系偏散(个人观点),类似于唯象的核物理,不能收束于一个整体;而星官体系的成熟体三垣二十八宿体系不止将星空进行了划分(这一点显然星座体系也做到了),还对每个星座都进行了功能划分,并与中国对宇宙整体的认知联系了起来,这一点我觉得是星座体系在开普勒之前都未能做到的。};再以数学为例,商高早在公元前十一世纪就提出了著名的勾股定理(相传毕达哥拉斯于公元前六世纪发现并证明了这一定理但是后者并无确切证据,一般认为欧几里得的《几何原本》中首次给出证明)\footnote{题外话,勾股定理中国早于毕达哥拉斯五世纪发现,尽管并未证明,但是还是冠名曰“勾股”,迥异于西方;但是同样是中国人所做出的成就,吴有训所发现证实的康普顿散射却并未在国内称作吴有训散射,确实奇怪。},祖冲之对圆周率的精确计算领先世界千余年。中国古代对“科学”的研究传承已久,却因为西学的传入而逐渐没落,直至今日,已经完全被西方科学所代替,不得不说是一件憾事\footnote{让人遐想,如果明末清初,国学能够继续研究,像张之洞所说的,中学为体,西学为用,吸收西学精华以为己用,如今会是何种情形。这也是我撰写本文的初衷。}。
\section{中国古代“科学”}
在此我特意在“科学”一词上加上引号,是因为对于“中国古代是否存在科学”这一问题的争议。当然如果将科学的定义限制到西方的自然科学体系上去,那么毫无疑问,中国古代不存在科学。但正如上边所提到的,中国古代在天文、地理、数学、医学等领域均有建树,并且自身存在逻辑性,属于成体系的研究。后人对此存在非议,我认为原因有以下几点:

其一,中国古代的科学发展偏于务实,从这种角度上看,或许以“科技”称之更为贴切。正如李约瑟博士所说的所谓“中国所有科学具备经验主义的根本特点”,纵观中国古代的科学发展,无一不缺乏对现象背后本质的探究。例如,中国虽然很正规地进行了数千年的天文观测,但是却从未进行过天体力学或者类似学科的研究;中国对水利、农业这类更加贴近生活的科学更加青睐,而现在所谓的“基础科学”--对现象背后公理化的本质探究则不甚宠爱(或许与中国属于农耕社会有关系)。或者说,即使有探究,也往往落入玄学的桎梏\footnote{这一说法未必中肯,因为古人对于宇宙的认知正是基于这些所谓的“玄学”的,而现在对古代学说的研究很大程度上是在现代科学的框架中将古代学说往里套(对于天文学和数学来说这么做似乎没什么问题,但“术数”的范围却不仅限于数学,“天文”也不仅限于现代的天文学),这种做法究竟正确与否我不敢妄加评判,但是或许这么做未必能够得到古人真意?即使是同样基于西方科学体系的量子理论和相对论的结合也多有不和,两个截然不同的框架是否能这么简单地凑起来呢?或者做个类比,今人理解古代框架,便如一个不通数学而要学习近代物理的人一般,强行以自己的经验代入理解,或许能进行概念(表层)的评论,但是不能触及框架真正的精髓,常常犯错,后者的代表正是所谓“民科”。}。关于这一点,我个人觉得,这是因为中国的天人合一思想(或者类似思想)的影响,使得中国更希望系统全面地研究问题,将所有方面都纳入一个体系之中,比如《黄帝内经》中所谓"上古有真人者",“提挈天地,把握阴阳,呼吸精气,独立守神”,“中古之时,有至人者”,“淳德全道,和于阴阳,调于四时,去世离俗”,“有贤人者”,“法则天地,象似日月,辨列星辰,逆从阴阳,分别四时”,可见天地与人,其理同一,又比如“无,名天地之始;有,名万物之母”,“太极生两仪,两仪生四象,四象生八卦”,“气之清者便为天,为日月,为星辰,只在外,常周环运转”,“轻清者为天,重浊者为地”\footnote{这部分与现代的宇宙观其实有些接近,所以我将之列出,但是其本质又有不同。中国古代的世界、万物观一言两语难以说清,我也理解不透,但是总归是和易、气能联系起来的。},但这么做无形中提高了研究的门槛与难度,或许是后人无法理解,也或许是这套体系本身失之高玄,不得宇宙真谛,无论如何,它在与西方科学体系的较量中败下阵来;暂且不论这么做(大统一)成功与否,起码这是西方科学至今都尚未做到的(生物学与医学和其它学科有很深的割裂,从这个角度上看,生物学与医学同样是李约瑟博士口中的经验科学)。

其二,儒家、道家等思想对科学的影响。对于这一点,我想归结于统治阶级(这种称呼未必恰当,我姑且用之)对科学的影响。西方同样有类似的情况,比如臭名昭著的教会与日心说\footnote{这里提及此事纯因其广为人知。}。但儒家本身其实并不排斥科学技术,《大学》所谓“物格而后知至”,格物乃是治学的第一步,后人往往误解此句,于是便有王阳明格竹子的传说。但是儒学的“格物”可不是对着一根竹子空想,“朱子格物何曾教人格竹”。后人格物不得其法,却不能归咎于儒学。那么儒道诸家\footnote{诸子百家就不一一提及了,有些与题无关如法家,有些则有关如墨家。}本身的思想体系是否适合于科学的发展呢?对于现代科学体系,答案应为否;但对于古代的“科学”,我不敢妄加评论\footnote{我的观点是,我们尚不能完全否定古代世界观、宇宙观(比如中医理论),那么对于一个相对陌生的架构就不应该贸然定论。}。
外族入侵又是另一种情况。凡外族入侵中华,必然忧虑前朝复辟,如元朝便对所有汉人进行残酷统治,汉人沦于社会底层,倒不限于科学了;清朝大约以史为鉴,有所进步,开始吸收中华文化,但是这种吸收是选择性的吸收,比如明朝高度发达的航海技术,再比如火器,都因为各种原因遭到限制,反倒是儒学中的忠君思想再次遭到扭曲放大,有些人或许会将之归结为集权统治的必然性,但是我认为外族统治底气不足的原因也不可忽视\footnote{这种高度强化的忠君思想始于(近似)明朝,但是本身应该说是和朱氏王朝的历史有关的,不能责怪儒学。某种意义上说,朱元璋的统治也显得底气不足。}。

其三,中国古代科学体系本身的缺陷。例如中国数学的非几何性质对天文学的影响。西方天文学得以发展出日心说等目前看来更接近事实的学说,很大程度归功于其几何化的数学,可以更加直观地描述天体运行的规律,易于理解,适合于科学发展的开端\footnote{现在看来不过是参考系选取的问题,而且代数化的表述方式在现代物理中更加流行,易于分析。}。而中国代数性质的数学,某种意义上说确实不适合天文学的初期发展\footnote{这点上,中国的符号也确实略显繁琐。}。
再比如中国偏“玄学”的陈述与理解方式,这一点我在上文也有提及。
最重要的一点,我认为是中国古代物理学研究的缺乏。或者说,相关研究更加趋向于运动学而非动力学。这一表述其实也不甚确切,中国古代似乎有很玄学的动力学表述。但总之不是定量的研究。
\section{西学东渐与其历史背景}
从上文来看,中国古代是存在某种成体系的“科学”的,而西学东渐之时恰值中国传统“科学”衰落之际,事实上,几乎是中国所有传统文化衰落的时期。正所谓“崖山之后无中国,明亡之后无华夏”,且先不论这种论调正确与否,在某些层面上这种说法难以辩驳。经历了元朝的摧残,明朝很多东西都沿革元而并非“复古”,比如行省制度,比如君尊臣卑以至于“直奴仆耳”,比如“匠户”此类户籍制度。后者不利于科学发展尤甚。何为“匠户”?精研技术者也。然有元一朝,掠掳工匠,世代为奴;元亡明兴,竟沿此陋制,对科学的发展实在有害无利。无怪乎李约瑟称明代以后自然科学走向没落。天文与数学是西学东渐的先锋,从天文一行来看,西学之所以能够获得压倒性优势,是因为中国传统历算的误差,归根结底,是中国官方(礼部?钦天监?)不思进取,存在误差不思改进,墨守成规,又不允许民间研究,正所谓“惟是朝征求,士乏讲究,间有草泽遗逸,通经知算之士,留心历理者,又皆独学寡助,独智师心,管窥有限”。国学既衰,西学便趁虚而入。于是贪图便利,一味讲求西学,从《崇祯历书》到《西洋新法历书》,“尽堕成宪而专用西法如今日者”,直至清朝国力略有缓和,王锡阐、梅文鼎等人对会通中西作出努力。

详细梳理这段历史,我们可以发现,西学东渐之时,正是中原最混乱的时期,也是中国(无论是明还是清)最孱弱的时期。崇祯二年(1629年),由于日食预测不准,徐光启开始了西法历的修订。但是满族之祸日重,同年12月,皇太极率军入关,直捣北直隶。1631年,大凌城沦陷,直接导致山东孔有德叛乱。1640年代,东北战火如火如荼。同时整个崇祯年间造反始于陕西,却一直没有平定过;相反的,造反(起义)潮愈演愈烈,逐渐蔓延全国,直至1644年北京沦陷,崇祯自缢身亡。而明朝的财政支柱--东南也并不平静,1634年初南直隶的桐城县同样发生造反,高税率使得东南不堪重负\footnote{依剑桥中国史原文来看,税率本身对于东南应不高(东南商业发达,而明朝商业税极低),主要在于用于缴税的白银难以得到。这一点大概要归咎于一条鞭法。对于其它地区如西北而言,过高的农业税早已使得地方不堪重负了。},日本停止往中国输入白银,导致银价猛涨,通货紧缩,崇祯不得不加税,恶性循环。加之朝内政治斗争激烈,大学士温体人忙于排除异己,最终还是在1637年倒台,而后杨嗣昌当政,却又遭到东林党攻讦。《崇祯历书》所幸于1634年撰成,躲过了东南动荡,躲过了崇祯中后期的乱象,但是由于各种阻挠,1643年才颁行天下。在这种局势中,很难想象天文学研究还能继续。1644年清军入关,《崇祯历书》改名《西洋新法历书》颁行天下。尽管如此,该历书仍然主要以第谷体系为主,与中国传统历算关系不大\footnote{这点我未经考证。}。此后康熙帝对西洋新法十分看重\footnote{这点从他对汤若望的恩宠即可看出,且他的兴趣不止于历算。},此后虽然有中西会通之说,但其实更倾向于西体中用,以中国之学问修订第谷体系等了\footnote{此处是个人看法,未必正确。}。

钱穆先生说过:“中国史上之东西交接,至少已经三期,第一期是近西的中印接触,第二期是远西的中回接触,第三期才是更远西的中欧接触。前两期各自经历六七百年的长期间,已见中华民族对外来异文化之一般态度及其成效。现在的中欧接触,自明末以来,为期只三百年,虽则西洋以其过强之势力压迫于我,但我们诚心接纳吸收异文化之热度,仍是与前一般。若以前两期的成绩来推论,再历三百年,中华民族一定能完成吸收融和更远西的欧洲文化。但是要吸收外面的养料,却不该先破坏自己的胃口。近代的中国人,也有笑林文忠为顽固糊涂,捧耆善、伊里布等为漂亮识大体的。这无异于站在外国人的立场,代外国人说话。”究竟再历三百年能否吸收融合欧洲文化谁也说不清,但是明末清初的政治环境以及压倒性的学习西学,正如钱穆先生口中的“破坏自己的胃口”,是绝不利于传统国学的发展,也绝非一个正常文化融合过程所应有的。
\section*{结语}
以上多是个人看法,难免谬误。本文意图以小见大,从某一小角度出发具体阐述这一问题,然行文之际不觉失之空泛,所欲及之者众,所能尽之者寡。许多想法过于零散,不成体系,抑或前后矛盾。然而时间有限,不得不草草结尾。所想说明的中心思想在于,若能给中国良好的政治环境,足够的时间发展,或许中国也能发展出自己独有的科学体系。

\begin{thebibliography}{}
  \bibitem{x}
  江晓原,钮卫星. 天文西学东渐集[J],2001.
  \bibitem{l}
  李约瑟. 中国科学技术史[J],2003.
%  \bibitem{1}
%理查德. S. 韦斯特福尔. 近代科学的建构: 机械论与力学[J], 2000, 119.
%\bibitem{2}
%彼得, 迈克尔, 哈曼, 等. 19世纪物理学概念的发展[J]. 19.
%\bibitem{3}
%乔治, 巴萨拉. 技术发展简史[J]. 2000.
\bibitem{6}
费正清,杜希德. 剑桥中国史:第8卷[J]. 2006.
\bibitem{4}
钱穆. 中国历史研究法[J]. 2001.
\end{thebibliography}


\end{document}
