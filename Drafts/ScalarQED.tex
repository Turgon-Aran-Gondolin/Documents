\PassOptionsToPackage{naturalnames}{hyperref}
\RequirePackage{luatex85}
\documentclass{article}
\usepackage{geometry}
%\usepackage{fullpage}
\usepackage{parskip}
\usepackage{physics}
\usepackage{amsmath}
\usepackage{amssymb}
\usepackage{xcolor}
\usepackage[colorlinks,linkcolor=blue,citecolor=green]{hyperref}
\usepackage{array}
\usepackage{longtable}
\usepackage{multirow}
\usepackage{comment}
\usepackage{graphicx}
\usepackage{cite}
\usepackage{amsfonts}
\usepackage{bm}
\usepackage{slashed}
\usepackage{dsfont}
\usepackage{mathtools}
\usepackage[compat=1.1.0]{tikz-feynman}
\usepackage{simplewick}
%\usepackage{fourier}
%\usepackage{slashbox}
%\usepackage{intent}
\usepackage{mathrsfs}
\usepackage{xparse}
\usepackage{enumerate}
%\usepackage{axodraw4j}
\usepackage[toc,page]{appendix}
\usepackage{multicol}

%\usepackage{luatexja-fontspec}

\geometry{left=0.5cm,right=0.5cm,top=1.5cm,bottom=2cm}

\newcommand{\gm}{\gamma^{\mu}}
\newcommand{\gn}{\gamma^{\nu}}
\newcommand{\gs}{\gamma^{\sigma}}
\newcommand{\gr}{\gamma^{\rho}}
\newcommand{\gnr}{g^{\nu\rho}}
\newcommand{\gmr}{g^{\mu\rho}}
\newcommand{\gms}{g^{\mu\sigma}}
\newcommand{\gns}{g^{\nu\sigma}}
\newcommand{\vbp}{\vb{p}}
\newcommand{\vbk}{\vb{k}}
\newcommand{\g}{\gamma}
\renewcommand{\a}{\alpha}
\renewcommand{\b}{\beta}
\renewcommand{\t}{\theta}
\newcommand{\la}{\lambda}
\newcommand{\p}{\phi}
\newcommand{\vp}{\varphi}
\newcommand{\s}{\sigma}
\newcommand{\G}{\Gamma}
\newcommand{\pars}{\slashed\partial}
\newcommand{\ps}{\slashed p}
\newcommand{\ks}{\slashed k}
\newcommand{\lag}{\mathcal{L}}
\newcommand{\da}{^{\dagger}}
\newcommand{\sm}{^{\mu}}
\newcommand{\sn}{^{\nu}}
\newcommand{\smn}{^{\mu\nu}}
\newcommand{\Dm}{D^{\mu}}
\newcommand{\dm}{\partial^{\mu}}
\newcommand{\Asquare}{A^{\mu}A_{\mu}}
\newcommand{\partialsquare}[2]{\partial^{\mu}{#1}\partial_{\mu}{#2}}

%\setmainjfont[BoldFont=FandolSong-Bold]{FandolSong-Regular}
%\setsansjfont{FandolSong-Bold}
%\setlength{\parindent}{2em}
%\linespread{1.2}

\title{Scalar QED}
\author{Yingsheng Huang}
\begin{document}
\maketitle
\section{Hydrogen Wavefunction Divergence in Klein-Gordon Equation and Schr\"odinger Equation}


\section{Non-relativistic Scalar QED (NRSQED) Matching}
\subsection{Feynman Rules}
\subsubsection{Scalar QED (SQED)}
Lagrangian
\begin{align}
  \lag_{SQED}=\abs{D_{\mu}\phi}^2-m^2\abs{\phi}^2+\Phi_v^*iv\cdot D\Phi_v
  \label{SQEDLAG}
\end{align}
with
\begin{align*}
  D_{\mu}\phi=\partial_{\mu}\phi+ieA_{\mu}\phi
\end{align*}
and 
\begin{align*}
  D_{\mu}\Phi_v=\partial_{\mu}\Phi-iZeA_{\mu}\Phi_v
\end{align*}
But note that no $\vb{A}$ can appear in actual calculation because here only static scalar potential exists. 
And the Feynman rules
\begin{multicols}{2}
  \begin{align*}
	\feynmandiagram[small,horizontal=a to b]{
	a -- [momentum=$p$] b,
    };&=\frac{i}{p^2-m^2+i\epsilon}\\
	\feynmandiagram[small,horizontal=a to o,baseline=(o.base)]{
	  a -- [momentum=$p_1$] o,
	  o -- [momentum=$p_2$] b,
	  c -- [photon] o,
	};&=-ie(p_1^{\mu}+p_2^{\mu})\\
	\feynmandiagram[small,horizontal=a to b,baseline=(o.base)]{
	  a -- [momentum'=$p_1$] o -- [momentum'=$p_2$] b,
	  c -- [photon] o,
	  d -- [photon] o,
	};&=2ie^2g^{\mu\nu}
  \end{align*}
  \begin{align*}
	\feynmandiagram[small,horizontal=a to b]{
		a -- [momentum=$mv+k$,double distance=1pt] b,
	};&=\frac{i}{v\cdot k}\\
	\feynmandiagram[small,baseline=(o.base)]{
	  a -- [double distance=1pt] o -- [double distance=1pt] b,
	  c -- [photon] o,
	};&=iZev^{\mu}\\\\\\
	\feynmandiagram[small,horizontal=a to b]{
	  a [particle=$A^0$] -- [photon,momentum=$q$] b,
	};&=\frac{i}{\vb{q}^2}
  \end{align*}
\end{multicols}
\subsubsection{NRSQED}
Lagrangian
\begin{align}
  \lag_{NRSQED}=\varphi^*\pqty{iD_0+\frac{\vb{D}^2}{2m}}\varphi+\delta\lag +\Phi_v^*iv\cdot D\Phi_v
  \label{NRSQEDLAG}
\end{align}
with the same notation above. Here $\vb{D}=\nabla-ie\vb{A}$.

Since we need to match it to $\mathcal{O}(v^2)$ order
\begin{align}
  \delta\lag=(D_0\varphi)^*(D_0\varphi)=\frac{\dot{\varphi}^*\dot{\varphi}}{2m}+\frac{e^2\varphi^*\varphi A_0^2}{2m}-\frac{ie}{2m}A_0(\varphi^*\dot{\varphi}-\dot{\varphi}^*\varphi)
  \label{<+label+>}
\end{align}<++>

Feynman rules are also the same except for the scalar electron side which becomes
\begin{align*}
	\feynmandiagram[small,horizontal=a to b]{
	a -- [momentum=$p$] b,
    };=\frac{i}{E-\frac{\vb{p}^2}{2m}+i\epsilon}\;\;\;\;\;\;\;\;\;\;\;\;\;\;\;
  \feynmandiagram[small,horizontal=a to o,baseline=(o.base)]{
	  a -- [momentum=$p_1$] o,
	  o -- [momentum=$p_2$] b,
	  c [particle=$A^0$] -- [photon] o,
	};=-ie(
\end{align*}
We can ignore all interacting terms involving $\vb{A}$. 
\subsection{LO}
\subsubsection{SQED}
\begin{align*}
  i\mathcal{M}_{SQED}^{(0)}&=\feynmandiagram[horizontal=i1 to f1,layered layout,inline=($(a)!0.5!(c)$),medium]{
	i1[particle=$P_N$] -- [double distance=1pt] a -- [double distance=1pt] f1[particle=$P_N$],
	i2[particle=$p_1$] -- [] c -- [] f2[particle=$p_2$],
	{ [same layer] a -- [photon,momentum'=$q$] c},
  };=-e^2v^{0}\frac{i(p_1^0+p_2^0)}{\vb{q}^2}
\end{align*}
\subsubsection{NRSQED}
 \begin{align*}
   i\mathcal{M}_{NRSQED}^{(0)}&=\feynmandiagram[horizontal=i1 to f1,layered layout,inline=($(a)!0.5!(c)$),medium]{
	i1[particle=$P_N$] -- [double distance=1pt] a -- [double distance=1pt] f1[particle=$P_N$],
	i2[particle=$p_1$] -- [] c -- [] f2[particle=$p_2$],
	{ [same layer] a -- [photon,momentum'=$q$] c},
  };=-e^2v^{0}\frac{i}{\vb{q}^2}
\end{align*}
\subsection{NLO}
\begin{align*}
  i\mathcal{M}_{SQED}^{(1)}&=\feynmandiagram[horizontal=i1 to f1,layered layout,inline=($(a)!0.5!(c)$),medium]{
	i1[particle=$P_N$] -- [double distance=1pt] a -- [double distance=1pt,momentum=$P_N-k^0$] b -- [double distance=1pt] f1[particle=$P_N$],
	i2[particle=$p_1$] -- [] c -- [momentum=$p_1+k$] d -- [] f2[particle=$p_2$],
	{ [same layer] a -- [photon,momentum'=$k$] c},
	{ [same layer] b -- [photon,rmomentum=$k-q$] d},
  };=-e^2v^{0}\int\frac{\dd^4k}{(2\pi)^4}\frac{i}{\vb{k}^2}\frac{}{}
\end{align*}






\end{document}
