%!mode::"Tex:UTF-8"
\PassOptionsToPackage{unicode}{hyperref}
\PassOptionsToPackage{naturalnames}{hyperref}
\documentclass{article}
\usepackage{fullpage}
\usepackage{parskip}
\usepackage{physics}
\usepackage{amsmath}
\usepackage{amssymb}
\usepackage{xcolor}
\usepackage[colorlinks,linkcolor=blue,citecolor=green]{hyperref}
\usepackage{array}
\usepackage{longtable}
\usepackage{multirow}
\usepackage{comment}
\usepackage{graphicx}
\usepackage{cite}
\usepackage{amsfonts}
\usepackage{bm}
\usepackage{extarrows}
%\usepackage{xeCJK}
\usepackage{luatexja-fontspec}

%\setmainfont{TeXGyreTermes}
%\setsansfont{TeXGyreHeros}

%\newcommand{\b}{\beta}
\newcommand{\s}{\sigma}

\setmainjfont[BoldFont=FandolSong-Bold]{FandolSong-Regular}
\setsansjfont{FandolSong-Bold}

\hypersetup{unicode=true}


\title{Ising Model \& Monte Carlo method}
\author{Yingsheng Huang}
\begin{document}
\maketitle
Ising Model:
\begin{itemize}
  \item Hamiltonian of the system:
  \begin{align*}
    H(\sigma)&\xlongequal{\phantom{h=0}}-J\sum_{\expval{ij}}\sigma_i\sigma_j-h\sum_{j}\sigma_j\\
    &\xlongequal{h=0}-J\sum_{\expval{ij}}\sigma_i\sigma_j
  \end{align*}
  $h$ is the external magnetic field (for simplicity we now consider $h=0$), and $J>0$ which means it's ferromagnetic. (And it's reasonable to consider the lowest energy state is when the spins are all +1.)
  \item Total Energy at configuration $\{\sigma_i \}$:
  $$E_{\{\sigma_i\}}=-J\sum_{\expval{ij}}\sigma_i\sigma_j-h\sum_{j}\sigma_j$$
  \item Spin state $\sigma_i$ is differed by
  $$\sigma_i=\begin{cases}
  +1\\-1
  \end{cases}$$
  \item Configuration probability:
  $$P_{\beta}(\s)=\frac{e^{-\beta H(\s)}}{Z_{\beta}}$$
  where $\beta=(k_BT)^{-1}$ and $Z_{\beta}$ is the partition function.
  \item Partition function:
  $$Z_{\beta}=\sum_{\sigma}e^{-\beta H(\sigma)}$$


\end{itemize}
Mento Carlo Method (Metropolis Method):
\begin{itemize}
  \item Given $L^d$ lattice point. (For instance $d=1$, which means 1-d Ising model.)

  \begin{itemize}{(1)}
    \item Give a initial state, and calculate its energy.
    \item
    \item Calculate the magnetic dipole and get the phase transition point.
  \end{itemize}
\end{itemize}


\begin{itemize}[1.]
  \item 随机生成态s1
  \item 随机生成态s2
  \item 计算s1与s2的权重p1,p2($p=exp[-H(s)$])
  \item 如果$p2>p1$,s2为第二取样点
  \item 如果p2<=p1,生成随机数r,如果$r<=\frac{p2}{p1}$,s2为第二取样点,否则s1为第二个取样点。
\end{itemize}



\end{document}
