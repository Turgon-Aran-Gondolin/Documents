%!mode::"Tex:UTF-8"
\PassOptionsToPackage{unicode}{hyperref}
\PassOptionsToPackage{naturalnames}{hyperref}
\documentclass{article}
\usepackage{fullpage}
\usepackage{parskip}
\usepackage{physics}
\usepackage{amsmath}
\usepackage{amssymb}
\usepackage{xcolor}
\usepackage[colorlinks,linkcolor=blue,citecolor=green]{hyperref}
\usepackage{array}
\usepackage{longtable}
\usepackage{multirow}
\usepackage{comment}
\usepackage{graphicx}
\usepackage{cite}
\usepackage{amsfonts}
\usepackage{bm}
\usepackage{extarrows}
%\usepackage{xeCJK}
\usepackage{luatexja-fontspec}

%\setmainfont{TeXGyreTermes}
%\setsansfont{TeXGyreHeros}

\setmainjfont[BoldFont=SimHei]{SimSun}
\setsansjfont{SimHei}

\hypersetup{unicode=true}


%\usepackage{CJK}
%\usepackage{fourier}
%\usepackage{slashbox}
%\usepackage{intent}

\newcommand{\bP}{\vb{P}}
\newcommand{\bA}{\vb{A}}
\newcommand{\ba}{\boldsymbol{\alpha}}
\newcommand{\po}{\vb{p}_1}
\newcommand{\ps}{\vb{p}_2}
\newcommand{\apo}{\abs{\vb{p}_1}}
\newcommand{\aps}{\abs{\vb{p}_2}}

\title{Homework: Particle Physics \#2}
\author{Yingsheng Huang}
\begin{document}
\maketitle
{\bf1.}\quad

Isospin transformation: transform one particle into its conjugated charge state composed of different numbers of u and d quarks. It's built in an abstract space, the group space of SU(2) group. The eigenstates of isospin transformation are u and d and commonly written as $\ket{II_3}$. (In early understanding, the three charge states of $\pi$ meson are also considered the eigenstates of isospin transformation.) Isospin is conserved in strong interaction.

C transformation (charge conjugation transformation): transform a particle into its antiparticle. The eigenstates of it are neutral particles. C parity is conserved in strong and electromagnetic interactions.

G transformation: change the signs of all internal additive quantum numbers but $I_3$. The eigenstates are all common mesons. G parity is only conserved in strong interaction.

P transformation: transform the spatial coordinate of particles, or, the parity of particles. The eigenstates are the eigenstates of orbital angular momentum. P parity is conserved in strong and electromagnetic interactions.

CP transformation: transform both the spatial coordinate and the charge state of a particle. The eigenstates are neutral particles (with the states of angular momentum). CP is conserved in strong and electromagnetic interactions.

{\bf2.}\quad
In LAB frame, the reaction is $pp\rightarrow\bar pX$ where $X$ is of charge +3 and mass $m_X$. Accually $X=ppp$ so $pp\rightarrow\bar pppp$. And for the lowest energy we can assume the products are at rest.

The initial state
$$E^2=E_{\vb{p}}+m_p,\vb{p}=\vb{p}_p+0=\vb{p}_p,m=2m_p,m_p=1GeV$$
The final state
$$k=k_{\bar p}+3k_{p},$$
So
$$p^2=(E_{\vb{p}}+m_p)^2-\vb{p}_p^2=(p_1+p_2)^2=2m_p^2+2E_{\vb{p}}m_p$$
$$k^2=16m_p^2$$
$$2m_p^2+2E_{\vb{p}}m_p=16m_p^2$$
$$E_{\vb{p}}=7m_p$$
The minimum momentum required is
$$\vb{p}=\sqrt{E_{\vb{p}}^2-m_p^2}=\sqrt{48}m_p=4\sqrt{3}m_p\approx6.928GeV$$

{\bf3.}\quad
For $\phi(1020)$ meson: $I^G(J^{PC})=0^-(1^{--})$, $\Gamma=(4.458)MeV$.

1) $K_L{ }^0K_S{ }^0$\quad$33.8\%$

The partial width of it is $\Gamma=4.458\times33.8\%\approx1.5MeV$ so it's strong interaction.

2) $K^+K^-$\quad$49.2\%$

Same as the first.

3) $\pi^+\pi^-\pi^0$\quad$15.5\%$

The isospin of the final state:
\begin{array}{cccc}
  \pi^+&\pi^-&\pi^0\\
  \ket{11}&\ket{1-1}&\ket{10}
\end{array}

and all possible $I$ of the final state are: $0,1$, $I_3=0$. From the branching ratio, we know that $\Gamma=4.458\times15.5\%\approx0.691MeV$ and it's about the order of strong interaction.

4) $\eta\gamma$\quad$1.3\%$

$\gamma$ is involved so it's EM interaction.

5) $\pi^0\gamma$\quad$1.26\times10^{-3}$

$\gamma$ is involved so it's EM interaction.

6) $\mu^+\mu^-$\quad$2.9\times10^{-4}$

$\Gamma=4.458\times2.9\times10^{-4}\approx10^3eV$, so it's EM interaction.

7) $\omega\pi^0$\quad$5.2\times10^{-5}$

The isospin of the final state:
\begin{array}{cccc}
  \omega&\pi^0\\
  \ket{00}&\ket{10}
\end{array}.
But $\Gamma=4.458\times5.2\times10^{-5}\approx232eV$ so it's EM interaction. G parity is not conserved as well.


8) $\pi^+\pi^-$\quad$7.3\times10^{-5}$

The isospin of the final state:
\begin{array}{cccc}
  \pi^+&\pi^-\\
  \ket{11}&\ket{1-1}
\end{array}.
But $\Gamma=4.458\times5.2\times10^{-5}\approx10^{3}eV$ so it's EM interaction. G parity is not conserved as well.


{\bf4.}\quad
Neutral system composed of two $\pi$.

$\pi^+\pi^-$: C number is $(-)^{L+S}$, G number is $C(-)^{I}$ and P number is $(-)^L$. For general identical particle, $L+S+I-2i=even$, and here $I=0,1,2$, $i=1$, $S=0$.

We have
\begin{cases}
  I=0,2, L+S=even,C=+,P=+, G=+\\
  I=1, L+S=odd,C=-,P=-, G=+
\end{cases}.

$\pi^0\pi^0$: Similarly, for $\pi^0$ we have $C=+$, $P=-$, $G=-$. And for this system we have $L+S=even$, $S=0$, so $C=+$, $P=--(-)^L=+$, $G=+$.

{\bf5.}\quad
Neutral system composed of $K^+K^-$ ($S=0$).

%First $L+S+I-2i=even$, and the isospin states are $\ket{\frac{1}{2}\frac{1}{2}}$ and $\ket{\frac{1}{2}-\frac{1}{2}}$ so that the total isospin could be 0 or 1 and $i=\frac{1}{2}$.

We have \begin{cases}
I=1,C=(-)^L,P=(-)^L,G=(-)^{L+1}\\
I=0,C=(-)^L,P=(-)^L,G=(-)^L
\end{cases}

{\bf6.}\quad
It's hard to identify $K^0$ and $\bar K^0$ meson via decay type. It's because their differences are only in strangeness and $I_3$. And in weak interaction these are not conserved. In the process decaying to $\pi$ mesons, if we view it in the eigenstates of CP transformation, we'll find one eigenstate have much longer lifetime than the other.

PS: \begin{cases}
K^0:\;S=1,I_3=\frac{1}{2}\\
\bar K^0:\; S=-1,I_3=-\frac{1}{2}
\end{cases}, and the eigenstates of CP are \begin{cases}
K_L=\frac{1}{\sqrt{2}}(\ket{K^0}+\ket{\bar K^0})\\
K_S=\frac{1}{\sqrt{2}}(\ket{K^0}-\ket{\bar K^0})
\end{cases}

{\bf7.}

1) $\pi^-p\rightarrow K^+\Sigma^-$
Strangeness is conserved. So it's strong interaction.

2) $pp\rightarrow n\pi^+\Sigma^-$
Charge not conserved. It can't happen.

3) $\pi^0\rightarrow e^+e^-e^+e^-$
The final states involves leptons, so it's EM interaction. G is not conserved.% Weak interaction is possible as well but with much smaller branching ratio.

4) $\rho^0\rightarrow\eta\pi^0$
If it can happen, CP must be conserved. So $(+)=(-)(-)(-)^L\Longrightarrow L=even$. And the angular momentum must be conserved, so $1=0+0+L\Longrightarrow L=odd$. So it can't happen.

5) $J/\Psi\rightarrow\pi^+\pi^-$
Charm's conserved. CP is conserved. If C is conserved, $L+S=odd$. If P is conserved, $L=odd$. And we know $L+S+I-2i=even$. If G is conserved, $L+S+I=odd$, so G isn't conserved, it can't be strong interaction. It's EM or weak interaction.

6) $J/\Psi\rightarrow\pi\rho$
Charm's conserved. C is conserved. If P is conserved, $L=odd$. The angular momentum $1\leq0+1+L$. G is conserved. It's strong interaction.

7) $p\bar n\rightarrow K^+K^0\pi^0$
Strangeness's not conserved. Baryon number is conserved. It can happen via weak interaction.

8) $K^+\rightarrow\pi^+\pi^0$
Isospin and strangeness are not conserved. It's weak interaction.

9) $\rho\rightarrow\pi^0\pi^0$
C parity not conserved. CP is conserved. We know that $L+S=even$, but for conserved angular momentum $L=1$. so it can't happen.

10) $K^0\rightarrow\pi^+\pi^-\pi^0$
Strangeness is not conserved. $K^0$ is the linear combination of $K^L$ and $K_S$, so its decay products should contain both two pion system and three pion system. It can happen via weak decay in the form of $K_L$.

11) $\eta\rightarrow\pi\pi$
CP is not conserved. It can't happen.

12) $\eta\rightarrow\pi^+\pi^-\pi^0$
CP is conserved. G is not conserved (for n pion system $G=(-1)^n$ but $G(\eta)=-$). It's EM or weak interaction.

13) $\phi\rightarrow K^SK^S$
$C(K_S)=-,C(K_L)=+$, $L+S=even,S=0$, $P(K_SK_S)=(-)^L=+$, $CP(\phi)=+$, $CP(K_SK_S)=-$, so it can't happen.

14) $\eta\rightarrow\pi^+\pi^-\gamma$
Photon is involved, C and P are conserved, it's EM interaction.

15) $\omega\rightarrow\pi^+\pi^-$
G is not conserved. C and P are conserved. It's EM or weak interaction.

%Set
%$$M_{1}=\mel{I=1}{H}{I=1},M_{0}=\mel{I=0}{H}{I=0}$$
%We know that the final state is the linear combination of $\ket{\frac{1}{2}-\frac{1}{2}}$ and %$\ket{\frac{1}{2}\frac{1}{2}}$, so
%$$\mel{f}{H}{i}=\frac{1}{\sqrt{2}}\mel{00}{H}{00}+\frac{1}{\sqrt{2}}\mel{10}{H}{00}$$

\end{document}
