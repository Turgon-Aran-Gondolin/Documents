%!mode::"Tex:UTF-8"
\documentclass{article}
\usepackage{fullpage}
\usepackage{parskip}
\usepackage{physics}
\usepackage{amsmath}
\usepackage{amssymb}
\usepackage{xcolor}
\usepackage[colorlinks,urlcolor=blue,citecolor=green,anchorcolor=blue]{hyperref}
\usepackage{array}
\usepackage{longtable}
\usepackage{multirow}
\usepackage{comment}
\usepackage{graphicx}
\usepackage{cite}
\usepackage{float}
%\usepackage{intent}
%\hypersetup{colorlinks,linkcolor=blue,citecolor=green}

\title{Linear potential $\abs{x}$}
\author{Yingsheng Huang}
\begin{document}
\maketitle
Given a linear potential $\lambda\abs{x}$, we can write down the Schr\"odinger equation as follows:
\begin{equation}\label{sch}
  (-\frac{1}{2\mu}\dv[2]{x}+\lambda\abs{x})\psi(x)=E\psi(x)
\end{equation}
where the reduced mass $\mu=\frac{5}{2}$ and $\lambda=\frac{\pi}{2}$. In bound state, the boundary condition can be given by
\begin{equation}
\begin{cases}
\psi(-\infty)=0\\
\psi(\infty)=0
\end{cases}
\end{equation}
For different parity, the boundary condition at $x=0$ is different.
\begin{equation}
  \begin{cases}
\psi(0)=0\;\;\;\text{Odd parity}\\
\psi'(0)=0\;\;\;\text{Even parity}
  \end{cases}
\end{equation}
So far, we have all the information needed to solve the eigen energy, which is listed in the following table. The eigen vectors are a set of Airy functions\footnote{If $x>0$, it's pure Airy function.But since our problem involves scenario when $x<0$, small changes must be made. When $x<0$, it's not Airy function, instead, it becomes a mirror of Airy function with positive x. The sign of it changes with different parity.}:
\begin{equation}
\psi(x)=
  \begin{cases}
  \text{Ai}\left(\frac{x}{\sqrt[3]{\frac{1}{2 \lambda  m}}}-\frac{e}{\sqrt[3]{\frac{\lambda ^2}{2 m}}}\right), x>0\\
  \pm \text{Ai}\left(-\frac{e}{\sqrt[3]{\frac{\lambda ^2}{2 m}}}-\frac{x}{\sqrt[3]{\frac{1}{2 \lambda  m}}}\right), x<0
  \end{cases}
\end{equation}
where the sign of Airy function when $x<0$ consist with the sign of parity.

Adding relativistic correction $\frac{p^4}{32\mu^3}$, the differential equation becomes a fourth order one:
\begin{equation}\label{sch}
  (-\frac{1}{32\mu^3}\dv[4]{x}-\frac{1}{2\mu}\dv[2]{x}+\lambda\abs{x})\psi(x)=E\psi(x)
\end{equation}
Of course we can solve it numerically, but due to the difficulties in solving high order differential equation, perturbation method is introduced as another approach.

First, we solve it directly with boundary condition\footnote{$x\rightarrow\infty$ condition is impossible numerically, so we can substitude $\infty$ with a large number, here we choose 100, which is large enough. And we use a small number instead of zero to compensate it.}
\begin{eqnarray}
    \begin{cases}
% \nonumber % Remove numbering (before each equation)
    \psi(\infty)=0\\
    \psi'(\infty)=0^-\\
    \psi''(\infty)=0\\
    \psi'''(\infty)=0^-
    \end{cases}
\end{eqnarray}
But the attemption of solving even parity situation was failed (something about numerically ill-conditioned boundary condition), so I changed the $p^4$ term using equation of motion the same way Lepage did:
\begin{equation}
  (-\frac{(E-\lambda\abs{x})^2}{8\mu}-\frac{1}{2\mu}\dv[2]{x}+\lambda\abs{x}-\mathcal{O}(\frac{p^6}{\mu^5}))\phi(x)=E\phi(x)
\end{equation}
Using parity condition, the eigen states can be easily solved.

With perturbation theory, we can solve this analytically. I only involved 1st-order perturbation. Considering $\frac{p^4}{32\mu^3}$ as a perturbation, the energy correction is
\begin{equation}
  E^{(1)}=-\frac{1}{32\mu^3}\expval{\dv[4]{x}}=-\frac{1}{32\mu^3}\braket{\dv[2]{\psi^*}{x}}{\dv[2]{\psi}{x}}
\end{equation}
Then we have our corrected energy.

Another approach is to solve the Schr\"odinger equation in momentum space, which is a lot easily to solve in the aspect of differential equation. But the hard part is the fourier transformation from momentum space to coordinate space. First, the Schr\"odinger equation in momentum space without relativistic correction is:
\begin{equation}
    \frac{p^2 \phi (p)}{2 \mu }+i \lambda  \hbar  \frac{\partial \phi (p)}{\partial p}=E \phi (p)
\end{equation}
and we have a solution
\begin{equation}
  \phi(p)=C_1e^{\frac{i \left(\frac{p^3}{3}-2 E \mu  p\right)}{2 \lambda  \mu  \hbar }}
\end{equation}
Then we fourier transform $\phi(p)$ into coordinate representation and apply the boundary condition demonstrated above to obtain the eigen energy, which gives identical results as in coordinate space. Adding a relativistic correction $\frac{p^4}{32\mu^3}$ makes the equation become
\begin{equation}
  -\frac{p^4 \phi (p)}{32 \mu^3 }+\frac{p^2 \phi (p)}{2 \mu }+i \lambda  \hbar  \frac{\partial \phi (p)}{\partial p}=E \phi (p)
\end{equation}
The solution of this differential equation is
\begin{equation}
  \phi(p)=C_2e^{\frac{i \left(-32 E \mu  p-\frac{p^5}{5}+\frac{16 p^3}{3}\right)}{32 \lambda  \mu^3  \hbar }}
\end{equation}
The fourier transformation becomes very difficult to solve analytically, so I try to solve it numerically, which leads to some strange results.

If we calculate the 1st-order perturbative energy in momentum representation, we'll find it divergent. The 1st-order perturbative energy would consist of something like $\int_{-\infty}^{\infty}p^4\dd p$. It should be the same as in coordinate representation. I still have no way around it.

The final results are listed in the following table (comparing with Shuangran's result):
\begin{table}[H]
  \centering
  \begin{tabular}{|c|c|c|c|c|c|}
    \hline
    % after \\: \hline or \cline{col1-col2} \cline{col3-col4} ...
    n & 0 & 1 & 2 & 3 \\\hline
    Schr\"odinger equation&0.805086&1.84766&2.56684&3.23044\\\hline
    't Hooft equation& 0.62674 & 1.62773 & 2.32744 & 2.95498 \\\hline
    1st-order perturbation&0.792475&1.81352&2.49903&3.12609\\\hline
    {\color{red}{error}} comparing with 't Hooft eqn&26.44\%&11.41\%&7.37\%&5.79\%\\\hline
    numerical solution ($-\frac{(E-\lambda\abs{x})^2}{8\mu}$)&0.708083 &1.81508&2.56503 &3.19580\\\hline
    {\color{red}{error}} comparing with 't Hooft eqn&12.98\%&11.50\%&10.21\%&8.15\%\\\hline
    momentum representation &0.591132&1.83398&2.68831&3.0525\\\hline
    {\color{red}{error}} comparing with 't Hooft eqn&-5.68\%&12.67\%&15.51\%&3.3\%\\\hline
  \end{tabular}
  \caption{$\mu=2.5$}
\end{table}

The last two columns are not so reliable (numerical solution in coordinate space and solution in momentum space), the error during the process of solving fourth-order differential equation and the tricky integration could be too large. In momentum space, the condition of even parity produces extra zero points (which indicate eigen energy) if the integration limits of fourier transformation are chosen too large. It's another reason I'm not convinced by these results.

Changing the reduced mass into 25, the energy will become
\begin{table}[H]
  \centering
  \begin{tabular}{|c|c|c|c|c|c|}
   \hline
   n & 0 & 1 & 2 & 3 \\\hline
   Schr\"odinger equation&0.373688& 0.857606& 1.19142& 1.49944\\\hline
   't Hooft equation& 0.3541& 0.836931& 1.17025&1.4773 \\\hline
   1st-order perturbation&0.373416&0.85687& 1.18996& 1.49719\\\hline
   {\color{red}{error}} comparing with 't Hooft eqn&5.45\%&2.38\%&1.68\%&1.34\%\\\hline
   momentum representation &0.352774&0.834191&1.1671&1.47727\\\hline
   {\color{red}{error}} comparing with 't Hooft eqn&-0.37\%&-0.33\%&-0.27\%&-0.002\%\\\hline
  \end{tabular}
  \caption{$\mu=25$}
\end{table}

Changing the reduced mass into 50, the error of 1st-order perturbative energy (as the error listed in the table above) will decrease accordingly (to about $3.4\%$ on ground state).
\end{document} 