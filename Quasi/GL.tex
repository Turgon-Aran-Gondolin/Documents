\RequirePackage{luatex85}
\PassOptionsToPackage{unicode}{hyperref}
\PassOptionsToPackage{naturalnames}{hyperref}
\documentclass{article}
\usepackage{geometry}
\usepackage[cm]{fullpage}
\usepackage{parskip}
\usepackage{physics}
\usepackage{amsmath}
\usepackage{amssymb}
\usepackage[dvipsnames]{xcolor}
\usepackage[colorlinks,linkcolor=blue,citecolor=green]{hyperref}
\usepackage{array}
\usepackage{longtable}
\usepackage{multirow}
\usepackage{comment}
\usepackage{graphicx}
\usepackage{cite}
\usepackage{amsfonts}
\usepackage{bm}
\usepackage{slashed}
\usepackage{dsfont}
\usepackage{mathtools}
\usepackage[compat=1.1.0]{tikz-feynman}
\usepackage{pgfplots}
\pgfplotsset{compat=newest}
\usepackage{simpler-wick} 
\usepackage{mathrsfs}
\usepackage{xparse}
\usepackage{enumerate}
\usepackage{extarrows}
\usepackage[caption=false]{subfig}
% ==============================================================================
% Endnotes
% ==============================================================================
\usepackage{enotez}
% HOWTO: 
% \endnote{}
% Print the notes by adding the following to the end of the document: 
% \printendnotes

% ==============================================================================
% Minted
% ==============================================================================
\usepackage{minted}
\usemintedstyle{colorful}

% ==============================================================================
% Changes: comments and highlights
% ==============================================================================
\let\comment\undefined
\usepackage[highlightmarkup=uwave]{changes}

\allowdisplaybreaks

% ==============================================================================
% mathtools
% ==============================================================================
\newcommand\MTkillspecial[1]{% helper macro
	\bgroup
	\catcode`\&=9
	\let\\\relax%
	\scantokens{#1}%
	\egroup
	}
\DeclarePairedDelimiter\BraceM\{\}
\reDeclarePairedDelimiterInnerWrapper\BraceM{star}{
	\mathopen{#1\vphantom{\MTkillspecial{#2}}\kern-\nulldelimiterspace\right.}
	#2
	\mathclose{\left.\kern-\nulldelimiterspace\vphantom{\MTkillspecial{#2}}#3}
	}
\DeclarePairedDelimiter\bracketM{[}{]}
\reDeclarePairedDelimiterInnerWrapper\bracketM{star}{
	\mathopen{#1\vphantom{\MTkillspecial{#2}}\kern-\nulldelimiterspace\right.}
	#2
	\mathclose{\left.\kern-\nulldelimiterspace\vphantom{\MTkillspecial{#2}}#3}
	}
\let\Bqty\relax
\let\bqty\relax
\newcommand{\Bqty}[1]{\BraceM*{#1}}
\newcommand{\bqty}[1]{\bracketM*{#1}}
\DeclarePairedDelimiter\ceil{\lceil}{\rceil}
\DeclarePairedDelimiter\floor{\lfloor}{\rfloor}

% ==============================================================================
% ifthen for User-definition
% ==============================================================================
\usepackage{ifthen}
% HOWTO:
% \ifthenelse{<test>}{<code for true>}{<code for false>}

% ==============================================================================
% User Definition
% ==============================================================================
\newcommand{\red}[1]{{\color{red}#1}}
\newcommand{\mm}[1]{\frac{\dd^4#1}{(2\pi)^4}}
\newcommand{\mme}[1]{\frac{\dd^3\vb{#1}}{(2\pi)^3}}
\newcommand{\mmd}[2][d]{\ifthenelse{\equal{#1}{1}}{\frac{\dd {#2}}{2\pi}}{\frac{\dd^{#1}{#2}}{(2\pi)^{#1}}}}

\newcommand{\glprog}[2]{\int\frac{\dd #1}{2\pi}\frac{ie^{-i#1 #2}}{#1+i\epsilon}}

\makeatletter
\newcommand{\pushright}[1]{\ifmeasuring@#1\else\omit\hfill$\displaystyle#1$\fi\ignorespaces}
\newcommand{\pushleft}[1]{\ifmeasuring@#1\else\omit$\displaystyle#1$\hfill\fi\ignorespaces}
\makeatother

\NewDocumentCommand\NL{s}{%
  \IfBooleanTF#1%
    {\notag\\\times}% If a star is seen
    {\notag\\}%     If no star is seen
}



% ==============================================================================
% Tikz-Feynman Externalization
% ==============================================================================
\usepackage{shellesc}
\usetikzlibrary{external}
% \usepgfplotslibrary{external}
\tikzexternalize[shell escape=-enable-write18,prefix=./GL/,system call={lualatex \tikzexternalcheckshellescape -halt-on-error -interaction=batchmode -jobname "\image" "\texsource"},up to date check=diff]

\tikzfeynmanset{
	Eikonal/.style={
		/tikz/draw=none,
		/tikz/decoration={name=none},
		/tikz/postaction={
			/tikz/draw,
			/tikz/double distance=2pt,
			% /tikzfeynman/with arrow=0.5,
		},
	},
	half right/.append style={
		/tikz/looseness=2
	},
	half left/.append style={
		/tikz/looseness=2
	}
}

% ==============================================================================
% Tikz-Feynman Auxiliary
% ==============================================================================
\def\FDWidth{3cm}
\def\FDHeight{3cm}
\def\FDWidthS{2cm}
\def\FDHeightS{2cm}


\newcommand{\WN}[1]{\textcolor{RawSienna}{#1}}
\newcommand{\CJ}[1]{\textcolor{RoyalBlue}{#1}}


\title{One Loop Gauge Link Self Energy}
\author{Yingsheng Huang}
\begin{document}
\maketitle
\section{One Loop}
\renewcommand*\thesubfigure{\roman{subfigure}}
\begin{figure}[!htpb]
	\centering\null\hfil
	\subfloat[g]{
		\centering
		\begin{tikzpicture}[baseline=($(a)!0.5!(exa)$.base)]
			\begin{feynman}
				\node[dot] (a);
				\node[right=\FDWidth of a,dot] (b);
				\vertex at ($(a)!0.5!(b)$) (o);
				\vertex at ($(a)!0.2!(b)$) (o1);
				\vertex at ($(a)!0.8!(b)$) (o2);
				\vertex[below=\FDHeight of a] (exa);
				\vertex[below=\FDHeight of b] (exb);
				\diagram*{
				(a) --[Eikonal] (o);
				(exa) --[fermion] (a);
				(b) --[fermion] (exb);
				(o1) --[gluon, half right] (o);
				};
			\end{feynman}
		\end{tikzpicture}
		\label{2-o}
	}\hfil
	\subfloat[h]{
		\centering
		\begin{tikzpicture}[baseline=($(a)!0.5!(exa)$.base)]
			\begin{feynman}
				\node[dot] (a);
				\node[right=\FDWidth of a,dot] (b);
				\vertex at ($(a)!0.5!(b)$) (o);
				\vertex at ($(a)!0.2!(b)$) (o1);
				\vertex at ($(a)!0.8!(b)$) (o2);
				\vertex[below=\FDHeight of a] (exa);
				\vertex[below=\FDHeight of b] (exb);
				\diagram*{
				(b) --[Eikonal] (o);
				(exa) --[fermion] (a);
				(b) --[fermion] (exb);
				(o) --[gluon, half right] (o2);
				};
			\end{feynman}
		\end{tikzpicture}
		\label{2o-}
	}\hfil\null
	\caption{Diagrams of quasi PDF in Feynman gauge. }
\end{figure}

The definition of the gauge link self energy diagram (diagram g) is
\begin{align}
	\frac{1}{2} \int \frac{\dd z}{2 \pi} e^{i x P^{z} z}\langle P, S|\bar{\psi}(z) \gamma^z \frac{\mathcal{P} \left[-i g_s \int_{0}^{\infty} \dd z^{\prime} A^{a,z}\left(z^{\prime}\right) \mathrm{t}^{a}\right]\left[-i g_s \int_{0}^{\infty} \dd z^{\prime\prime} A^{a,z}\left(z^{\prime\prime}\right) \mathrm{t}^{a}\right]}{2}  \psi(0)| P, S\rangle
\end{align}
Applying Feynman rule straightaway gives (the overall $1/2$ factor has been counted in)
\begin{align}
	\Gamma_g(l)=\begin{tikzpicture}[baseline=($(a)!0.5!(exa)$.base)]
		\begin{feynman}
			\node[dot] (a);
			\node[right=\FDWidthS of a,dot] (b);
			\vertex at ($(a)!0.6!(b)$) (o);
			\vertex at ($(a)!0.3!(b)$) (o1);
			\vertex at ($(a)!0.8!(b)$) (o2);
			\vertex[below=\FDHeightS of a] (exa);
			\vertex[below=\FDHeightS of b] (exb);
			\diagram*{
			(a) --[Eikonal,momentum=\(0\)] (o1) --[Eikonal,rmomentum=\(l\)] (o);
			(exa) --[fermion,momentum=\(P\)] (a);
			(b) --[fermion,momentum=\(P\)] (exb);
			(o1) --[gluon, half right,momentum'=\(l\)] (o);
			};
		\end{feynman}
	\end{tikzpicture}=-g_s^2C_F\delta(1-x)\int\mm{l}\frac{i}{l^2+i\epsilon}
	\frac{i}{i\epsilon}\frac{i}{-l^z+i\epsilon}
	\label{FR}
\end{align}

\subsection{Direct Contraction}
\subsubsection{Left}
\begin{align*}
	  & \frac{1}{2!}\mathcal{P} \left[ \int_{0}^{\infty} \dd z^{\prime} A^{a,z}\left(z^{\prime}\right) \int_{0}^{\infty} \dd z^{\prime\prime} A^{a,z}\left(z^{\prime\prime}\right) \right]                                           \\
	= & \int_{0}^{\infty} \dd z^{\prime} A^{a,z}\left(z^{\prime}\right) \int_{0}^{\infty} \dd z^{\prime\prime} A^{a,z}\left(z^{\prime\prime}\right)\theta(z'-z'')                                                                    \\
	= & \int \dd z^{\prime} A^{a,z}\left(z^{\prime}\right) \int \dd z^{\prime\prime} A^{a,z}\left(z^{\prime\prime}\right)\theta(z'-z'')\theta(z'')                                                                    \\
	= & \wick{\int \dd z^{\prime} \c2A^{a,z}\left(z^{\prime}\right)\int \dd z^{\prime\prime} \c2 A^{a,z}\left(z^{\prime\prime}\right) }
	\theta(z'-z'')\theta(z'')  \\
	= & \int \dd z^{\prime} \int \dd z^{\prime\prime} \int\mm{l}\frac{i}{l^2+i\epsilon}e^{-il\cdot(z''-z')}
	\theta(z'-z'')\theta(z'')  
\end{align*}
\subsubsection{Right}
\begin{align*}
	& \frac{1}{2!}\mathcal{P} \left[ \int_{0}^{\infty} \dd z^{\prime\prime} A^{a,z}\left(z^{\prime\prime}+z\right) \int_{0}^{\infty} \dd z^{\prime} A^{a,z}\left(z^{\prime}+z\right) \right]                                           \\
	= & \int_{0}^{\infty} \dd z^{\prime\prime} A^{a,z}\left(z^{\prime\prime}+z\right) \int_{0}^{\infty} \dd z^{\prime} A^{a,z}\left(z^{\prime}+z\right)\theta(z''-z')                                                                    \\
	= & \int \dd z^{\prime\prime} A^{a,z}\left(z^{\prime\prime}+z\right) \int \dd z^{\prime} A^{a,z}\left(z^{\prime}+z\right)\theta(z''-z')\theta(z')                                                                    \\
	= & \wick{\int \dd z^{\prime\prime} \c2A^{a,z}\left(z^{\prime\prime}+z\right)\int \dd z^{\prime} \c2 A^{a,z}\left(z^{\prime}+z\right) }
	\theta(z''-z')\theta(z')  \\
	= & \int \dd z^{\prime} \int \dd z^{\prime\prime} \int\mm{l}\frac{i}{l^2+i\epsilon}e^{-il\cdot(z''-z')}
	\theta(z''-z')\theta(z')  \\
\end{align*}
\subsubsection{Summing together}
\begin{align*}
	\int \dd z^{\prime} \int \dd z^{\prime\prime} \int\mm{l}\frac{i}{l^2+i\epsilon}e^{-il\cdot(z''-z')}
	\bqty{\theta(z'-z'')\theta(z'')+\theta(z''-z')\theta(z') }
\end{align*}
\subsection{Adding two path order together}
\begin{align*}
	  & \mathcal{P} \left[ \int_{0}^{\infty} \dd z^{\prime} A^{a,z}\left(z^{\prime}\right) \int_{0}^{\infty} \dd z^{\prime\prime} A^{a,z}\left(z^{\prime\prime}\right) \right]                                                       \\
	= & \int_{0}^{\infty} \dd z^{\prime} A^{a,z}\left(z^{\prime}\right) \int_{0}^{\infty} \dd z^{\prime\prime} A^{a,z}\left(z^{\prime\prime}\right) \bqty{\theta(z'-z'')+\theta(z''-z')}                                             \\
	= & \int_{0}^{\infty} \dd z^{\prime} A^{a,z}\left(z^{\prime}\right) \int_{0}^{\infty} \dd z^{\prime\prime} A^{a,z}\left(z^{\prime\prime}\right)                                                                                  \\
	= & \int \dd z^{\prime} A^{a,z}\left(z^{\prime}\right)\int \dd z^{\prime\prime} A^{a,z}\left(z^{\prime\prime}\right) \int\frac{\dd w}{2\pi}\frac{ie^{-iwz' }}{w+i\epsilon}\int\frac{\dd h}{2\pi}\frac{ie^{-ihz'' }}{h+i\epsilon} \\
	= & \wick{\int \dd z^{\prime} \c2A^{a,z}\left(z^{\prime}\right)\int \dd z^{\prime\prime} \c2 A^{a,z}\left(z^{\prime\prime}\right) }
	\int\frac{\dd w}{2\pi}\frac{ie^{-iwz' }}{w+i\epsilon}\int\frac{\dd h}{2\pi}\frac{ie^{-ihz'' }}{h+i\epsilon}\\
	= & \int \dd z^{\prime} \int \dd z^{\prime\prime} \int\mm{l}\frac{i}{l^2+i\epsilon}e^{-il\cdot(z''-z')}
	\int\frac{\dd w}{2\pi}\frac{ie^{-iwz' }}{w+i\epsilon}\int\frac{\dd h}{2\pi}\frac{ie^{-ihz'' }}{h+i\epsilon}\\
	= & \int\mm{l}\frac{i}{l^2+i\epsilon}
	\int \dd z^{\prime} \int \dd z^{\prime\prime}\int\frac{\dd w}{2\pi}\frac{i}{w+i\epsilon}\int\frac{\dd h}{2\pi}\frac{i}{h+i\epsilon}e^{-i(w-l)\cdot z'}e^{-i(h+l)\cdot z''} \\
	= & \int\mm{l}\frac{i}{l^2+i\epsilon}
	\frac{i}{l^z+i\epsilon}\frac{i}{-l^z+i\epsilon}
\end{align*}
The amplitude is
\begin{align}
	\begin{tikzpicture}[baseline=($(a)!0.5!(exa)$.base)]
		\begin{feynman}
			\node[dot] (a);
			\node[right=\FDWidthS of a,dot] (b);
			\vertex at ($(a)!0.6!(b)$) (o);
			\vertex at ($(a)!0.3!(b)$) (o1);
			\vertex at ($(a)!0.8!(b)$) (o2);
			\vertex[below=\FDHeightS of a] (exa);
			\vertex[below=\FDHeightS of b] (exb);
			\diagram*{
			(a) --[Eikonal,momentum=\(l\)] (o1) --[Eikonal,rmomentum=\(l\)] (o);
			(exa) --[fermion,momentum=\(P\)] (a);
			(b) --[fermion,momentum=\(P\)] (exb);
			(o1) --[gluon, half right,momentum'=\(l\)] (o);
			};
		\end{feynman}
	\end{tikzpicture}=-\frac{g_s^2C_F}{2}\delta(1-x)\int\mm{l}\frac{i}{l^2+i\epsilon}
	\frac{i}{l^z+i\epsilon}\frac{i}{-l^z+i\epsilon}
\end{align}
\subsection{Adding $\Gamma_g(l)$ and $\Gamma_g(-l)$}
\begin{align*}
	\Gamma_g(l)=\frac{\Gamma_g(l)+\Gamma_g(-l)}{2} & =-\frac{1}{2}\bqty{g_s^2C_F\delta(1-x)\int\mm{l}\frac{i}{l^2+i\epsilon}
		\frac{i}{i\epsilon}\frac{i}{-l^z+i\epsilon}+g_s^2C_F\delta(1-x)\int\mm{l}\frac{i}{l^2+i\epsilon}
		\frac{i}{i\epsilon}\frac{i}{l^z+i\epsilon}}\\
	                                               & =-\frac{1}{2}g_s^2C_F\delta(1-x)\int\mm{l}\frac{i}{l^2+i\epsilon}
	\frac{i}{i\epsilon}\bqty{\frac{i}{-l^z+i\epsilon}+\frac{i}{l^z+i\epsilon}}\\
	                                               & =-\frac{1}{2}g_s^2C_F\delta(1-x)\int\mm{l}\frac{i}{l^2+i\epsilon}
	\frac{i}{i\epsilon}\frac{2 \epsilon}{{l^z}^2+\epsilon^2}\\
	                                               & =g_s^2C_F\delta(1-x)\int\mm{l}\frac{i}{l^2+i\epsilon}\frac{i}{l^z+i\epsilon}\frac{i}{l^z-i\epsilon} \\
\end{align*}
There's an overall factor of $1/2$ missing.
\subsection{Taking derivatives}
Add a small momentum to the gauge link line and consider an actual self energy diagram
\begin{align}
	\begin{tikzpicture}[baseline=($(a)!0.5!(exa)$.base)]
		\begin{feynman}
			\node[dot] (a);
			\node[right=\FDWidthS of a,dot] (b);
			\vertex at ($(a)!0.6!(b)$) (o);
			\vertex at ($(a)!0.3!(b)$) (o1);
			\vertex at ($(a)!0.8!(b)$) (o2);
			\vertex[below=\FDHeightS of a] (exa);
			\vertex[below=\FDHeightS of b] (exb);
			\diagram*{
			(o1) --[Eikonal,momentum=\(k-l\)] (o);
			(exa) --[fermion,momentum=\(P\)] (a);
			(b) --[fermion,momentum=\(P\)] (exb);
			(o1) --[gluon, half right,momentum'=\(l\)] (o);
			};
		\end{feynman}
	\end{tikzpicture}=-g_s^2C_F\delta(1-x)\int\mm{l}\frac{i}{l^2+i\epsilon}
	\frac{i}{k^z-l^z+i\epsilon}
\end{align}
take the derivative
\begin{align}
	  & -g_s^2C_F\delta(1-x)\lim_{k^z\to0}\pdv{k^z}\bqty{(i)\int\mm{l}\frac{i}{l^2+i\epsilon}
		\frac{i}{k^z-l^z}}\\
	= & ig_s^2C_F\delta(1-x)\int\mm{l}\frac{i}{l^2+i\epsilon}
	\frac{i}{\pqty{l^z}^2}
\end{align}
Adding $i\epsilon$ by hand and we got
\begin{align}
	g_s^2C_F\delta(1-x)\int\mm{l}\frac{i}{l^2+i\epsilon}
	\frac{i}{l^z+i\epsilon}\frac{i}{l^z-i\epsilon}
\end{align}

\section{Two Loop}
In coordinate space
\begin{align}
	&\langle P, S| \bar{\psi}(z) \gamma^z
	\mathcal{P}\frac{\bqty{-ig_sn_{\mu}\int_0^\infty\dd z_1A^{a,\mu}(z_1)t^a}\bqty{-ig_sn_{\nu}\int_0^\infty\dd z_2A^{b,\nu}(z_2)t^b}\bqty{-ig_sn_{\rho}\int_0^\infty\dd z_3A^{c,\rho}(z_3)t^c}\bqty{-ig_sn_{\sigma}\int_0^\infty\dd z_4A^{d,\sigma}(z_4)t^d}}{4!}\notag\\
	&\hspace{6in}\psi(0)| P, S\rangle
\end{align}
\begin{align*}
	  & \frac{1}{4!}\mathcal{P}\bqty{\int_0^\infty\dd z_1A^{a,\mu}(z_1)}\bqty{\int_0^\infty\dd z_2A^{b,\nu}(z_2)}\bqty{\int_0^\infty\dd z_3A^{c,\rho}(z_3)}\bqty{\int_0^\infty\dd z_4A^{d,\sigma}(z_4)} \\&
	=\bqty{\int_0^\infty\dd z_1A^{a,\mu}(z_1)}\bqty{\int_0^\infty\dd z_2A^{b,\nu}(z_2)}\bqty{\int_0^\infty\dd z_3A^{c,\rho}(z_3)}\bqty{\int_0^\infty\dd z_4A^{d,\sigma}(z_4)}\theta(z_1-z_2)\theta(z_2-z_3)\theta(z_3-z_4)
\end{align*}
The coefficient of the above expression is
\begin{align}
	\mel{P}{\bar\psi(z)\gamma^z\psi(0)}{P}(-ig_sn^{\mu})(-ig_sn^{\nu})(-ig_sn^{\rho})(-ig_sn^{\sigma})t^at^bt^ct^d
\end{align}
take a trace
\begin{align}
	e^{-iP^zz}\Tr{(\slashed P+m)\gamma^z}\Tr{t^at^bt^ct^d}g_s^4n^{\mu}n^{\nu}n^{\rho}n^{\sigma}
\end{align}

\subsection{Diag. 50}
The amplitude for 
\begin{center}
	\begin{tikzpicture}[baseline=($(a)!0.5!(exa)$.base)]
		\begin{feynman}
			\node[dot] (a);
			\node[right=\FDWidth of a,dot] (b);
			\vertex at ($(a)!0.6!(b)$) (o);
			\vertex at ($(a)!0.2!(b)$) (o1);
			\vertex at ($(a)!0.4!(b)$) (o2);
			\vertex at ($(a)!0.6!(b)$) (o3);
			\vertex at ($(a)!0.8!(b)$) (o4);
			\vertex[below=\FDHeight of a] (exa);
			\vertex[below=\FDHeight of b] (exb);
			\diagram*{
			(a) --[Eikonal] (o1) --[Eikonal] (o2) --[Eikonal] (o3) --[Eikonal] (o4);
			(exa) --[fermion] (a);
			(b) --[fermion] (exb);
			(o1) --[gluon, half right,looseness=2] (o3);
			(o2) --[gluon, half right,looseness=2] (o4);
			};
		\end{feynman}
	\end{tikzpicture}
\end{center}
is related to the color ordering $t^at^bt^at^b$. 

\begin{align*}
	&\int_0^\infty\dd z_1\int_0^\infty\dd z_2\int_0^\infty\dd z_3\int_0^\infty\dd z_4\wick{\c1A^{a,\mu}(z_1)\c2A^{b,\nu}(z_2)\c1A^{c,\rho}(z_3)\c2A^{d,\sigma}(z_4)}\theta(z_1-z_2)\theta(z_2-z_3)\theta(z_3-z_4)\\
	&=\int_0^\infty\dd z_1\int_0^\infty\dd z_2\int_0^\infty\dd z_3\int_0^\infty\dd z_4\int\mm{l_1}\frac{-ig^{\mu\rho}\delta^{ac}}{l_1^2+i\epsilon}e^{-il_1\cdot(z_3-z_1)}\int\mm{l_2}\frac{-ig^{\nu\sigma}\delta^{bd}}{l_2^2+i\epsilon}e^{-il_2\cdot(z_4-z_2)}\\&\theta(z_1-z_2)\theta(z_2-z_3)\theta(z_3-z_4)\\
	&=\int\dd z_1\int\dd z_2\int\dd z_3\int\dd z_4\int\mm{l_1}\frac{-ig^{\mu\rho}\delta^{ac}}{l_1^2+i\epsilon}e^{-il_1\cdot(z_3-z_1)}\int\mm{l_2}\frac{-ig^{\nu\sigma}\delta^{bd}}{l_2^2+i\epsilon}e^{-il_2\cdot(z_4-z_2)}\\&\theta(z_1-z_2)\theta(z_2-z_3)\theta(z_3-z_4)\theta(z_4)
\end{align*}
The exponent is (for simplicity we assume vectors $z_i=(0,0,0,z_i)$ and $k_i=(0,0,0,-k_i)$) 
\begin{align*}
	&-il_1\cdot(z_3-z_1)-il_2\cdot(z_4-z_2)-ik_1\cdot(z_1-z_2)-ik_2\cdot(z_2-z_3)-ik_3\cdot(z_3-z_4)-ik_4\cdot z_4\\
	&=-iz_3\cdot(l_1+k_3-k_2)-iz_1\cdot(k_1-l_1)-iz_4\cdot(l_2+k_4-k_3)-iz_2\cdot(k_2-k_1-l_2)
\end{align*}
which gives 4 delta functions. The propagators involved are then
\begin{align*}
	\int\mm{l_1}\int\mm{l_2}\frac{-ig^{\mu\rho}\delta^{ac}}{l_1^2+i\epsilon}\frac{-ig^{\nu\sigma}\delta^{bd}}{l_2^2+i\epsilon}\frac{i}{-l_1^z+i\epsilon}\frac{i}{-l_1^z-l_2^z+i\epsilon}\frac{i}{-l_2^z+i\epsilon}\frac{i}{i\epsilon}
\end{align*}

\begin{align*}
	&\int\dd z_1\int\dd z_2\int\dd z_3\int\dd z_4\wick{\c1A^{a,\mu}(z_1)\c2A^{b,\nu}(z_2)\c1A^{c,\rho}(z_3)\c2A^{d,\sigma}(z_4)}\theta(z_1-z_2)\theta(z_2-z_3)\theta(z_3-z_4)\theta(z_4)\\
	&=\frac{1}{2}\int\dd z_1\int\dd z_2\int\dd z_3\int\dd z_4\wick{\c1A^{a,\mu}(z_1)\c2A^{b,\nu}(z_2)\c1A^{c,\rho}(z_3)\c2A^{d,\sigma}(z_4)}\bqty{
		\theta(z_1-z_2)\theta(z_2-z_3)\theta(z_3-z_4)\theta(z_4)\\&+\theta(z_3-z_4)\theta(z_4-z_1)\theta(z_1-z_2)\theta(z_2)
	}\\
	&=\int\mm{l_1}\int\mm{l_2}\frac{-ig^{\mu\rho}\delta^{ac}}{l_1^2+i\epsilon}\frac{-ig^{\nu\sigma}\delta^{bd}}{l_2^2+i\epsilon}\frac{ -\left({l_1^z}^2+3 l_1^z l_2^z+{l_2^z}^2-\epsilon ^2\right)}{(l_1^z-i \epsilon ) (l_1^z+i \epsilon ) ( l_2^z+i\epsilon ) ( l_2^z-i\epsilon ) (l_1^z+l_2^z-i \epsilon ) (l_1^z+l_2^z+i \epsilon )}
\end{align*}
Taking the derivative of the former expression, we can also arrive at a divergence-free form
\begin{align*}
	\frac{i}{2}\lim_{p\to0}\pdv{p}\bqty{\frac{i}{p+l_1^z}\frac{i}{p+l_1^z+l_2^z}\frac{i}{p+l_2^z}}= \frac{-\pqty{{l_1^z}^2+3l_1^zl_2^z+{l_2^z}^2}}{{l_1^z}^2{l_2^z}^2(l_1^z+l_2^z)^2}
\end{align*}
which is equivalent to above expression. 

Adding $\epsilon$ in the definition so that the definition is $\mathcal{P}e^{-ig_s\int_0^\infty e^{-z \epsilon}n\cdot A^a(z)t^a}$, the expression becomes
\begin{align*}
	&\int\dd z_1\int\dd z_2\int\dd z_3\int\dd z_4\wick{\c1A^{a,\mu}(z_1)\c2A^{b,\nu}(z_2)\c1A^{c,\rho}(z_3)\c2A^{d,\sigma}(z_4)}e^{-(z_1+z_2+z_3+z_4)\epsilon}\theta(z_1-z_2)\theta(z_2-z_3)\theta(z_3-z_4)\theta(z_4)\\
	&=\frac{1}{2}\int\dd z_1\int\dd z_2\int\dd z_3\int\dd z_4\wick{\c1A^{a,\mu}(z_1)\c2A^{b,\nu}(z_2)\c1A^{c,\rho}(z_3)\c2A^{d,\sigma}(z_4)}e^{-(z_1+z_2+z_3+z_4)\epsilon}\bqty{
		\theta(z_1-z_2)\theta(z_2-z_3)\theta(z_3-z_4)\theta(z_4)\\&+\theta(z_3-z_4)\theta(z_4-z_1)\theta(z_1-z_2)\theta(z_2)
	}\\
	&=\int\mm{l_1}\int\mm{l_2}\frac{-ig^{\mu\rho}\delta^{ac}}{l_1^2+i\epsilon}\frac{-ig^{\nu\sigma}\delta^{bd}}{l_2^2+i\epsilon}\frac{-3 {l_1^z}^2-6 {l_1^z} {l_2^z}-{l_2^z}^2}{4 ({l_1^z}-i \epsilon ) ({l_1^z}+i \epsilon ) (3 \epsilon -i {l_2^z}) (3 \epsilon +i {l_2^z}) ({l_1^z}+{l_2^z}-2 i \epsilon ) ({l_1^z}+{l_2^z}+2 i \epsilon )}
\end{align*}

The full amplitude in coordinate space is
\begin{align}
	&\begin{aligned}[t]
		&4P^ze^{-iP^zz}g_s^4\Tr{t^at^bt^ct^d}n^{\mu}n^{\nu}n^{\rho}n^{\sigma}\\&\int\mm{l_1}\int\mm{l_2}\frac{-ig^{\mu\rho}\delta^{ac}}{l_1^2+i\epsilon}\frac{-ig^{\nu\sigma}\delta^{bd}}{l_2^2+i\epsilon}\frac{-3 {l_1^z}^2-6 {l_1^z} {l_2^z}-{l_2^z}^2}{4 ({l_1^z}-i \epsilon ) ({l_1^z}+i \epsilon ) (3 \epsilon -i {l_2^z}) (3 \epsilon +i {l_2^z}) ({l_1^z}+{l_2^z}-2 i \epsilon ) ({l_1^z}+{l_2^z}+2 i \epsilon )}
	\end{aligned}\\
	=&\begin{aligned}[t]
		&4P^ze^{-iP^zz}g_s^4\Tr{t^at^bt^at^b}\\&\int\mm{l_1}\int\mm{l_2}\frac{-in^2}{l_1^2+i\epsilon}\frac{-in^2}{l_2^2+i\epsilon}\frac{-3 {l_1^z}^2-6 {l_1^z} {l_2^z}-{l_2^z}^2}{4 ({l_1^z}-i \epsilon ) ({l_1^z}+i \epsilon ) (3 \epsilon -i {l_2^z}) (3 \epsilon +i {l_2^z}) ({l_1^z}+{l_2^z}-2 i \epsilon ) ({l_1^z}+{l_2^z}+2 i \epsilon )}
	\end{aligned}\\
	=&\begin{aligned}[t]
		&4P^ze^{-iP^zz}g_s^4\Tr{t^at^bt^at^b}\\&\int\mm{l_1}\int\mm{l_2}\frac{i}{l_1^2+i\epsilon}\frac{i}{l_2^2+i\epsilon}\frac{-3 {l_1^z}^2-6 {l_1^z} {l_2^z}-{l_2^z}^2}{4 ({l_1^z}-i \epsilon ) ({l_1^z}+i \epsilon ) (  {l_2^z}+3 i\epsilon) (  {l_2^z}-3 i\epsilon) ({l_1^z}+{l_2^z}-2 i \epsilon ) ({l_1^z}+{l_2^z}+2 i \epsilon )}
	\end{aligned}
\end{align}

\subsection{Diag. 37}
The amplitude for 
\begin{center}
	\begin{tikzpicture}[baseline=($(a)!0.5!(exa)$.base)]
		\begin{feynman}
			\node[dot] (a);
			\node[right=\FDWidth of a,dot] (b);
			\vertex at ($(a)!0.6!(b)$) (o);
			\vertex at ($(a)!0.2!(b)$) (o1);
			\vertex at ($(a)!0.4!(b)$) (o2);
			\vertex at ($(a)!0.6!(b)$) (o3);
			\vertex at ($(a)!0.8!(b)$) (o4);
			\vertex[below=\FDHeight of a] (exa);
			\vertex[below=\FDHeight of b] (exb);
			\diagram*{
			(a) --[Eikonal] (o1) --[Eikonal] (o2) --[Eikonal] (o3) --[Eikonal] (o4);
			(exa) --[fermion] (a);
			(b) --[fermion] (exb);
			(o1) --[gluon, half right,looseness=2] (o4);
			(o2) --[gluon, half right,looseness=2] (o3);
			};
		\end{feynman}
	\end{tikzpicture}
\end{center}
is related to the color ordering $t^at^bt^bt^a$. 
\begin{align*}
	&\int_0^\infty\dd z_1\int_0^\infty\dd z_2\int_0^\infty\dd z_3\int_0^\infty\dd z_4\wick{\c1A^{a,\mu}(z_1)\c2A^{b,\nu}(z_2)\c2A^{c,\rho}(z_3)\c1A^{d,\sigma}(z_4)}\theta(z_1-z_2)\theta(z_2-z_3)\theta(z_3-z_4)\\
	&=\int\dd z_1\int\dd z_2\int\dd z_3\int\dd z_4\int\mm{l_1}\frac{-ig^{\mu\sigma}\delta^{ad}}{l_1^2+i\epsilon}e^{-il_1\cdot(z_3-z_1)}\int\mm{l_2}\frac{-ig^{\nu\rho}\delta^{bc}}{l_2^2+i\epsilon}e^{-il_2\cdot(z_4-z_2)}\\&\theta(z_1-z_2)\theta(z_2-z_3)\theta(z_3-z_4)\theta(z_4)
\end{align*}
The propagators involved are 
\begin{align*}
	\int\mm{l_1}\int\mm{l_2}\frac{-ig^{\mu\sigma}\delta^{ad}}{l_1^2+i\epsilon}\frac{-ig^{\nu\rho}\delta^{bc}}{l_2^2+i\epsilon}\frac{i}{-l_1^z+i\epsilon}\frac{i}{-l_1^z-l_2^z+i\epsilon}\frac{i}{-l_1^z+i\epsilon}\frac{i}{i\epsilon}
\end{align*}
\begin{align*}
	&\begin{aligned}
		\frac{1}{2}\int\dd z_1\int\dd z_2\int\dd z_3\int\dd z_4\wick{\c1A^{a,\mu}(z_1)\c2A^{b,\nu}(z_2)\c2A^{c,\rho}(z_3)\c1A^{d,\sigma}(z_4)}\bqty{\theta(z_1-z_2)\theta(z_2-z_3)\theta(z_3-z_4)\theta(z_4)\\+\theta(z_4-z_3)\theta(z_3-z_2)\theta(z_2-z_1)\theta(z_1)}
	\end{aligned}\\
	&=\int\mm{l_1}\int\mm{l_2}\frac{-ig^{\mu\sigma}\delta^{ad}}{l_1^2+i\epsilon}\frac{-ig^{\nu\rho}\delta^{bc}}{l_2^2+i\epsilon}\frac{-3 {l_1^z}^2-2 l_1^z l_2^z+\epsilon ^2}{\left({l_1^z}^2+\epsilon ^2\right)^2 \left({l_1^z}^2+2 l_1^z l_2^z+{l_2^z}^2+\epsilon ^2\right)}
\end{align*}
Taking the derivative of the former expression, we can also arrive at a divergence-free form
\begin{align*}
	\frac{i}{2}\lim_{p\to0}\pdv{p}\bqty{\frac{i}{p+l_1^z}\frac{i}{p+l_1^z+l_2^z}\frac{i}{p+l_1^z}}= \frac{-\pqty{3l_1^z+2l_2^z}}{{l_1^z}^3(l_1^z+l_2^z)^2}
\end{align*}
which is equivalent to above expression. 
\begin{align*}
	\frac{-3 {l_1^z}^2-2 {l_1^z} {l_2^z}+3 \epsilon ^2}{2({l_1^z}-i \epsilon ) ({l_1^z}+i \epsilon ) ({l_1^z}-3 i \epsilon ) ({l_1^z}+3 i \epsilon ) ({l_1^z}+{l_2^z}-2 i \epsilon ) ({l_1^z}+{l_2^z}+2 i \epsilon )}
\end{align*}

The full amplitude in coordinate space is
\begin{align}
	&\begin{aligned}[t]
		&4P^ze^{-iP^zz}g_s^4\Tr{t^at^bt^ct^d}n^{\mu}n^{\nu}n^{\rho}n^{\sigma}\\&\int\mm{l_1}\int\mm{l_2}\frac{-ig^{\mu\sigma}\delta^{ad}}{l_1^2+i\epsilon}\frac{-ig^{\nu\rho}\delta^{bc}}{l_2^2+i\epsilon}\frac{-3 {l_1^z}^2-2 {l_1^z} {l_2^z}+3 \epsilon ^2}{2({l_1^z}-i \epsilon ) ({l_1^z}+i \epsilon ) ({l_1^z}-3 i \epsilon ) ({l_1^z}+3 i \epsilon ) ({l_1^z}+{l_2^z}-2 i \epsilon ) ({l_1^z}+{l_2^z}+2 i \epsilon )}
	\end{aligned}\\
	=&\begin{aligned}[t]
		&4P^ze^{-iP^zz}g_s^4\Tr{t^at^bt^bt^a}\\&\int\mm{l_1}\int\mm{l_2}\frac{-in^2}{l_1^2+i\epsilon}\frac{-in^2}{l_2^2+i\epsilon}\frac{-3 {l_1^z}^2-2 {l_1^z} {l_2^z}+3 \epsilon ^2}{2({l_1^z}-i \epsilon ) ({l_1^z}+i \epsilon ) ({l_1^z}-3 i \epsilon ) ({l_1^z}+3 i \epsilon ) ({l_1^z}+{l_2^z}-2 i \epsilon ) ({l_1^z}+{l_2^z}+2 i \epsilon )}
	\end{aligned}\\
	=&\begin{aligned}[t]
		&4P^ze^{-iP^zz}g_s^4\Tr{t^at^bt^bt^a}\\&\int\mm{l_1}\int\mm{l_2}\frac{i}{l_1^2+i\epsilon}\frac{i}{l_2^2+i\epsilon}\frac{-3 {l_1^z}^2-2 {l_1^z} {l_2^z}}{2({l_1^z}-i \epsilon ) ({l_1^z}+i \epsilon ) ({l_1^z}-3 i \epsilon ) ({l_1^z}+3 i \epsilon ) ({l_1^z}+{l_2^z}-2 i \epsilon ) ({l_1^z}+{l_2^z}+2 i \epsilon )}
	\end{aligned}
\end{align}

\subsection{Diag. 43}
The amplitude for 
\begin{center}
	\begin{tikzpicture}[transform shape,scale=1,baseline=($(a)!0.5!(exa)$.base)]
		\begin{feynman}
			\node[dot] (a);
			\node[right=\FDWidth of a,dot] (b);
			\vertex[below=\FDHeight of a] (exa);
			\vertex[below=\FDHeight of b] (exb);
			% 
			\vertex at ($(a)!0.33!(b)$) (o13);
			\vertex at ($(a)!0.66!(b)$) (o23);
			\vertex at ($(a)!0.25!(b)$) (o14);
			\vertex at ($(a)!0.5!(b)$) (o12);
			\vertex at ($(a)!0.75!(b)$) (o34);
			% 
			\vertex at ($(a)!0.2!(b)$) (o15);
			\vertex at ($(a)!0.4!(b)$) (o25);
			\vertex at ($(a)!0.6!(b)$) (o35);
			\vertex at ($(a)!0.8!(b)$) (o45);
			% HOWTO: One can modify this to get vertexes lower than the gauge link line. 
			\vertex at ($(o13)+1/4*(0,-\FDHeight)$) (o13m);
			\vertex at ($(o23)+1/4*(0,-\FDHeight)$) (o23m);
			% 
			\vertex at ($(exa)!0.25!(a)$) (a14);
			\vertex at ($(exa)!0.5!(a)$) (a12);
			\vertex at ($(exa)!0.75!(a)$) (a34);
			\vertex at ($(exa)!0.33!(a)$) (a13);
			\vertex at ($(exa)!0.66!(a)$) (a23);
			% 
			\vertex at ($(exb)!0.25!(b)$) (b14);
			\vertex at ($(exb)!0.5!(b)$) (b12);
			\vertex at ($(exb)!0.75!(b)$) (b34);
			\vertex at ($(exb)!0.33!(b)$) (b13);
			\vertex at ($(exb)!0.66!(b)$) (b23);
			% 
			\vertex at ($(a12)!0.33!(b12)$) (ab13);
			\vertex at ($(a12)!0.66!(b12)$) (ab23);
			\vertex at ($(a12)!0.5!(b12)$) (ab12);
			% 
			\vertex at ($(a14)!0.5!(a34)!1.2!-90:(a34)$) (g1);
			% 
			\diagram*{
			(a) --[Eikonal] (o15) --[Eikonal] (o25) --[Eikonal] (o35) --[Eikonal] (o45);
			(exa) --[fermion] (a);
			(b) --[fermion] (exb);
			(o15) --[gluon,half right] (o25);
			(o35) --[gluon,half right] (o45);
			};
		\end{feynman}
	\end{tikzpicture}
\end{center}
is related to the color ordering $t^at^at^bt^b$. 
\begin{align*}
	&\int_0^\infty\dd z_1\int_0^\infty\dd z_2\int_0^\infty\dd z_3\int_0^\infty\dd z_4\wick{\c1A^{a,\mu}(z_1)\c1A^{b,\nu}(z_2)\c1A^{c,\rho}(z_3)\c1A^{d,\sigma}(z_4)}\theta(z_1-z_2)\theta(z_2-z_3)\theta(z_3-z_4)\\
	&=\int\dd z_1\int\dd z_2\int\dd z_3\int\dd z_4\int\mm{l_1}\frac{-ig^{\mu\sigma}\delta^{ad}}{l_1^2+i\epsilon}e^{-il_1\cdot(z_2-z_1)}\int\mm{l_2}\frac{-ig^{\nu\rho}\delta^{bc}}{l_2^2+i\epsilon}e^{-il_2\cdot(z_4-z_3)}\\&\theta(z_1-z_2)\theta(z_2-z_3)\theta(z_3-z_4)\theta(z_4)
\end{align*}
The propagators involved are 
\begin{align*}
	\int\mm{l_1}\int\mm{l_2}\frac{-ig^{\mu\sigma}\delta^{ad}}{l_1^2+i\epsilon}\frac{-ig^{\nu\rho}\delta^{bc}}{l_2^2+i\epsilon}\frac{i}{-l_1^z+i\epsilon}\frac{i}{i\epsilon}\frac{i}{-l_2^z+i\epsilon}\frac{i}{i\epsilon}
\end{align*}
\begin{align*}
	&\begin{aligned}
		\frac{1}{4}\int\dd z_1\int\dd z_2\int\dd z_3\int\dd z_4\wick{\c1A^{a,\mu}(z_1)\c1A^{b,\nu}(z_2)\c1A^{c,\rho}(z_3)\c1A^{d,\sigma}(z_4)}\bqty{\theta(z_1-z_2)\theta(z_2-z_3)\theta(z_3-z_4)\theta(z_4)\\+\theta(z_2-z_1)\theta(z_1-z_3)\theta(z_3-z_4)\theta(z_4)+\theta(z_1-z_2)\theta(z_2-z_4)\theta(z_4-z_3)\theta(z_3)+\theta(z_2-z_1)\theta(z_1-z_4)\theta(z_4-z_3)\theta(z_3)}
	\end{aligned}\\
	&=\int\mm{l_1}\int\mm{l_2}\frac{-ig^{\mu\sigma}\delta^{ab}}{l_1^2+i\epsilon}\frac{-ig^{\nu\rho}\delta^{cd}}{l_2^2+i\epsilon}\frac{1}{\left({l_1^z}^2+\epsilon ^2\right) \left({l_2^z}^2+\epsilon ^2\right)}
\end{align*}
\begin{align*}
	&\int_0^\infty\dd z_1\int_0^\infty\dd z_2\int_0^\infty\dd z_3\int_0^\infty\dd z_4\wick{\c1A^{a,\mu}(z_1)\c1A^{b,\nu}(z_2)\c1A^{c,\rho}(z_3)\c1A^{d,\sigma}(z_4)}e^{-(z_1+z_2+z_3+z_4)\epsilon}\theta(z_1-z_2)\theta(z_2-z_3)\theta(z_3-z_4)\\
	&=\int\dd z_1\int\dd z_2\int\dd z_3\int\dd z_4\int\mm{l_1}\frac{-ig^{\mu\sigma}\delta^{ab}}{l_1^2+i\epsilon}e^{-il_1\cdot(z_2-z_1)}\int\mm{l_2}\frac{-ig^{\nu\rho}\delta^{cd}}{l_2^2+i\epsilon}e^{-il_2\cdot(z_4-z_3)}e^{-(z_1+z_2+z_3+z_4)\epsilon}\\&\theta(z_1-z_2)\theta(z_2-z_3)\theta(z_3-z_4)\theta(z_4)
\end{align*}
and the eikonal part is
\begin{align*}
	\frac{3}{8 ({l_1^z}-i \epsilon ) ({l_1^z}+i \epsilon ) (3 \epsilon -i {l_2^z}) (3 \epsilon +i {l_2^z})}
\end{align*}

The full amplitude in coordinate space is
\begin{align}
	&\begin{aligned}[t]
		&4P^ze^{-iP^zz}g_s^4\Tr{t^at^bt^ct^d}n^{\mu}n^{\nu}n^{\rho}n^{\sigma}\int\mm{l_1}\int\mm{l_2}\frac{-ig^{\mu\sigma}\delta^{ab}}{l_1^2+i\epsilon}\frac{-ig^{\nu\rho}\delta^{cd}}{l_2^2+i\epsilon}\frac{3}{8 ({l_1^z}-i \epsilon ) ({l_1^z}+i \epsilon ) (3 \epsilon -i {l_2^z}) (3 \epsilon +i {l_2^z})}
	\end{aligned}\\
	=&\begin{aligned}[t]
		&4P^ze^{-iP^zz}g_s^4\Tr{t^at^at^bt^b}\int\mm{l_1}\int\mm{l_2}\frac{-in^2}{l_1^2+i\epsilon}\frac{-in^2}{l_2^2+i\epsilon}\frac{3}{8 ({l_1^z}-i \epsilon ) ({l_1^z}+i \epsilon ) (3 \epsilon -i {l_2^z}) (3 \epsilon +i {l_2^z})}
	\end{aligned}\\
	=&\begin{aligned}[t]
		&4P^ze^{-iP^zz}g_s^4\Tr{t^at^at^bt^b}\int\mm{l_1}\int\mm{l_2}\frac{i}{l_1^2+i\epsilon}\frac{i}{l_2^2+i\epsilon}\frac{3}{8 ({l_1^z}-i \epsilon ) ({l_1^z}+i \epsilon ) ( {l_2^z}-3i \epsilon ) ( {l_2^z}+3i \epsilon )}
	\end{aligned}
\end{align}

\subsection{Diag. 47}
The amplitude for 
\begin{center}
	\begin{tikzpicture}[transform shape,scale=1,baseline=($(a)!0.5!(exa)$.base)]
		\begin{feynman}
			\node[dot] (a);
			\node[right=\FDWidth of a,dot] (b);
			\vertex[below=\FDHeight of a] (exa);
			\vertex[below=\FDHeight of b] (exb);
			% 
			\vertex at ($(a)!0.33!(b)$) (o13);
			\vertex at ($(a)!0.66!(b)$) (o23);
			\vertex at ($(a)!0.25!(b)$) (o14);
			\vertex at ($(a)!0.5!(b)$) (o12);
			\vertex at ($(a)!0.75!(b)$) (o34);
			% 
			\vertex at ($(exa)!0.25!(a)$) (a14);
			\vertex at ($(exa)!0.5!(a)$) (a12);
			\vertex at ($(exa)!0.75!(a)$) (a34);
			% 
			\vertex at ($(exb)!0.25!(b)$) (b14);
			\vertex at ($(exb)!0.5!(b)$) (b12);
			\vertex at ($(exb)!0.75!(b)$) (b34);
			% 
			\vertex at ($(a12)!0.33!(b12)$) (ab13);
			\vertex at ($(a12)!0.66!(b12)$) (ab23);
			\vertex at ($(a12)!0.5!(b12)$) (ab12);
			% 
			\vertex at ($(a14)!0.5!(a34)!1.2!-90:(a34)$) (g1);
			\vertex at ($(ab12)+(0,1/5*\FDHeight)$) (ab123);
			% 
			\diagram*{
			(a) --[Eikonal] (o14) --[Eikonal] (o12) --[Eikonal] (o34);
			(exa) --[fermion] (a);
			(b) --[fermion] (exb);
			(o14) --[gluon, quarter right] (ab123);
			(o12) --[gluon] (ab123);
			(o34) --[gluon, quarter left] (ab123);
			};
		\end{feynman}
	\end{tikzpicture}
\end{center}
is related to
\begin{align}
	\langle P, S| \bar{\psi}(z) \gamma^z V_3
	\mathcal{P}\frac{\bqty{-ig_sn_{\mu}\int_0^\infty\dd z_1A^{a,\mu}(z_1)t^a}\bqty{-ig_sn_{\nu}\int_0^\infty\dd z_2A^{b,\nu}(z_2)t^b}\bqty{-ig_sn_{\rho}\int_0^\infty\dd z_3A^{c,\rho}(z_3)t^c}}{3!}
	\psi(0)| P, S\rangle
\end{align}
where 
\begin{align*}
	V_3=-\frac{g_s}{2} f^{d e f}\int\dd^4 t\left(\partial^{\alpha} A_{d}^{\beta}-\partial^{\beta} A_{d}^{\alpha}\right) A_{\alpha}^{e} A_{\beta}^{f}
\end{align*}
The gluon related contraction (one out of three, others can be obtained by exchanging $d,e,f$. )
\begin{align*}
	&\int\dd z_1\int\dd z_2\int\dd z_3\int\dd^4 t\wick{\c1{\left(\partial^{\alpha} A^{d,\beta}-\partial^{\beta} A^{d,\alpha}\right)} \c2A^{e}_{\alpha} \c3A^{f}_{\beta}\c2A^{a,\mu}(z_1)\c3A^{b,\nu}(z_2)\c1A^{c,\rho}(z_3)}\theta(z_1-z_2)\theta(z_2-z_3)\theta(z_3)\\ 
	&=\int\dd z_1\int\dd z_2\int\dd z_3\int\dd^4 t
	\int\mm{l_1}\frac{-i\delta^{dc}}{l_1^2+i\epsilon}\bqty{g^{\beta\rho}\partial^{\alpha}e^{-il_1\cdot(z_3-t)}-g^{\alpha\rho}\partial^{\beta}e^{-il_1\cdot(z_3-t)}}
	\int\mm{l_2}\frac{-ig_{\alpha}^{\mu}\delta^{ea}}{l_2^2+i\epsilon}e^{-il_2\cdot(z_1-t)}\\&
	\int\mm{l_3}\frac{-ig_{\beta}^{\nu}\delta^{fb}}{l_3^2+i\epsilon}e^{-il_3\cdot(z_2-t)}
	\theta(z_1-z_2)\theta(z_2-z_3)\theta(z_3)\\ 
	&=\int\dd z_1\int\dd z_2\int\dd z_3\int\dd^4 t
	\int\mm{l_1}\frac{-i\delta^{dc}}{l_1^2+i\epsilon}\bqty{g^{\beta\rho}il_1^{\alpha}-g^{\alpha\rho}il_1^{\beta}}e^{-il_1\cdot(z_3-t)}
	\int\mm{l_2}\frac{-ig_{\alpha}^{\mu}\delta^{ea}}{l_2^2+i\epsilon}e^{-il_2\cdot(z_1-t)}\\&
	\int\mm{l_3}\frac{-ig_{\beta}^{\nu}\delta^{fb}}{l_3^2+i\epsilon}e^{-il_3\cdot(z_2-t)}
	\theta(z_1-z_2)\theta(z_2-z_3)\theta(z_3)\\ 
	&=\int\dd z_1\int\dd z_2\int\dd z_3\int\dd^4 t
	\int\mm{l_1}\frac{-i\delta^{dc}}{l_1^2+i\epsilon}\bqty{g^{\nu\rho}il_1^{\mu}-g^{\mu\rho}il_1^{\nu}}e^{-il_1\cdot(z_3-t)}
	\int\mm{l_2}\frac{-i\delta^{ea}}{l_2^2+i\epsilon}e^{-il_2\cdot(z_1-t)}\\&
	\int\mm{l_3}\frac{-i\delta^{fb}}{l_3^2+i\epsilon}e^{-il_3\cdot(z_2-t)}
	\theta(z_1-z_2)\theta(z_2-z_3)\theta(z_3)
\end{align*}
Multiplied by $n$
\begin{align*}
	n_{\mu}n_{\nu}n_{\rho}\bqty{g^{\nu\rho}il_1^{\mu}-g^{\mu\rho}il_1^{\nu}}=in^2\bqty{l_1^z-l_1^z}=0
\end{align*}
Contracting with different fields in the parenthesis
\begin{align*}
	\wick{\partial^{\alpha} \c1A^{d,\beta} \c2A^{e}_{\alpha} \c3A^{f}_{\beta}\c2A^{a,\mu}(z_1)\c3A^{b,\nu}(z_2)\c1A^{c,\rho}(z_3)}-\wick{\partial^{\beta} \c1A^{d,\alpha} \c2A^{e}_{\alpha} \c3A^{f}_{\beta}\c2A^{a,\mu}(z_1)\c1A^{b,\nu}(z_2)\c3A^{c,\rho}(z_3)}
\end{align*}
\begin{align*}
	&\int\dd z_1\int\dd z_2\int\dd z_3\int\dd^4 t
	\int\mm{l_1}\int\mm{l_2}\int\mm{l_3}\frac{-i\delta^{dc}}{l_3^2+i\epsilon}\frac{-ig_{\alpha}^{\mu}\delta^{ea}}{l_1^2+i\epsilon}e^{-il_1\cdot(z_1-t)}\frac{-ig_{\beta}^{\nu}\delta^{fb}}{l_2^2+i\epsilon}\\&
	\bqty{e^{-il_2\cdot(z_2-t)}g^{\beta\rho}\partial^{\alpha}e^{-il_3\cdot(z_3-t)}-e^{-il_3\cdot(z_3-t)}g^{\alpha\rho}\partial^{\beta}e^{-il_2\cdot(z_2-t)}}
	\theta(z_1-z_2)\theta(z_2-z_3)\theta(z_3)\\ 
	&=\int\dd z_1\int\dd z_2\int\dd z_3\int\dd^4 t
	\int\mm{l_1}\int\mm{l_2}\int\mm{l_3}\frac{-i\delta^{dc}}{l_3^2+i\epsilon}\frac{-i\delta^{ea}}{l_1^2+i\epsilon}\frac{-i\delta^{fb}}{l_2^2+i\epsilon}e^{-il_1\cdot(z_1-t)}e^{-il_2\cdot(z_2-t)}e^{-il_3\cdot(z_3-t)}\\&
	\bqty{g^{\nu\rho}il_3^{\mu}-g^{\mu\rho}il_2^{\nu}}
	\theta(z_1-z_2)\theta(z_2-z_3)\theta(z_3)
\end{align*}
Make $z_1\to-z_1,t\to-t,l_2\to-l_2$, 
\begin{align*}
	&\int\dd z_1\int\dd z_2\int\dd z_3\int\dd^4 t
	\int\mm{l_1}\int\mm{l_2}\int\mm{l_3}\frac{-i\delta^{dc}}{l_3^2+i\epsilon}\frac{-i\delta^{ea}}{l_1^2+i\epsilon}\frac{-i\delta^{fb}}{l_2^2+i\epsilon}e^{-il_1\cdot(z_1-t)}e^{-il_2\cdot(z_2-t)}e^{-il_3\cdot(z_3-t)}\\&
	\bqty{g^{\nu\rho}il_3^{\mu}-g^{\mu\rho}il_2^{\nu}}
	\theta(-z_1-z_2)\theta(z_2-z_3)\theta(z_3)\\
	&=\int\mm{l_2}\int\mm{l_3}\frac{-i\delta^{dc}}{l_3^2+i\epsilon}\frac{-i\delta^{ea}}{(l_3+l_2)^2+i\epsilon}\frac{-i\delta^{fb}}{l_2^2+i\epsilon}\frac{i}{-l_3^z+i\epsilon}\frac{i}{-2l_3^z+i\epsilon}\frac{i}{-2l_3^z-l_2^z+i\epsilon}\bqty{g^{\nu\rho}il_3^{\mu}-g^{\mu\rho}il_2^{\nu}}
\end{align*}
\begin{align*}
	\wick{\partial^{\alpha} \c1A^{d,\beta} \c2A^{e}_{\alpha} \c3A^{f}_{\beta}\c2A^{a,\mu}(z_1)\c1A^{b,\nu}(z_2)\c3A^{c,\rho}(z_3)}-\wick{\partial^{\beta} \c1A^{d,\alpha} \c2A^{e}_{\alpha} \c3A^{f}_{\beta}\c1A^{a,\mu}(z_1)\c2A^{b,\nu}(z_2)\c3A^{c,\rho}(z_3)}
\end{align*}
\begin{align*}
	&\int\dd z_1\int\dd z_2\int\dd z_3\int\dd^4 t\int\mm{l_1}\int\mm{l_2}\int\mm{l_3}\frac{-i\delta^{dc}}{l_3^2+i\epsilon}\frac{-ig_{\alpha}^{\mu}\delta^{ea}}{l_1^2+i\epsilon}e^{-il_1\cdot(z_1-t)}\frac{-ig_{\beta}^{\nu}\delta^{fb}}{l_2^2+i\epsilon}\\&
	\bqty{e^{-il_2\cdot(z_2-t)}g^{\beta\rho}\partial^{\alpha}e^{-il_3\cdot(z_3-t)}-e^{-il_3\cdot(z_3-t)}g^{\alpha\rho}\partial^{\beta}e^{-il_2\cdot(z_2-t)}}
	\theta(z_1-z_2)\theta(z_2-z_3)\theta(z_3)
\end{align*}



\begin{center}
	\begin{tikzpicture}[transform shape,scale=1,baseline=($(a)!0.5!(exa)$.base)]
		\begin{feynman}
			\node[dot] (a);
			\node[right=\FDWidth of a,dot] (b);
			\vertex[below=\FDHeight of a] (exa);
			\vertex[below=\FDHeight of b] (exb);
			% 
			\vertex at ($(a)!0.33!(b)$) (o13);
			\vertex at ($(a)!0.66!(b)$) (o23);
			\vertex at ($(a)!0.25!(b)$) (o14);
			\vertex at ($(a)!0.5!(b)$) (o12);
			\vertex at ($(a)!0.75!(b)$) (o34);
			% 
			\vertex at ($(a)!0.2!(b)$) (o15);
			\vertex at ($(a)!0.4!(b)$) (o25);
			\vertex at ($(a)!0.6!(b)$) (o35);
			\vertex at ($(a)!0.8!(b)$) (o45);
			% 
			\vertex at ($(exa)!0.25!(a)$) (a14);
			\vertex at ($(exa)!0.5!(a)$) (a12);
			\vertex at ($(exa)!0.75!(a)$) (a34);
			% 
			\vertex at ($(exb)!0.25!(b)$) (b14);
			\vertex at ($(exb)!0.5!(b)$) (b12);
			\vertex at ($(exb)!0.75!(b)$) (b34);
			% 
			\vertex at ($(a12)!0.33!(b12)$) (ab13);
			\vertex at ($(a12)!0.66!(b12)$) (ab23);
			\vertex at ($(a12)!0.5!(b12)$) (ab12);
			% 
			\vertex at ($(a14)!0.5!(a34)!1.2!-90:(a34)$) (g1);
			\vertex at ($(o12)!0.66!(a12)$) (g13);
			\vertex at ($(o12)!0.33!(a12)$) (g23);
			% 
			\diagram*{
			(o14) --[Eikonal] (o12) --[Eikonal] (o34) --[Eikonal] (b);
			(exa) --[fermion] (a12) --[fermion] (a);
			(b) --[fermion] (exb);
			(o34) --[gluon] (a12);
			(o12) --[gluon,half right] (o14);
			};
		\end{feynman}
	\end{tikzpicture}
\end{center}
is related to
\begin{align}
	\langle P, S| \bar{\psi}(z) \gamma^z Q_3
	\mathcal{P}\frac{\bqty{-ig_sn_{\mu}\int_0^\infty\dd z_1A^{a,\mu}(z_1)t^a}\bqty{-ig_sn_{\nu}\int_0^\infty\dd z_2A^{b,\nu}(z_2)t^b}\bqty{-ig_sn_{\rho}\int_0^\infty\dd z_3A^{c,\rho}(z_3)t^c}}{3!}
	\psi(0)| P, S\rangle
\end{align}
where
\begin{align*}
	Q_3=(-ig_s\gamma_{\sigma})\int\dd^4 t \bar\psi\psi A^{d,\sigma}
\end{align*}
\begin{align*}
	&\int\dd z_1\int\dd z_2\int\dd z_3\wick{\c1A^{d,\sigma}(t) \c2A^{a,\mu}(z_1)\c2A^{b,\nu}(z_2)\c1A^{c,\rho}(z_3)}\theta(z_1-z_2)\theta(z_2-z_3)\theta(z_3)\\ 
	&=\int\dd z_1\int\dd z_2\int\dd z_3
	\int\mm{l_1}\int\mm{l_2}\frac{-ig^{\mu\nu}\delta^{ab}}{l_1^2+i\epsilon}\frac{-ig^{\sigma\rho}\delta^{dc}}{l_2^2+i\epsilon}e^{-il_1\cdot(z_2-z_1)}e^{-il_2\cdot(z_3-t)}
	\theta(z_1-z_2)\theta(z_2-z_3)\theta(z_3)\\ 
\end{align*}
The eikonal propagators are
\begin{align*}
	\frac{i}{-l_1^z+i\epsilon}\frac{i}{i\epsilon}\frac{i}{l_2^z+i\epsilon}
\end{align*}
Flip $z_1$ and $z_2$
\begin{align}
	-\frac{i}{-l_1^z+i\epsilon}\frac{i}{l_1^z+i\epsilon}\frac{i}{l_2^z+i\epsilon}
\end{align}
This covers for all diagrams involved one gauge field contracting with the would-be-divergent three gauge link fields, and the extra eikonal line behaves exactly as in one loop level. 

\end{document}