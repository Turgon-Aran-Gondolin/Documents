\documentclass{article}
\usepackage{fullpage}
\usepackage{parskip}
\usepackage{physics}
\usepackage{amsmath}
\usepackage{amssymb}
\usepackage{xcolor}
\usepackage[colorlinks,urlcolor=blue]{hyperref}
\usepackage{array}
\usepackage{longtable}
\usepackage{multirow}
\usepackage{comment}
\usepackage{graphicx}
\usepackage{cite}
\newcolumntype{M}{>{$\displaystyle}c<{$}}
\newcolumntype{L}{>{$\displaystyle}l<{$}}

\newcommand\Vtextvisiblespace[1][.3em]
{%
	\mbox{\kern.06em\vrule height.3ex}%
	\vbox{\hrule width#1}%
	\hbox{\vrule height.3ex}
}

\newcommand{\cbox}[2][cyan]
{\mathchoice
	{\setlength{\fboxsep}{0pt}\colorbox{#1}{$\displaystyle#2$}}
	{\setlength{\fboxsep}{0pt}\colorbox{#1}{$\textstyle#2$}}
	{\setlength{\fboxsep}{0pt}\colorbox{#1}{$\scriptstyle#2$}}
	{\setlength{\fboxsep}{0pt}\colorbox{#1}{$\scriptscriptstyle#2$}}
}

\newcommand{\typical}{\cbox{\phantom{A}}}
\newcommand{\tall}{\cbox{\phantom{A^{\vphantom{x^x}}_x}}}
\newcommand{\grande}{\cbox{\phantom{\frac{1}{xx}}}}
\newcommand{\venti}{\cbox{\phantom{\sum_x^x}}}

\begin{document}
The Newton's second law is F=ma.

The Newton's second law is $F=ma$.

The Newton's second law is
$$F=ma$$

The Newton's second law is
\[F=ma\]

Greek Letters $\eta$ and $\mu$

Fraction $\frac{a}{b}$

Power $a^b$

Subscript $a_b$

Derivate $\frac{\partial y}{\partial t} $

Vector $\vec{n}$

Bold $\mathbf{n}$



$\Delta E_{\text{fine structure}}=\frac{(Z\alpha)^2}{n}\left(\frac{1}{j+\frac{1}{2}}-\frac{3}{4n}\right)E_n.$


%\begin{equation}\label{}
%  \Delta E_{\text{fine structure}}=\frac{(Z\alpha)^2}{n}\left(\frac{1}{j+\frac{1}{2}}-\frac{3}{4n}\right)E_n.
%\end{equation}
To time differential $\dot{F}$

Matrix (lcr here means left, center or right for each column)
\[
\left[
\begin{array}{lcr}
a1&b22& c333 \\
d444 & e555555 & f6
\end{array}
\right]
\]

Equations(here \& is the symbol for aligning different rows)
\begin{align}
a+b&=c\\
d&=e+f+g
\end{align}

\[
\left\{
\begin{aligned}
&a+b=c\\
&d=e+f+g
\end{aligned}
\right.\]

$$
i\hbar \pdv{t}(\grande\ket{\alpha,t_0;t}_S)=(H_0+V)\ket{\alpha,t_0;t}_S
$$
%\cite{Epelbaum:2012nz}
\begin{thebibliography}{1}

\bibitem{Epelbaum:2012nz}
  E.~Epelbaum,
  %``Nuclear forces: Theory and applications,''
  PoS QNP {\bf 2012}, 004 (2012).
  %%CITATION = POSCI,QNP2012,004;%%

%\cite{Dyhdalo:2016ygz}
\bibitem{Dyhdalo:2016ygz}
  A.~Dyhdalo, R.~J.~Furnstahl, K.~Hebeler and I.~Tews,
  %``Regulator Artifacts in Uniform Matter for Chiral Interactions,''
  arXiv:1602.08038 [nucl-th].
  %%CITATION = ARXIV:1602.08038;%%
  
\bibitem{Kuo:2015lea}
  T.~T.~S.~Kuo, J.~W.~Holt and E.~Osnes,
  %``Introduction to low-momentum effective interactions with Brown-Rho scaling and three-nucleon forces,''
  Phys.\ Scripta {\bf 91} (2016) 3,  033009
  doi:10.1088/0031-8949/91/3/033009
  [arXiv:1510.04432 [nucl-th]].
  %%CITATION = doi:10.1088/0031-8949/91/3/033009;%%
  %1 citations counted in INSPIRE as of 08 Mar 2016
\end{thebibliography}

\end{document} 