%!mode::"Tex:UTF-8"
\documentclass{article}
\usepackage{fullpage}
\usepackage{parskip}
\usepackage{physics}
\usepackage{amsmath}
\usepackage{amssymb}
\usepackage{xcolor}
\usepackage[colorlinks,urlcolor=blue]{hyperref}
\usepackage{array}
\usepackage{longtable}
\usepackage{multirow}
\usepackage{comment}
\usepackage{graphicx}
\usepackage{cite}

\newcolumntype{M}{>{$\displaystyle}c<{$}}
\newcolumntype{L}{>{$\displaystyle}l<{$}}

\newcommand\Vtextvisiblespace[1][.3em]
{%
	\mbox{\kern.06em\vrule height.3ex}%
	\vbox{\hrule width#1}%
	\hbox{\vrule height.3ex}
}

\newcommand{\cbox}[2][cyan]
{\mathchoice
	{\setlength{\fboxsep}{0pt}\colorbox{#1}{$\displaystyle#2$}}
	{\setlength{\fboxsep}{0pt}\colorbox{#1}{$\textstyle#2$}}
	{\setlength{\fboxsep}{0pt}\colorbox{#1}{$\scriptstyle#2$}}
	{\setlength{\fboxsep}{0pt}\colorbox{#1}{$\scriptscriptstyle#2$}}
}

\newcommand{\typical}{\cbox{\phantom{A}}}
\newcommand{\tall}{\cbox{\phantom{A^{\vphantom{x^x}}_x}}}
\newcommand{\grande}{\cbox{\phantom{\frac{1}{xx}}}}
\newcommand{\venti}{\cbox{\phantom{\sum_x^x}}}
%\linespread{1.2}
%\geometry{left=2.5cm,right=2.5cm,top=2.5cm,bottom=2.5cm}
\author{hys}
\title{Notes:Scattering Theory}
\date{}
\begin{document}
\maketitle
\paragraph{}
In the previous study,scattering theory was not been introduced in detail.The textbook just briefly shows that the whole wavefunction,the formal solution of the Schrodinger equation can be described as following:
\begin{equation}
\psi\pqty{r,\theta}\thickapprox A\Bqty{e^{ikz}+f\pqty{\theta}\frac{e^{ikr}}{r}},
\end{equation}
and thus led to the scattering amplitude $f(\theta)$.In the interest of understanding the Lippmann-Schwinger equation,I choose to start from the very beginning of scattering theory.

\section{Scattering as a time-dependent perturbation}

\subsection{}
\paragraph{}
Given the Hamiltonian $H$'s form:
$$
H=H_0+V(\vb*{r}),
$$
where    $$H_0=\frac{\vb*{p^2}}{2m},$$
if we consider $V(\vb*{r})$ as a perturbation,and we already know that $H_0$'s eigenvectors is the plane-wave solution,set as $\ket{\vb*{k}}$,the eigenvalue can be easily written as $$E_{\vb*{k}}=\frac{\hbar^2\vb*{k}^2}{2m}.$$
\paragraph{}
Assuming $V\pqty{\vb*{r}}$ is time-independent(from the incoming particle's view,the potential of the scatterer $V(\vb*{r})$ is only visible during the short period of the interaction,so it's logical to consider $V(\vb*{r})$ as a time-independent value),using the time-dependent perturbation theory,the transition amplitude therefore is:
\begin{equation}\label{}
  \bra{n}U_I(t,t0)\ket{i}.
\end{equation}

\end{document}
