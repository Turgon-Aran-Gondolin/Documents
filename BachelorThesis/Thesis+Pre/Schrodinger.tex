%!mode::"Tex:UTF-8"
\PassOptionsToPackage{unicode}{hyperref}
\PassOptionsToPackage{naturalnames}{hyperref}
\documentclass[hyperref,cs4size,titlepage,twoside]{ctexart}
%\usepackage{fullpage}
\usepackage{geometry}
\usepackage{parskip}
\usepackage{physics}
\usepackage{amsmath}
\usepackage{amssymb}
\usepackage{xcolor}
%\usepackage[colorlinks,urlcolor=red,citecolor=green,anchorcolor=blue]{hyperref}
\usepackage{array}
\usepackage{longtable}
\usepackage{multirow}
\usepackage{comment}
\usepackage{graphicx}
\usepackage{cite}
\usepackage{slashbox}
%\usepackage{intent}
\usepackage{fancyhdr}
\geometry{left=3cm,right=2.5cm,top=3cm,bottom=2.5cm}
\pagestyle{fancy}
\fancyhead[L]{}
\fancyhead[CE]{\zihao{-5}薛定谔方程的重整化和有效理论}
\fancyhead[CO]{\zihao{-5}\leftmark}
\fancyhead[R]{}
\fancyfoot[LE]{-\thepage-}
\fancyfoot[RO]{-\thepage-}
\fancyfoot[C]{}
\renewcommand{\headrule}{%
\hrule width\headwidth height1.2pt \vspace{1pt}%
\hrule width\headwidth}
\hypersetup{colorlinks,linkcolor=blue,citecolor=green,CJKbookmarks=true}
\CTEXsetup[format+={\zihao{4}\songti}]{section}
\CTEXsetup[format+={\zihao{4}\songti}]{subsection}
\CTEXsetup[format+={\zihao{4}\songti}]{subsubsection}
\title{\zihao{3}薛定谔方程的重整化和有效理论}
\author{\zihao{-4}黄应生}
\date{}
\begin{document}
\maketitle
\pagestyle{empty}
\cleardoublepage
\pagenumbering{roman}
\songti\tableofcontents
\cleardoublepage
\section*{摘要}
\cleardoublepage
\section*{Abstract}
\cleardoublepage
\pagenumbering{arabic}
\pagestyle{fancy}
\section*{引言}
\addcontentsline{toc}{section}{引言}
重整化和有效理论是
\clearpage
\section{有效势的建立}
\clearpage

\clearpage

\clearpage
\section*{结论}
\addcontentsline{toc}{section}{结论}

\clearpage
\section*{致谢}
\addcontentsline{toc}{section}{致谢}

\clearpage
\begin{thebibliography}{b}

\bibitem{Lepage}
Lepage P. How to renormalize the Schrodinger equation[J]. arXiv preprint nucl-th/9706029, 1997.
\bibitem{Hill:2000yj}
Hill R. Nonperturbative techniques for QED bound states[J]. arXiv preprint hep-ph/0008002, 2000.
\bibitem{sakurai}
Sakurai J J, Tuan S F, Commins E D. Modern Quantum Mechanics, Second Edition. Pearson Schweiz Ag, 2013.
\bibitem{Jinyan}
曾谨言. 量子力学(卷I)(第5版)(现代物理学丛书)(精)[M]. 科学, 2013.
\bibitem{Griffiths}
Griffiths, David Jeffery. Introduction to quantum mechanics. Pearson Education India, 2013.
\bibitem{Chen}
陈鄂生. 量子力学基础教程(第5版)[M]. 山东大学出版社, 2013.
\bibitem{Korsch}
Korsch H J, Glck M. Computing quantum eigenvalues made easy[J]. European Journal of Physics, 2002, 23(23):413-426.
\bibitem{Landau}
朗道, 栗弗席兹. 量子力学:非相对论理论[M]. 高等教育出版社, 2008.
\bibitem{Weinberg}
Weinberg S. Lectures on Quantum Mechanics[J]. Lectures on Quantum Mechanics by Steven Weinberg Cambridge Uk Cambridge University Press, 2012, 83(9):xiv,528.
\bibitem{MMA}
徐安农. 科学计算引论: 基于 Mathematica 的数值分析[M]. 机械工业出版社, 2010.
\bibitem{dong1}
董键. Mathematica与大学物理计算 第2版[M]. 清华大学出版社, 2013.
\bibitem{nonlocal}
Kidun O, Fominykh N, Berakdar J. Scattering and bound-state problems with non-local potentials: application of the variable-phase approach[J]. Journal of Physics A: Mathematical and General, 2002, 35(44): 9413.
\bibitem{mma3}
Lucha W, Schöberl F F. Solving the Schrödinger equation for bound states with Mathematica 3.0[J]. International Journal of Modern Physics C, 1999, 10(04): 607-619.

\end{thebibliography}
% \appendix
% % \renewcommand{\appendixname}{Appendix~\Alph{section}}
% \section{附录 1}
%  some text...
% \section{Some Examples 2}
%  some text...
\end{document} 