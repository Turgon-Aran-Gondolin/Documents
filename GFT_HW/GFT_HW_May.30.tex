\PassOptionsToPackage{unicode}{hyperref}
\PassOptionsToPackage{naturalnames}{hyperref}
\documentclass{article}
\usepackage{geometry}
%\usepackage{fullpage}
\usepackage{parskip}
\usepackage{physics}
\usepackage{amsmath}
\usepackage{amssymb}
\usepackage{xcolor}
\usepackage[colorlinks,linkcolor=blue,citecolor=green]{hyperref}
\usepackage{array}
\usepackage{longtable}
\usepackage{multirow}
\usepackage{comment}
\usepackage{graphicx}
\usepackage{cite}
\usepackage{amsfonts}
\usepackage{bm}
\usepackage{slashed}
\usepackage{dsfont}
\usepackage{mathtools}
\usepackage[compat=1.1.0]{tikz-feynman}
\usepackage{simplewick}
%\usepackage{fourier}
%\usepackage{slashbox}
%\usepackage{intent}
\usepackage{mathrsfs}
\usepackage{xparse}
\usepackage{enumerate}

\geometry{left=0.9cm,right=0.9cm,top=1.5cm,bottom=2cm}

\newcommand{\gm}{\gamma^{\mu}}
\newcommand{\gn}{\gamma^{\nu}}
\newcommand{\gs}{\gamma^{\sigma}}
\newcommand{\gr}{\gamma^{\rho}}
\newcommand{\gnr}{g^{\nu\rho}}
\newcommand{\gmr}{g^{\mu\rho}}
\newcommand{\gms}{g^{\mu\sigma}}
\newcommand{\gns}{g^{\nu\sigma}}
\newcommand{\vbp}{\vb{p}}
\newcommand{\vbk}{\vb{k}}
\newcommand{\g}{\gamma}
\renewcommand{\a}{\alpha}
\renewcommand{\b}{\beta}
\renewcommand{\t}{\theta}
\newcommand{\la}{\lambda}
\newcommand{\p}{\phi}
\newcommand{\vp}{\varphi}
\newcommand{\s}{\sigma}
\renewcommand{\G}{\Gamma}
\newcommand{\pars}{\slashed\partial}
\newcommand{\ps}{\slashed p}
\newcommand{\ks}{\slashed k}
\newcommand{\lag}{\mathcal{L}}
\newcommand{\da}{^{\dagger}}
\newcommand{\sm}{^{\mu}}
\newcommand{\sn}{^{\nu}}
\newcommand{\smn}{^{\mu\nu}}
\newcommand{\Dm}{D^{\mu}}
\newcommand{\dm}{\partial^{\mu}}
\newcommand{\Asquare}{A^{\mu}A_{\mu}}
\newcommand{\partialsquare}[2]{\partial^{\mu}{#1}\partial_{\mu}{#2}}

\title{Homework: Gauge Field Theory \#3}
\author{Yingsheng Huang}
\begin{document}
\maketitle
\begin{enumerate}[{\bf 1.}]
  \item The standard model Lagrangian without fermion part:
	$$\lag=-\frac{1}{2}\Tr G^{\mu\nu}G_{\mu\nu}-\frac{1}{2}\Tr W^{\mu\nu}W_{\mu\nu}-\frac{1}{4}B^{\mu\nu}B_{\mu\nu}+\abs{D^{\mu}\phi}^2-V(\phi)$$
	  note that $G$ has 3 colors and $W$ has 3 flavours, the covariant derivatives are
	  $$D^{\mu}=\partial^{\mu}-igW^{a,\mu}T^a+ig_BB^{\mu}$$
		and the gauge transforms are:

    SO(3):
    $$G^{a,\mu}\rightarrow G^{a,\mu}+\frac{1}{g_G}\partial^{\mu}\b^a-f^{abc}\b^bG^{c,\mu}$$
		U(1):
		$$B^{\mu}\rightarrow B^{\mu}-\frac{1}{g_B}\partial^{\mu}\b$$
		$$\phi\rightarrow e^{i\b(x)}\phi$$
		SU(2):
		$$W^{a,\mu}\rightarrow W^{a,\mu}+\frac{1}{g}\partial^{\mu}\a^a-f^{abc}\a^bW^{c,\mu}$$
		$$\phi\rightarrow e^{i\a^a(x)T^a}\phi$$
		Make $\phi=\frac{1}{\sqrt{2}}(v+h(x))$
		$$\delta h=i\b(x)(v+h)(U(1))$$
    $$\delta h=i\a^a(x)T^a(v+h)(SU(2))$$
	  With $R_{\xi}$ gauge, the gauge fixing term is
    $$\lag_{GF,gluon}=-\frac{1}{2\xi}(\partial_{\mu}G^{a,\mu})^2$$
	  $$\lag_{GF}=-\frac{1}{2\xi}(\partial_{\mu}W^{a,\mu}-\xi gT^a_{ij}v_jh_i)^2-\frac{1}{2\xi}(\partial_{\mu}B^{a,\mu}-\xi g_Bv_ih_i)^2$$
    Then for gluon, the FP determinant is
    $$\frac{\delta \partial_{\mu}G^{a,\mu}}{\delta \b^b}=\frac{1}{g_G}\partial_{\mu}\partial^{\mu}\delta^{ab}-f^{abc}\partial_{\mu}G^{c,\mu}$$
    so the ghost field part is
    $$\lag_{FP}=\bar c_G^a(\partial^2\delta^{ab}-g_Gf^{abc}\partial_{\mu}G^{c,\mu})c_G^b$$
    For the electro-weak part, the determinant is
    $$\fdv{(\partial_{\mu}W^{a,\mu}-\xi gT^a_{ij}v_jh_i)}{\a^b}=\frac{1}{g}\partial^2\delta^{ab}-f^{abc}\partial_{\mu}W^{c,\mu}-i\xi gT^a_{ij}v_jT^b_{ik}(v+h)_k$$
    $$\fdv{(\partial_{\mu}B^{a,\mu}-\xi g_Bv_ih_i)}{\b}=-\frac{1}{g_B}\partial^2-i\xi g_Bv_i(v+h)_i$$
    The ghost fields are
    $$\lag_{FP,W}=\bar c_W^a(\partial^2\delta^{ab}-gf^{abc}\partial_{\mu}W^{c,\mu}-i\xi gT^a_{ij}v_jT^b_{ik}(v+h)_k)c_W^b$$
    $$\lag_{FP,B}=\bar c_B(\partial^2-i\xi g_B^2v_i(v+h)_i)c_B$$
    It should be easy to introduce Weinberg angle. (For real SM Higgs is a doublet, and for $W^{\pm}$ the higgs part should be real scalar fields $phi^+$ and $\phi^-$ without breaking and for $Z$ and $A$ the higgs part should be complex scalar field.)


	\item QED Lagrangian:
	$$\lag=-\frac{1}{4}F_{\mu\nu}F^{\mu\nu}+\bar\psi(i\slashed D+m)\psi$$
	We can add gauge fixing term
	$$\lag_{GF}=-\frac{1}{2\xi}(\partial_{\mu}A^{\mu})^2$$
	and ignore the fermion part, the generating functional is then
	$$Z[J]=\int D[A]e^{i\int\dd^4x(-\frac{1}{4}F_{\mu\nu}F^{\mu\nu}-\frac{1}{2\xi}(\partial_{\mu}A^{\mu})^2-J^{\mu}A_{\mu})}$$
	Note that the kinetic term can be rewrite as
	$$-\frac{1}{4}F_{\mu\nu}F^{\mu\nu}=-\frac{1}{2}(\partial_{\mu}A^{\nu})^2+\frac{1}{2}(\partial_{\mu}A^{\mu})^2$$
	so
	\begin{align*}
		Z[J]&=\int D[A]e^{i\int\dd^4x(-\frac{1}{2}(\partial_{\mu}A^{\nu})^2+\frac{1-\xi^{-1}}{2}(\partial_{\mu}A^{\mu})^2-J^{\mu}A_{\mu})}\\
		&=\int D[A]e^{i\int\dd^4x(A_{\mu}(\frac{1}{2}g^{\mu\nu}\partial^2+\frac{1-\xi^{-1}}{2}\partial^{\mu}\partial^{\nu})A_{\nu}-J^{\mu}A_{\mu})}
	\end{align*}
	The propagator is then
	$$(g^{\mu\nu}\partial^2-(1-\xi^{-1})\partial^{\mu}\partial^{\nu})\Delta_{\mu\nu}(x-y)=i\delta^4(x-y)$$
		$$\Delta^{\mu\nu}(x-y)=\int\frac{\dd^4k}{(2\pi)^4}(-i)(\frac{g^{\mu\nu}}{k^2+i\epsilon}-\frac{(1-\xi)k^{\mu}k^{\nu}/k^2}{k^2+i\epsilon})e^{ik\cdot(x-y)}$$
	\item BRST symmetry.

	We have
	\begin{align*}
		\delta_B\psi=-igc^aT^a\psi,\delta_B\bar\psi=\bar\psi(-igc^aT^a)\\
		\delta_BG^{a,\mu}=(D^{\mu})^{ab}c^b,\delta_Bc^a=\frac{1}{2}gf^{abc}c^bc^c\\
		\delta_B\bar c^a=B^a(x),\delta_BB^a=0i,(D^{\mu})^{ab}=\partial^{\mu}\delta^{ab}+gf^{cab}G^{c,\mu}
	\end{align*}
	so ($T^aT^b=if^{abc}T^c+T^bT^a=if^{abc}T^c+\frac{1}{2}\delta^{ab}-T^aT^b=\frac{i}{2}f^{abc}T^c$)
	\begin{align*}
		&\delta_B(\delta_B\psi)=-ig(\delta_Bc^a)T^a\psi+g^2c^aT^ac^bT^b\psi=-\frac{ig^2}{2}f^{abc}c^bc^cT^a\psi+g^2c^aT^ac^bT^b\psi=0\\
		&\delta_B^2c^a=\frac{1}{2}gf^{abc}(\frac{1}{2}gf^{bde}c^dc^ec^c-\frac{1}{2}gf^{cde}c^{b}c^{d}c^{e})=\frac{g^2}{4}(f^{eac}f^{cbd}c^bc^dc^e-f^{abc}f^{cde}c^bc^dc^e)=-\frac{g^2}{4}f^{adc}f^{ceb}c^bc^dc^e=0\\
		&\delta_B^2\bar c^a=0\\
		&\delta_B^2\bar\psi=\bar\psi(-igc^aT^a)(-igc^bT^b)-\bar\psi(-ig\frac{1}{2}gf^{abc}c^bc^cT^a)=\bar\psi\frac{ig^2}{2}f^{abc}c^bc^cT^a-\bar\psi g^2c^aT^ac^bT^b=0\\
		&\delta_B^2G^{a,\mu}=\delta_B(\partial^{\mu}c^a+gf^{cab}G^{c,\mu}c^b)=\frac{1}{2}gf^{abc}\partial^{\mu}(c^bc^c)+gf^{cab}(\partial^{\mu}c^c+gf^{dce}G^{d,\mu}c^e)c^b+gf^{cab}G^{c,\mu}\frac{1}{2}gf^{bde}c^dc^e=\frac{1}{2}gf^{abc}\partial^{\mu}(c^bc^c)\\&-gf^{abc}c^b(\partial^{\mu}c^c)+g^2f^{cab}f^{dce}G^{d,\mu}c^ec^b-\frac{g^2}{2}f^{cab}f^{bde}G^{c,\mu}c^dc^e=g^2f^{cab}f^{dce}G^{d,\mu}c^ec^b+\frac{g^2}{2}f^{cab}f^{bde}G^{c,\mu}c^dc^e\\&=\frac{g^2}{2}(f^{aeb}f^{bdc}-f^{adb}f^{bec}+f^{cab}f^{bde})G^{c,\mu}c^dc^e=0
	\end{align*}

\end{enumerate}




\end{document}
