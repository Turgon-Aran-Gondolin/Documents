\PassOptionsToPackage{unicode}{hyperref}
\PassOptionsToPackage{naturalnames}{hyperref}
\documentclass{article}
\usepackage{geometry}
%\usepackage{fullpage}
\usepackage{parskip}
\usepackage{physics}
\usepackage{amsmath}
\usepackage{amssymb}
\usepackage{xcolor}
\usepackage[colorlinks,linkcolor=blue,citecolor=green]{hyperref}
\usepackage{array}
\usepackage{longtable}
\usepackage{multirow}
\usepackage{comment}
\usepackage{graphicx}
\usepackage{cite}
\usepackage{amsfonts}
\usepackage{bm}
\usepackage{slashed}
\usepackage{dsfont}
\usepackage{mathtools}
\usepackage[compat=1.1.0]{tikz-feynman}
\usepackage{simplewick}
%\usepackage{fourier}
%\usepackage{slashbox}
%\usepackage{intent}
\usepackage{mathrsfs}
\usepackage{xparse}
\usepackage{enumerate}

\geometry{left=0.9cm,right=0.9cm,top=1.5cm,bottom=2cm}

\newcommand{\gm}{\gamma^{\mu}}
\newcommand{\gn}{\gamma^{\nu}}
\newcommand{\gs}{\gamma^{\sigma}}
\newcommand{\gr}{\gamma^{\rho}}
\newcommand{\gnr}{g^{\nu\rho}}
\newcommand{\gmr}{g^{\mu\rho}}
\newcommand{\gms}{g^{\mu\sigma}}
\newcommand{\gns}{g^{\nu\sigma}}
\newcommand{\vbp}{\vb{p}}
\newcommand{\vbk}{\vb{k}}
\newcommand{\g}{\gamma}
\renewcommand{\a}{\alpha}
\renewcommand{\b}{\beta}
\renewcommand{\t}{\theta}
\newcommand{\la}{\lambda}
\newcommand{\p}{\phi}
\newcommand{\vp}{\varphi}
\newcommand{\s}{\sigma}
\renewcommand{\G}{\Gamma}
\newcommand{\pars}{\slashed\partial}
\newcommand{\ps}{\slashed p}
\newcommand{\ks}{\slashed k}
\newcommand{\lag}{\mathcal{L}}
\newcommand{\da}{^{\dagger}}
\newcommand{\sm}{^{\mu}}
\newcommand{\sn}{^{\nu}}
\newcommand{\smn}{^{\mu\nu}}
\newcommand{\Dm}{D^{\mu}}
\newcommand{\dm}{\partial^{\mu}}
\newcommand{\Asquare}{A^{\mu}A_{\mu}}
\newcommand{\partialsquare}[2]{\partial^{\mu}{#1}\partial_{\mu}{#2}}

\title{Homework: Gauge Field Theory}
\author{Yingsheng Huang}
\begin{document}
\maketitle
\begin{enumerate}[\bf 1.]
  \item Lagrangian
	$$\lag=\frac{1}{2}(\partial_{\mu}\phi)^2-V(\phi)$$
	$$V(\phi)=-\frac{1}{2}\mu^2\phi^2+\frac{\la^2}{4}\phi^4$$
	which satisfies
	$$\phi\rightarrow-\phi$$
	For such symmetry to break, we perform the following presedure:

	First, the minimum of $V(\phi)$ can be found in $\phi=\pm \frac{\mu^2}{\la^2}$, and we can define $v^2=\abs{\mel{0}{\phi}{0}}^2=\frac{\mu^2}{\la}$, which yields the broken symmetry of vacuum.

	By redefining the field $\phi(x)=\rho(x)+v$ such that $\rho(x)$ has the right vacuum, the Lagrangian is now
	$$\lag=\frac{1}{2}(\partial_{\mu}\rho)^2-\mu^2\rho^2-\la^2\rho^3v-\frac{\la^2}{4}\rho^4+\frac{\mu^4}{4\la^2}$$
	and we can see that there is no massless Goldstone particle. That's because that although the symmetry $\phi\rightarrow-\phi$ has broken, but it's discrete symmetry, therefore can't produce Goldstone particles.
  \item {\bf{$\mathbf{R_{\xi}}$ Gauge.}} The Lagrangian is
	$$\lag(\phi, A^{\mu})=(D^{\mu}\phi)^{\dagger}(D_{\mu}\phi)+\mu^2\phi^{\dagger}\phi-\la(\phi\da\phi)^2-\frac{1}{4}F^{\mu\nu}F_{\mu\nu} $$
	where $D^{\mu}=\partial^{\mu}+igA\sm$, $F\smn=\dm A\sn-\partial\sn A\sm$. Also we have $\abs{\mel{0}{\phi}{0}}=v$, $v^2=\frac{\mu^2}{\la}$.

	$R_{\xi}$ gauge
	$$\lag\rightarrow\lag-\frac{1}{2\xi}(\partial\sm A_{\mu}-\xi gvb)^2$$
	Choose $\phi$ to be $\phi=\frac{1}{\sqrt{2}}(v+h(x)+i b(x))$,
	$$D^{\mu}\phi=\frac{1}{\sqrt{2}}[\partial\sm h+i\partial\sm b+igA\sm (v+h)-gbA\sm]=\frac{1}{\sqrt{2}}[(\partial\sm h-gbA\sm)+i(\partial\sm b+g(v+h)A\sm)]   $$
	so the kinetic term
	$$(D\sm\phi)\da(D_{\mu}\phi)=\frac{1}{2}[(\partial\sm h-gbA\sm)^2+(\dm b+g(v+h)A\sm)^2]  $$
	this gives
	\begin{align*}
	  (D\sm\phi)\da(D_{\mu}\phi)&=\frac{1}{2}\partial\sm h\partial_{\mu}h-gb\partial\sm hA_{\mu}+\frac{1}{2}g^2b^2A^{\mu}A_{\mu}+\frac{1}{2}\partial^{\mu}b\partial_{\mu}b+g(v+h)\partial\sm bA_{\mu}+\frac{1}{2}g^2(v+h)^2A^{\mu}A_{\mu}  \\
	  &=\frac{1}{2}\partial\sm h\partial_{\mu}h+\frac{1}{2}\partial^{\mu}b\partial_{\mu}b+\frac{1}{2}g^2v^2\Asquare+gv\partial\sm bA_{\mu}+g^2vhA^{\mu}A_{\mu}+\frac{1}{2}g^2(b^2+h^2)\Asquare+g(h\partial\sm b-b\partial\sm h)A_{\mu}
	\end{align*}
	now we got the kinetic terms of scalar fields $h(x)$ and $b(x)$, mass term for gauge field $A^{\mu}$, crossing term of $b$ and $A^{\mu}$, and some interacting terms in the end.

	The mass term of original scala field gives
	$$\mu^2\phi\da\phi=\frac{1}{2}\mu^2(v+h)^2-\frac{1}{2}\mu^2b^2$$
	so the rest part of scalar field is
	$$-\frac{b^4 \lambda }{4}-\frac{1}{2} b^2 h^2 \lambda -b^2 h \lambda  v-\frac{h^4 \lambda }{4}-h^3 \lambda  v-h^2 \mu ^2+\frac{\mu ^4}{4 \lambda }$$

	Now the gauge fixing term is
	\begin{align*}
	  -\frac{1}{2\xi}\partial\sm A_{\mu}\partial^{\nu}A_{\nu}+gvb\partial_{\mu}A^{\mu}-\frac{\xi g^2v^2}{2}b^2
	\end{align*}
	we know that $F\smn F_{\mu\nu}$ can always be written in two terms, so
	$$-\frac{1}{4}F\smn F_{\mu\nu}=-\frac{1}{2}(\partial^{\mu}A_{\nu})^2+\frac{1}{2}(1-\xi^{-1})(\partial\sm A_{\mu})^2+gvb\partial^{\mu}A_{\mu}-\frac{\xi g^2v^2}{2}b^2$$
	and $$gvb\partial^{\mu}A_{\mu}=-gvA_{\mu}\partial^{\mu}b$$
	the crossing term is cancelled. The last term also gives $b$ field mass $\frac{\xi g^2v^2}{2}$.

	The Lagrangian is now
	\begin{align*}
	  \lag&=\frac{1}{2}\partial\sm h\partial_{\mu}h+\frac{1}{2}\partial^{\mu}b\partial_{\mu}b-\frac{1}{2}(\partial^{\mu}A_{\nu})^2+\frac{1}{2}(1-\xi^{-1})(\partial\sm A_{\mu})^2+\frac{1}{2}g^2v^2\Asquare-\mu^2h^2-\frac{\xi g^2v^2}{2}b^2(+\frac{\mu^4}{4\la})\\&+g^2vh\Asquare+\frac{1}{2}g^2(b^2+h^2)\Asquare+g(h\partial\sm b-b\partial\sm h)A_{\mu} -\frac{b^4 \lambda }{4} -b^2 h \lambda  v-\frac{h^4 \lambda }{4}-h^3 \lambda  v
	\end{align*}
  Then we have some standard 3 and 4 particle vertexs. Now we just need to deal with the propagators and the vertex with derivative.

  The propagators of both scalar fields are trival, with $m_h=\sqrt{2}\mu$, $m_b=\sqrt{\xi}gv$. The propagator of the vector field is, however, a bit more complicated.  $$\Delta_A^{\mu\nu}(x-y)=\frac{g^{\mu\nu}-\frac{k^{\mu}k^{\nu}}{k^2}}{k^2-m^2+i\epsilon}+\frac{\xi\frac{k^{\mu}k^{\nu}}{k^2}}{k^2-\xi m^2+i\epsilon}$$
  where the mass of vector field $m=gv$.

  Now we'll show how to derive the propagator: Define $\lag_0$
  $$\lag_0=-\frac{1}{2}\partial_{\mu}A^{\nu}\partial^{\mu}A_{\nu}+\frac{1}{2}(1-\xi^{-1})\partial^{\nu}A_{\mu}\partial^{\mu}A_{\nu}+\frac{1}{2}m^2A^{\nu}A_{\nu}$$
  and $$S_0=\int \dd^4x \lag_0$$
  Transform to momentum space
  $$S_0=-\frac{1}{2}\int\frac{\dd^4 k}{(2\pi)^4}\Bqty{\tilde A_{\mu}(k)\pqty{g^{\mu\nu}k^2-(1-\xi^{-1})k^{\mu}k^{\nu}-m^2g^{\mu\nu}}\tilde A_{\nu}(-k)-\tilde J^{\mu}(k)\tilde A_{\mu}(-k)-\tilde J^{\mu}(-k)\tilde A_{\mu}(k)}$$
  Define $\tilde D^{\mu\nu}(k)=g^{\mu\nu}k^2-(1-\xi^{-1})k^{\mu}k^{\nu}-m^2g^{\mu\nu}$
  \begin{align*}
    \tilde D^{\mu\nu}(k)&=g^{\mu\nu}k^2-(1-\xi^{-1})k^{\mu}k^{\nu}-m^2g^{\mu\nu}\\
    &=(k^2-m^2)g^{\mu\nu}-(1-\xi^{-1})k^{\mu}k^{\nu}\\
    &=(k^2-m^2)(g^{\mu\nu}-\frac{k^{\mu}k^{\nu}}{k^2})+(k^2-m^2)\frac{k^{\mu}k^{\nu}}{k^2}-(1-\xi^{-1})k\sm k\sn\\
    &=(k^2-m^2)(g^{\mu\nu}-\frac{k^{\mu}k^{\nu}}{k^2})+\xi^{-1}(k^2-\xi m^2)\frac{k\sm k\sn}{k^2}
  \end{align*}
  then to have the result
  $$S_0=-\frac{1}{2}\int\frac{\dd^4k}{(2\pi)^4}\tilde J_{\mu}(k)\tilde \Delta_F^{\mu\nu}(k)\tilde J_{\nu}(-k)$$
  we must have
  $$\tilde D_{\mu\nu}\tilde \Delta_F^{\nu\rho}=\delta_{\mu}^{\rho}$$
  that is
  \begin{align*}
    &\tilde D_{\mu\nu}(k)\tilde \Delta_F^{\nu\rho}(k)=\delta_{\mu}^{\rho}\\
    =&\Bqty{(k^2-m^2)(g_{\mu\nu}-\frac{k_{\mu}k_{\nu}}{k^2})+\xi^{-1}(k^2-\xi m^2)\frac{k_{\mu} k_{\nu}}{k^2}}\Bqty{Ag^{\nu\rho}+Bk^{\nu}k^{\rho}}\\
    =&A(k^2-m^2)\delta^{\rho}_{\mu}-A(k^2-m^2)\frac{k_{\mu}k^{\rho}}{k^2}+\xi^{-1}(k^2-\xi m^2)Ak_{\mu}k^{\rho}+\xi^{-1}(k^2-\xi m^2)Bk_{\mu}k^{\rho}
  \end{align*}
  such that $A=\frac{1}{k^2-m^2+i\epsilon}$ and $B=\frac{\xi}{(k^2-\xi m^2+i\epsilon)k^2}-\frac{1}{k^2(k^2-m^2+i\epsilon)}$ (with the Feynman prescription). The propagator is now
  $$\tilde \Delta_F^{\mu\nu}(k)=\frac{g\smn-\frac{k\sm k\sn}{k^2}}{k^2-m^2+i\epsilon}+\frac{\xi k\sm k\sn/k^2}{k^2-\xi m^2+i\epsilon}$$

%	Perform the gauge transformation
%	$$\phi\rightarrow e^{ig\theta(x)}\phi=\phi+ig\theta\phi$$
%	$$A^{\mu}\rightarrow A^{\mu}-\partial^{\mu}\theta$$
%	which gives
%	$$h\rightarrow e^{ig\theta(x)}h, b\rightarrow e^{ig\theta(x)}b$$
%	so the kinetic term becomes
%	\begin{align*}
%	  (\Dm\phi)\da(D_{\mu}\phi)&=g A(x) h(x) b'(x)+g v A(x) b'(x)+\frac{1}{2} g^2 A(x)^2 b(x)^2-g A(x) b(x) h'(x)+g^2 v A(x)^2 h(x)+\frac{1}{2} g^2 A(x)^2 h(x)^2\\&+\frac{1}{2} g^2 v^2 A(x)^2+\frac{1}{2} b'(x)^2+\frac{1}{2} h'(x)^2\\
%	  &=\frac{1}{2}b'^2+\frac{1}{2}h'^2+\frac{1}{2}g^2v^2A^2+gvb'A+g^2vhA+g(hb'-bh')A+\frac{1}{2}g^2(h^2+b^2)A^2
%	\end{align*}
%	here for the sack of simplicity, we write $\partial\sm f=f'(x)$. As what we hoped, this part is gauge invariant.
%
%	And we also have
%	\begin{align*}
%	  -\frac{1}{4}F^{\mu\nu}F_{\mu\nu}=-\frac{1}{2}(\partial^{\mu}A_{\nu})^2+\frac{1}{2}(1-\xi^{-1})(\partial\sm A_{\mu})^2-\frac{\xi^{-1}}{2}[(\partial^2\theta)^2-2\partial^2\theta\partial^{\nu}A_{\nu}]+gvbe^{ig\theta}(\partial^{\mu}A_{\mu}-\partial^2\theta)-\frac{\xi g^2v^2}{2}b^2
%	\end{align*}

	\item $Z^0\rightarrow l\bar l$.

  Write down the Lagrangian
   $$\lag=\lag_W-\frac{1}{4}B^{\mu\nu}B_{\mu\nu}+\pmqty{\bar \nu_L,\bar e_L,\bar e_R}i\slashed \partial\pmqty{\nu_L\\e_L\\e_R}+\lag_N+\lag_W$$
   $$\lag_N=(\frac{gg'}{\sqrt{g^2+g'^2}}\bar e_L\gm e_L-\frac{gg'}{\sqrt{g^2+g'^2}}Y_R\bar e_R\gm e_R) A_{\mu}+(-\frac{g'^2}{\sqrt{g^2+g'^2}}Y_R\bar e_R\gm e_R+\frac{g'^2-g^2}{2\sqrt{g^2+g'^2}}\bar e_L\gm e_L)Z_{\mu}-\frac{\sqrt{g^2+g'^2}}{2}\bar \nu_L\gm\nu_L Z_{\mu}$$
   The interaction term of Z boson and leptons is
   $$\lag_Z= (-\frac{g'^2}{\sqrt{g^2+g'^2}}Y_R\bar e_R\gm e_R+\frac{g'^2-g^2}{2\sqrt{g^2+g'^2}}\bar e_L\gm e_L)Z_{\mu}$$
   Note that the outstate don't have any explicit handness, so we can rewrite it as
   $$\lag_Z=\bar e \gm(\a+\b\g^5) e Z_{\mu}$$
   where $\a=\frac{g'^2(1-2Y_R)-g^2}{4\sqrt{g^2+g'^2}}$, $\b=\frac{-g'^2(2Y_R+1)+g^2}{4\sqrt{g^2+g'^2}}$. Take $Y_R=-1$ and $\frac{gg'}{\sqrt{g^2+g'^2}}=e$, then
  $$\a=-\frac{e}{c_Ws_W}(\frac{1}{4}-s_W^2),\;\;\b=\frac{e}{4c_Ws_W}$$
  where $c_W=\cos{\theta_W}=\frac{g}{\sqrt{g^2+g'^2}}$, $s_W=\sin{\theta_W}=\frac{g'}{\sqrt{g^2+g'^2}}$,
  so that
  $$\lag_Z=\bar e\gm(\a+\b\g^5)eZ_{\mu}$$
  Now the Lagrangian in the full form is
  $$\lag_Z=-\frac{e}{c_Ws_W}\bar e\gm(\frac{1-\g^5}{4}-s_W^2)eZ_{\mu}$$
   And $m_Z=\frac{ev}{2s_Wc_W}$.

  Now the amplitude is
  \begin{align*}
    i\mathcal{M}&=\feynmandiagram[small,horizontal=a to b,baseline=(b.base)]{
     a[particle={\(Z^0\)}]--[boson,momentum=k]b--[fermion,momentum=\(p_1\)]f1[particle=\(l\)],
     b--[anti fermion,rmomentum=\(p_2\)]f2[particle=\(\bar l\)],
     },\\
     &=i\bar u^s\gm(\a+\b\g^5)v^r\epsilon_{\mu}^{\la}
  \end{align*}
  And do the spin \& polarization sum
  \begin{align*}
    \frac{1}{3}\sum_{r,s,\la}\abs{\mathcal{M}}^2&=\frac{1}{3}\sum_{r,s,\la}[\bar u\gm(\a+\b\g^5)v\epsilon_{\mu}][\bar v\gn(\a+\b\g^5)u\epsilon_{\nu}^*]\\
    &=\frac{1}{3}\sum_{r,s,\la}\epsilon_{\mu}\epsilon_{\nu}^*\tr{u\bar u\gm(\a+\b\g^5)v\bar v\gn(\a+\b\g^5)}\\
    &=\frac{1}{3}(-g_{\mu\nu}+\frac{k_{\mu}k_{\nu}}{m_Z^2})\tr{(\ps_1+m)\gm(\a+\b\g^5)(\ps_2-m)\gn(\a+\b\g^5)}
  \end{align*}
  Note that
  \begin{align*}
    (\a+\b\g^5)(\ps-m)\gn(\a+\b\g^5)&=[\a(\ps-m)-\b(\ps+m)\g^5]\gn(\a+\b\g^5)\\
    &=\a(\ps-m)\gn(\a+\b\g^5)+\b(\ps+m)\gn\g^5(\a+\b\g^5)\\
    &=\a(\ps-m)\gn(\a+\b\g^5)+\b(\ps+m)\gn(\b+\a\g^5)\\
    &=(\a^2+\b^2)\ps\gn+2\a\b\ps\gn\g^5-(\a^2-\b^2)m\gn\\
    &=(\a'\ps-\b'm)\gn+2\a\b\ps\gn\g^5
  \end{align*}
  where $\a'=\a^2+\b^2$, $\b'=\a^2-\b^2$.

  So the trace part becomes
  \begin{align*}
    \tr{(\ps_1+m)\gm[(\a'\ps_2-\b'm)\gn+2\a\b\ps_2\gn\g^5]}&=\tr{(\ps_1+m)\gm(\a'\ps_2-\b'm)\gn+2\a\b(\ps_1+m)\gm\ps_2\gn\g^5}
  \end{align*}
  \begin{align*}
    \tr{(\ps_1+m)\gm(\a'\ps_2-\b'm)\gn}&=4[\a'p_1^{\mu}p_2^{\nu}+\a'p_1^{\nu}p_2^{\mu}-g^{\mu\nu}(\a'p_1\cdot p_2+\b'm^2)]
  \end{align*}
  \begin{align*}
    \tr{(\ps_1+m)\gm\ps_2\gn\g^5}&=\tr{\ps_1\gm\ps_2\gn\g^5}\\
    &=-4i\epsilon^{\rho\mu\s\nu}p_1_{\rho}p_2_{\s}
  \end{align*}
  and the latter term will vanish (anti symmetry multiplies symmetry).
  \begin{align*}
    \frac{1}{3}\sum_{r,s,\la}\abs{\mathcal{M}}^2&=\frac{4}{3}(-g_{\mu\nu}+\frac{k_{\mu}k_{\nu}}{m_Z^2}) [\a'p_1^{\mu}p_2^{\nu}+\a'p_1^{\nu}p_2^{\mu}-g^{\mu\nu}(\a'p_1\cdot p_2+\b'm^2)]\\
    &=-\frac{4}{3} [2\a'p_1\cdot p_2-4(\a'p_1\cdot p_2+\b'm^2)]+\frac{4}{3} [\frac{2}{m_Z^2}\a'(p_1\cdot k)(p_2\cdot k)-(\a'p_1\cdot p_2+\b'm^2)]\\
    &=\frac{4}{3}[\frac{2}{m_Z^2}\a'(p_1\cdot k)(p_2\cdot k)-(\a'p_1\cdot p_2+\b'm^2)+2\a'p_1\cdot p_2+4\b'm^2]\\
    &=\frac{4}{3}[\frac{2}{m_Z^2}\a'(p_1\cdot k)(p_2\cdot k)+\a'p_1\cdot p_2+3\b'm^2]\\
    &=\frac{4}{3}[\frac{2}{m_Z^2}\a'(m^2+p_1\cdot p_2)^2+\a'p_1\cdot p_2+3\b'm^2]
  \end{align*}
  Knowing that in centre-of-mass frame
  $$p_1\cdot p_2=E_1^2+\vb{p}_1^2=2E_1^2-m^2,\;\;E_1=\frac{m_Z}{2}$$
  $$\frac{1}{3}\sum_{r,s,\la}\abs{\mathcal{M}}^2=\frac{4}{3}[\a'm_Z^2-(\a'-3\b')m^2]$$
  The decay width is
  \begin{align*}
    \G&=\frac{1}{2m_Z}\int\frac{\dd^3p_1}{(2\pi)^3}\frac{\dd^3p_2}{(2\pi)^3}\frac{\abs{\mathcal{M}}^2}{4E_1E_2}(2\pi)^4\delta^4(k-p_1-p_2)\\
    &=\frac{1}{2\pi m_Z}\int\dd\abs{\vb{p_1}} \abs{\vb{p_1}}^2\frac{\abs{\mathcal{M}}^2}{4E_1^2}\delta({m_Z-2E_1})\\
    &=\frac{1}{2\pi m_Z}\int\dd E_1 \abs{\vb{p_1}}\frac{\abs{\mathcal{M}}^2}{8E_1}\delta({m_Z-2E_1})\\
    &=\frac{\abs{\mathcal{M}}^2}{16\pi m_Z^2}\sqrt{m_Z^2-4m^2}\\
    &=\frac{\frac{4}{3}[\a'm_Z^2-(\a'-3\b')m^2]}{16\pi m_Z^2}\sqrt{m_Z^2-4m^2}\\
    &=\frac{\a'm_Z^2-(\a'-3\b')m^2}{12\pi m_Z^2}\sqrt{m_Z^2-4m^2}
  \end{align*}
  Use $m_Z=91.187GeV$, $s_W^2=0.231$, $e^2=\frac{4\pi}{128}$, we have $\G=84.032MeV$.
  % Since they're all three particle vertex and the only difference is the coupling constant (maybe the helicity as well), we only consider vertex $G\bar l_L\gm l_L Z_{\mu}$ for now.
%
  % The amplitude is as following
  % \begin{align*}
    % i\mathcal{M}&=
    % \feynmandiagram[small,horizontal=a to b,baseline=(b.base)]{
    % a[particle={\(Z^0\)}]--[boson,momentum=p]b--[fermion,momentum=\(p_1\)]f1[particle=\(l_L\)],
    % b--[anti fermion,rmomentum=\(p_2\)]f2[particle=\(\bar l_L\)],
    % },
    % \\
    % &=iG\bar v_L\gm u_L\epsilon_{\mu}\\
    % &=iG \bar v\gm\frac{1-\g^5}{2}u\epsilon_{\mu}
  % \end{align*}
  % If we don't take spin sum,
  % \begin{align*}
    % i\mathcal{M}&=iG \bar v\gm\frac{1-\g^5}{2}u\epsilon_{\mu}\\
    % &=
  % \end{align*}
  % And
  % \begin{align*}
    % \sum_{spins}\abs{M}^2&=\sum_{spins}G^2\epsilon_{\mu}\epsilon^*_{\nu}\tr{v\bar v\gm\frac{1-\g^5}{2}u \bar u\gn\frac{1-\g^5}{2}}\\
    % &=G^2\epsilon_{\mu}\epsilon_{\nu}^*\tr{(\ps_2-m)\gm\frac{1-\g^5}{2}(\ps_1+m)\gn\frac{1-\g^5}{2}}
  % \end{align*}
  % Note that
  % \begin{align*}
    % (1-\g^5)(\ps+m)\gn(1-\g^5)&=[(\ps+m)+(\ps-m)\g^5]\gn(1-\g^5)\\
    % &=(\ps+m)\gn(1-\g^5)+(\ps-m)\gn(1-\g^5)\\
    % &=2\ps\gn(1-\g^5)
  % \end{align*}
  % so the trace part becomes
  % $$\frac{1}{2}\tr{(\ps_2-m)\gm\ps_1\gn(1-\g^5)}$$
  % We know that
  % \begin{align*}
    % \tr{(\ps_2-m)\gm(\ps_1+m)\gn}=4[p_2^{\mu}p_1^{\nu}+p_2^{\nu}p_1^{\nu}-g^{\mu\nu}(p_1\cdot p_2+m^2)]
  % \end{align*}
  % and
  % \begin{align*}
    % \tr{(\ps_2-m)\gm\g^5(\ps_1+m)\gn\g^5}&=\tr{(\ps_2-m)\gm(\ps_1-m)\gn}\\

%
  % So we'll find the mass automatically vanishes, and the rest is
  % \begin{align*}
    % \sum_{spins}\abs{M}^2&=\frac{G^2}{2}\epsilon_{\mu}\epsilon_{\nu}^*\tr{\ps_2\gm\ps_1\gn(1-\g^5)}\\
    % &=2G^2\epsilon_{\mu}\epsilon_{\nu}^*(p_2^{\mu}p_1^{\nu}+p_2^{\nu}p_1^{\mu}-p_2\cdot p_1g^{\mu\nu}+i\epsilon^{\rho\mu\s\nu}p_2_{\rho}p_1_{\s})

\end{enumerate}




\end{document}
