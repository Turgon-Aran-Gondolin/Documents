\documentclass[letterpaper,11pt]{article}

\usepackage[in]{fullpage}
\usepackage[sort&compress,numbers]{natbib}
\usepackage{doi}%<----------
\usepackage{hyperref}

\title{Research Statement}
\author{Yingsheng Huang}
\date{}
\begin{document}
\maketitle

% \section*{Research Interest}
% My main reserch interest is applications of effective field theories (EFTs). A major part of those are nonrelativistic EFTs for both QCD and atomic systems. As an bridge to the nonperturbative properties of quarkonia, the nonrelativistic QCD (NRQCD) takes the advantage of heavy quark masses, and separates the 

\section*{Research Experience}
My main field of research has been focused on nonrelativistic effective field theories. It involves two fronts: nonrelativistic EFTs for atomic systems and nonrelativistic QCD (NRQCD) factorization. For atomic systems, we treat the atomic system with nonrelativistic EFT. We then examine the asymptotic behavior of the wave functions near origin with operator product expansion (OPE)~\cite{Huang2018,Huang2018a,Huang2019}. I also studied the NRQCD factorization of fully-heavy tetraquark production: for inclusive production at LHC, we factorized the fragmentation function of fully-heavy tetraquark production via the NRQCD factorization approach, derived the short-distance coefficients (SDCs) for it, and obtained phenomenological results with diquark and tetraquark wave functions at the origin~\cite{Feng2020}; for exclusive production at B factory, we factorized the cross section directly with NRQCD, and obtained both the polarized and unpolarized cross section~\cite{Feng2020a}. Apart from these EFT-related works, I also participated in the study on meson-meson scattering in the context of 't Hooft model, a model based on the large-$N_c$ limit of 1+1-dimensional QCD~\cite{Chen2019}.
\subsection*{Atomic wave functions near origin}
Our motivation originated from the fact that the wave functions of relativistic wave equations (i.e. Klein-Gordon equation or Dirac equation) diverge at the origin. Intuitively, as wave functions approaching the origin, it would probe the short distance behavior of the system that was never the intent of the corresponding wave equation. It is our goal to utilize nonrelativistic EFTs and OPE to study this near-origin asymptotic behavior of wave functions.

To describe the atom in a field-theoretical way, we use nonrelativistic QED (NRQED) to describe electrons, and nucleus with a heavy nucleus effective theory (HNET, similar to heavy quark effective theory) to describe the nucleus. The central idea is to employ OPE to separate the short-distance behavior where the divergence happens from the long-distance one. As we know, the atomic wave function, in terms of fields, is the matrix element of nonlocal electron fields and nucleus field, sandwiched by vacuum and the atom state. We take the nonlocal composite field operators of electrons and nucleus and expand it into products of SDCs and local operators with OPE. It comes as a surprise to us that if we ignore relativistic corrections (thus only taking leading order from NRQED), the SDC at order-$\alpha$ reproduces the renown Kato's cusp relation in atomic physics~\cite{Huang2018}. Considering the leading relativistic corrections, the SDC at order-$\alpha^2$ matches the logarithmic divergence of the wave function~\cite{Huang2018a,Huang2019}. Furthermore, we resumed all leading logarithms with the renormalization group equation, and the result agrees with what comes out of the wave function.

\subsection*{NRQCD factorization for fully-heavy tetraquark}
Very recently, a fully-charmed tetraquark candidate, dubbed X(6900),  was discovered at LHCb. As an advantage of fully-heavy tetraquarks, the effect of light quarks and gluons popping out of the vacuum is highly suppressed by the heavy quark masses. Therefore, it is reasonable to claim the leading Fock state to be only composed of four valence quarks. This allows us to proceed similarly as quarkonium.

For tetraquark inclusive production at LHC, the cross section can be factorized into the convolution of parton distribution functions (PDFs), partonic cross section and fragmentation function (FF). The leading contribution comes from the partonic process $gg\to gg$. Then, one of the produced gluons would fragment into a tetraquark via gluon FF. As we already have the PDFs and partonic cross section, we are left with the FF undetermined.
Similar to quarkonium production, we assume the FF can be further factorized into the product of SDCs and NRQCD long-distance matrix elements (LDMEs). The NRQCD operators forming the LDMEs are determined with assumptions on the quantum numbers of the final state tetraquark. We constructed all leading order S-wave NRQCD local operators. With heavy quarks, vacuum saturation approximation is also applied, which limits the inclusive final states to be only a single gluon. We then acquired the SDCs by matching the FF and the LDMEs with free four-quark states. The factorization formula is now complete, and to obtain the cross section, we need the value of the LDMEs from lattice QCD.

As an attempt to the value of the production cross section, however, we used diquark and tetraquark wave functions at the origin to obtain the value of the LDMEs. Furthermore, we also dropped the $\mathbf{6}\otimes \mathbf{\bar 6}$ component of the tetraquark. As a result, we obtained the $p_T$ distributions of the inclusive fully-heavy tetraquark production. We also tried to obtain the total cross sections and event numbers at {HL-LHC}. However, due to the limitation of our factorization, we can not approach the small $p_T$ region, which contributes to most parts of the total cross section.

We studied the exclusive production processes at B factory as well~\cite{Feng2020a}. We gave both the amplitude and cross section level NRQCD factorization formula, and calculated the perturbative helicity amplitudes, and in turn the helicity selection rules. We matched their SDCs and the SDCs for cross sections. Again we took the wave functions at the origin to estimate the value of LDMEs, and we obtained both the polarized and unpolarized cross section. 


\subsection*{Meson-meson scattering in 1+1 dimension}
The notorious nonperturbative nature of QCD in 3+1 dimension has been a major roadblock for decades. One of the main nonperturbative methods in the market is large-$N_c$ expansion, but it can hardly make any concrete quantitative predictions in 3+1 dimension. In 1+1 dimension, however, there is neither gluon degree-of-freedom nor spin, angular momentum or transverse momentum. As proposed by 't Hooft in 1973, in 1+1-dimensional QCD, the meson wave function can be obtained by applying large-$N_c$ expansion in light-cone gauge. In 1+1 dimension, there are rigorously only planar diagrams, thus the so-called “rainbow approximation” is exact. One can solve the Dyson-Schwinger equation to obtain the dressed quark propagator, then solve the Bethe-Salpeter equation for $q\bar q$ pair to obtain the mass eigenvalues and wave functions of the mesons. The latter equation is exactly the celebrated 't Hooft equation.

Having the wave functions of the mesons opens up lots of opportunities to investigate the nonperturbative properties of QCD. Some tried to search for indications of tetraquarks in the meson-meson scattering amplitudes. Among those, Batiz et al. claimed to have found possible Breit-Wigner resonance in the amplitudes. The goal of our work is to examine such a conclusion.

In order to obtain the meson-meson scattering amplitudes, we can again use the Bethe-Salpeter equation to derive the vertex function between meson and $q\bar q$ pair. Then it is straightforward to get the scattering amplitudes. It is worth noting that Batiz et al. failed to distinguish the outgoing and incoming quark for the meson vertex, and they missed the gluon-exchanging diagrams. We eventually calculated the scattering amplitudes numerically, and we found no evidence of tetraquark or possible resonance structure. We concluded that the structure indicating the presence of tetraquarks can not be found at leading order of large-$N_c$ expansion, and the near-threshold enhancements we observed in our numerical results are merely kinematic effects.

\section*{Research Interests and Future Plans}
During my postdoctoral period, I would like to continue working on OPE and atomic wave functions. In our first paper, we only discussed the asymptotic behavior of one electron going closer to one nucleus. However, we can extend the method to the coalescence behavior of multiple electrons and nuclei. Also, we can discuss higher partial-wave operators and their impacts on the wave function. While one could measure the wave functions with weak value measurement, our method can provide insights on other atomic distributions.
OPE could also apply to cold atom systems or nuclear systems, as first proposed by Braaten et al. in 2008. It is a possible direction to look into Coulomb interactions in nuclear systems, and investigate the long-distance quantity ``contact'' in such scenario. And I could also include higher order interactions in the short-range EFT.

Another field of research I am interested in is the NRQCD factorization. With more experimental data in the future, I could further work on the production and decay properties of fully-heavy tetraquark. The physics of quarkonia or top quark and how NRQCD factorization works on these subjects also intrigues me. Apart from these, I have diverse interests in perturbative QCD and relevant topics as well.

% In addition, I hope I can learn more about various EFTs and how they uniquely stand at their own scale of physics. 

\bibliography{papers}
\bibliographystyle{apsrev4-1}


\end{document}