
\PassOptionsToPackage{unicode}{hyperref}
\PassOptionsToPackage{naturalnames}{hyperref}
\documentclass{article}
\usepackage{geometry}
%\usepackage{fullpage}
\usepackage{parskip}
\usepackage{physics}
\usepackage{amsmath}
\usepackage{amssymb}
\usepackage{xcolor}
\usepackage[colorlinks,linkcolor=blue,citecolor=green]{hyperref}
\usepackage{array}
\usepackage{longtable}
\usepackage{multirow}
\usepackage{comment}
\usepackage{graphicx}
\usepackage{cite}
\usepackage{amsfonts}
\usepackage{bm}
\usepackage{slashed}
\usepackage{dsfont}
\usepackage{mathtools}
\usepackage[compat=1.1.0]{tikz-feynman}
\usepackage{simplewick}
%\usepackage{fourier}
%\usepackage{slashbox}
%\usepackage{intent}
\usepackage{mathrsfs}
\usepackage{xparse}
\usepackage{enumerate}

\geometry{left=0.9cm,right=0.9cm,top=1.5cm,bottom=2cm}

\newcommand{\gm}{\gamma^{\mu}}
\newcommand{\gn}{\gamma^{\nu}}
\newcommand{\gs}{\gamma^{\sigma}}
\newcommand{\gr}{\gamma^{\rho}}
\newcommand{\gnr}{g^{\nu\rho}}
\newcommand{\gmr}{g^{\mu\rho}}
\newcommand{\gms}{g^{\mu\sigma}}
\newcommand{\gns}{g^{\nu\sigma}}
\newcommand{\vbp}{\vb{p}}
\newcommand{\vbk}{\vb{k}}
\newcommand{\g}{\gamma}
\renewcommand{\a}{\alpha}
\renewcommand{\b}{\beta}
\renewcommand{\t}{\theta}
\newcommand{\la}{\lambda}
\newcommand{\p}{\phi}
\newcommand{\vp}{\varphi}
\newcommand{\s}{\sigma}
\renewcommand{\G}{\Gamma}
\newcommand{\pars}{\slashed\partial}
\newcommand{\ps}{\slashed p}
\newcommand{\ks}{\slashed k}
\newcommand{\lag}{\mathcal{L}}
\newcommand{\da}{^{\dagger}}
\newcommand{\sm}{^{\mu}}
\newcommand{\sn}{^{\nu}}
\newcommand{\smn}{^{\mu\nu}}
\newcommand{\Dm}{D^{\mu}}
\newcommand{\dm}{\partial^{\mu}}
\newcommand{\Asquare}{A^{\mu}A_{\mu}}
\newcommand{\partialsquare}[2]{\partial^{\mu}{#1}\partial_{\mu}{#2}}

\title{$\bar c\gm c$ matrix element}
\author{Yingsheng Huang}
\begin{document}
\maketitle
\section{Kinematics and Conventions}
Quark and antiquark momenta are
\begin{align}
    &p_1=P/2+p=(E,\vb{p})\\
    &p_2=P/2-p=(E,-\vb{p})
\end{align}
where in rest frame
\begin{align}
    &P=(2E(p),0)\\
    &p=(0,\vb{p})
\end{align}
Dirac spinors are normalized as following
\begin{align}
    &u(\vb{p})=\sqrt{\frac{E+m}{2E}}\pmqty{\xi\\\frac{\vb{p}\cdot\sigma}{E+m}\xi}\\
    &v(\vb{p})=\sqrt{\frac{E+m}{2E}}\pmqty{\frac{-\vb{p}\cdot\sigma}{E+m}\eta\\\eta}
\end{align}
where
\begin{align}
    &\xi=\pmqty{1\\0}\text{ or }\pmqty{0\\1}\\
    &\eta=\pmqty{0\\1}\text{ or }\pmqty{-1\\0}
\end{align}
\section{State Projection}
The bound state is\cite{Weinberg2015}
\begin{align}
    \ket{P,E;J,m_j;L;S}=\int\dd\Omega_{\vb{p_1}}\sum_{s_1s_2s_zm_l}Y_l^m(\hat{\vb{p_1}})\braket{S_1\lambda_{1}S_2\lambda_{2}}{Ss_z}\braket{Ss_zLm_l}{Jm_J}\ket{\vb{p_1},\lambda_{1}}\ket{\vb{P-p_1},\lambda_{2}}
\end{align}
\section{Bilinears\cite{Bodwin:2002hg}}
\begin{align}
    \Pi_0(P,p)&\equiv-\sum_{\lambda_1,\lambda_2}u(\vb{p},\lambda_1)\bar v(-\vb{p},\lambda_2)\braket{\frac{1}{2}\lambda_1\frac{1}{2}\lambda_2}{00}\\
    &=\frac{1}{2\sqrt{2}E(E+m)}\pqty{\frac{1}{2}\slashed P+m+\slashed p}\frac{\slashed P+2E}{4E}\gamma_5\pqty{\frac{1}{2}\slashed P-m-\slashed p}\\
    \Pi_1(P,p)&\equiv\sum_{\lambda_1,\lambda_2}u(\vb{p},\lambda_1)\bar v(-\vb{p},\lambda_2)\braket{\frac{1}{2}\lambda_1\frac{1}{2}\lambda_2}{1\epsilon}\\
    &=\frac{-1}{2\sqrt{2}E(E+m)}\pqty{\frac{1}{2}\slashed P+m+\slashed p}\frac{\slashed P+2E}{4E}\slashed\epsilon\pqty{\frac{1}{2}\slashed P-m-\slashed p}
\end{align}
\section{$^3S_1$}
Average over $\vb{p}$ with Bodwin's convention (extra $1/(4\pi)$):
$$\mel{0}{\bar c\gm c}{^3S_1}^{(0)}=\frac{1}{4\pi}\int\dd\Omega\tr[\Pi_1\gm]=\sqrt{2}\left(\frac{m}{3 E}+\frac{2}{3}\right)\epsilon^{\mu}$$
\section{$^3D_1$}
The matrix element reads:
$$\mel{0}{\bar c\gm c}{^3D_1}^{(0)}=\int\dd\Omega\sum_{\la_1\la_2 s_zm_l}\tr{\Pi_1\gm}\braket{2m_l;1s_z}{1J_z}Y_{2m_l}(\theta,\phi)$$
while the trace part is the same as $^3S_1$:
$$\tr{\Pi_1\gm}=\frac{\sqrt{2}p^{\mu}(p\cdot\epsilon)}{E(E+m)}+\epsilon^{\mu}$$
Chosen polarization vectors: 
$$\epsilon^{(-)}=\frac{1}{\sqrt{2}}(0,1,-i,0),\epsilon^{(0)}=(0,0,0,1),\epsilon^{(+)}=\frac{1}{\sqrt{2}}(0,-1,-i,0)$$
Result (the first row and the last are orthogonal):  
%$$\left(
%\begin{array}{cccc}
% 0 & -\frac{4 p^2 \sqrt{2 \pi }}{15 E (m+E)} & \frac{4 i p^2 \sqrt{2 \pi }}{15 E (m+E)} & 0 \\
% 0 & 0 & 0 & -\frac{4 p^2 \sqrt{\pi }}{15 E (m+E)} \\
% 0 & \frac{4 p^2 \sqrt{2 \pi }}{15 E (m+E)} & \frac{4 i p^2 \sqrt{2 \pi }}{15 E (m+E)} & 0 \\
%\end{array}
%\right)$$
$$\left(
\begin{array}{c|cccc}
    \epsilon^{(-)} & 0 & \frac{2 \sqrt{2 \pi } p^2}{3 E (m+E)} & -\frac{2 i \sqrt{2 \pi } p^2}{3 E (m+E)} & 0 \\\hline
    \epsilon^{(0)} & 0 & 0 & 0 & \frac{4 \sqrt{\pi } p^2}{3 E (m+E)} \\\hline
    \epsilon^{(+)} & 0 & -\frac{2 \sqrt{2 \pi } p^2}{3 E (m+E)} & -\frac{2 i \sqrt{2 \pi } p^2}{3 E (m+E)} & 0 \\
\end{array}
\right)$$
and the decay constant is $\frac{4 \sqrt{\pi } p^2}{3 E (E+m)}$ where $p^2=\vb{p}^2 =E^2-m^2$. 

\section{NRQCD $^3D_1$}
\begin{align}
    \langle 0 | \frac {1} {2 m^{2}} \chi^{\dagger} D^{\{i} D^{j \}} \sigma^{j} \psi | Q \overline {Q} [^{3} D_{1} (\varepsilon) ] \rangle^{(0)} &=\int\dd\Omega\sum_{\la_1\la_2 s_zm_l}\frac{-p^{\{i}p^{j\}}}{2m^2}\eta^\dagger_{\lambda_2}\sigma^j\xi_{\lambda_1}\braket{2m_l;1s_z}{1J_z}\braket{\frac{1}{2}\lambda_1\frac{1}{2}\lambda_2}{1s_z}Y_{2m_l}(\theta,\phi)\\
    &=  \sqrt {\pi} \frac {4 p^{2}} {3 m^{2}} \varepsilon^{i}
\end{align}
if $\partial^{\{i} \partial^{j \}}=2\partial^{i} \partial^{j }$. 

\bibliography{Bib.bib}
\bibliographystyle{apsrev4-1}

\end{document}
