\PassOptionsToPackage{unicode}{hyperref}
\PassOptionsToPackage{naturalnames}{hyperref}
\documentclass{article}
\usepackage{geometry}
%\usepackage{fullpage}
\usepackage{parskip}
\usepackage{physics}
\usepackage{amsmath}
\usepackage{amssymb}
\usepackage{xcolor}
\usepackage[colorlinks,linkcolor=blue,citecolor=green]{hyperref}
\usepackage{array}
\usepackage{longtable}
\usepackage{multirow}
\usepackage{comment}
\usepackage{graphicx}
\usepackage{cite}
\usepackage{amsfonts}
\usepackage{bm}
\usepackage{slashed}
\usepackage{dsfont}
\usepackage{mathtools}
\usepackage[compat=1.1.0]{tikz-feynman}
\usepackage{simplewick}
%\usepackage{fourier}
%\usepackage{slashbox}
%\usepackage{intent}
\usepackage{mathrsfs}
\usepackage{xparse}
\usepackage{enumerate}

\geometry{left=0.9cm,right=0.9cm,top=1.5cm,bottom=2cm}

\newcommand{\gm}{\gamma^{\mu}}
\newcommand{\gn}{\gamma^{\nu}}
\newcommand{\gs}{\gamma^{\sigma}}
\newcommand{\gr}{\gamma^{\rho}}
\newcommand{\gnr}{g^{\nu\rho}}
\newcommand{\gmr}{g^{\mu\rho}}
\newcommand{\gms}{g^{\mu\sigma}}
\newcommand{\gns}{g^{\nu\sigma}}
\newcommand{\vbp}{\vb{p}}
\newcommand{\vbk}{\vb{k}}
\newcommand{\g}{\gamma}
\renewcommand{\a}{\alpha}
\renewcommand{\b}{\beta}
\renewcommand{\t}{\theta}
\newcommand{\la}{\lambda}
\newcommand{\p}{\phi}
\newcommand{\vp}{\varphi}
\newcommand{\s}{\sigma}
\renewcommand{\G}{\Gamma}
\newcommand{\pars}{\slashed\partial}
\newcommand{\ps}{\slashed p}
\newcommand{\ks}{\slashed k}
\newcommand{\lag}{\mathcal{L}}
\newcommand{\da}{^{\dagger}}
\newcommand{\sm}{^{\mu}}
\newcommand{\sn}{^{\nu}}
\newcommand{\smn}{^{\mu\nu}}
\newcommand{\Dm}{D^{\mu}}
\newcommand{\dm}{\partial^{\mu}}
\newcommand{\Asquare}{A^{\mu}A_{\mu}}
\newcommand{\partialsquare}[2]{\partial^{\mu}{#1}\partial_{\mu}{#2}}

\title{$\bar c\gm c$ matrix element}
\author{Yingsheng Huang}
\begin{document}
\maketitle
\section{$^3S_1$}
Ignore the overall factor:
$$\mel{0}{\bar c\gm c}{^3S_1}=\int\dd\Omega\tr[\Pi_1\gm]\propto\sqrt{2}\pi(\frac{m}{3 E}+\frac{2}{3})\epsilon^{\mu}$$
\section{$^3D_1$}
The matrix element reads:
$$\mel{0}{\bar c\gm c}{^3D_1}=\int\dd\Omega\sum_{\la_1\la_2 S_zm}\tr{\Pi_1\gm}\braket{1J_z}{2m;1S_z}Y_{2m}(\theta,\phi)$$
while the trace part is the same as $^3S_1$:
$$\tr{\Pi_1\gm}=\frac{\sqrt{2}p^{\mu}(p\cdot\epsilon)}{E(E+m)}+\epsilon^{\mu}$$
Chosen polarization vectors: 
$$\epsilon^{(-)}=\frac{1}{\sqrt{2}}(0,1,-i,0),\epsilon^{(0)}=(0,0,0,1),\epsilon^{(+)}=\frac{1}{\sqrt{2}}(0,-1,-i,0)$$
Result (the first row and the last are orthogonal):  
%$$\left(
%\begin{array}{cccc}
% 0 & -\frac{4 p^2 \sqrt{2 \pi }}{15 E (m+E)} & \frac{4 i p^2 \sqrt{2 \pi }}{15 E (m+E)} & 0 \\
% 0 & 0 & 0 & -\frac{4 p^2 \sqrt{\pi }}{15 E (m+E)} \\
% 0 & \frac{4 p^2 \sqrt{2 \pi }}{15 E (m+E)} & \frac{4 i p^2 \sqrt{2 \pi }}{15 E (m+E)} & 0 \\
%\end{array}
%\right)$$
$$\left(
\begin{array}{cccc}
 0 & \frac{2 \sqrt{2 \pi } p^2}{3 E (m+E)} & -\frac{2 i \sqrt{2 \pi } p^2}{3 E (m+E)} & 0 \\
 0 & 0 & 0 & \frac{4 \sqrt{\pi } p^2}{3 E (m+E)} \\
 0 & -\frac{2 \sqrt{2 \pi } p^2}{3 E (m+E)} & -\frac{2 i \sqrt{2 \pi } p^2}{3 E (m+E)} & 0 \\
\end{array}
\right)$$
<<<<<<< HEAD
and the decay constant is $\frac{4 \sqrt{\pi } p^2}{3 E (E+m)}$ where $p=\vb{p}=E^2-m^2$.
=======
and the decay constant is $\frac{4 \sqrt{\pi } p^2}{3 E (E+m)}$.
>>>>>>> a4e2822326627fe8a0fd948898a4e83aef8061fd

\end{document}
