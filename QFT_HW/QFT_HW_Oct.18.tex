%!mode::"Tex:UTF-8"
\PassOptionsToPackage{unicode}{hyperref}
\PassOptionsToPackage{naturalnames}{hyperref}
\documentclass{article}
\usepackage{fullpage}
\usepackage{parskip}
\usepackage{physics}
\usepackage{amsmath}
\usepackage{amssymb}
\usepackage{xcolor}
\usepackage[colorlinks,linkcolor=blue,citecolor=green]{hyperref}
\usepackage{array}
\usepackage{longtable}
\usepackage{multirow}
\usepackage{comment}
\usepackage{graphicx}
\usepackage{cite}
\usepackage{amsfonts}
\usepackage{slashed}
%\usepackage{bm}
\usepackage{dsfont}
%\usepackage{fourier}
%\usepackage{slashbox}
%\usepackage{intent}

\newcommand{\ac}[1]{a_{\vb{#1}}}
\newcommand{\ad}[1]{a_{\vb{#1}}^{\dagger}}
\newcommand{\bc}[1]{b_{\vb{#1}}}
\newcommand{\bd}[1]{b_{\vb{#1}}^{\dagger}}
\newcommand{\omp}[1]{\omega_{\vb{#1}}}
\newcommand{\intphead}[1]{\int\frac{\dd^3#1}{(2\pi)^3}}
\newcommand{\del}[2]{\delta_{#1#2}}
\newcommand{\id}{\int\dd^4x}
\newcommand{\lag}{\mathcal{L}}
\newcommand{\bm}[1]{\boldsymbol{#1}}
\newcommand{\trh}{\Lambda_{\frac{1}{2}}}
\newcommand{\gm}{\gamma^{\mu}}
\newcommand{\gn}{\gamma^{\nu}}
\newcommand{\gs}{\gamma^{\sigma}}
\newcommand{\gr}{\gamma^{\rho}}
\newcommand{\gmn}{g^{\mu\nu}}
\newcommand{\gnm}{g^{\nu\mu}}
\newcommand{\gmm}{g^{\mu\mu}}
\newcommand{\gnn}{g^{\nu\nu}}
\newcommand{\gnr}{g^{\nu\rho}}
\newcommand{\gmr}{g^{\mu\rho}}
\newcommand{\gms}{g^{\mu\sigma}}
\newcommand{\gns}{g^{\nu\sigma}}
\newcommand{\g}{\gamma}
\renewcommand{\a}{\alpha}
\renewcommand{\b}{\beta}
\renewcommand{\t}{\theta}
\newcommand{\p}{\phi}
\newcommand{\vp}{\varphi}
\renewcommand{\G}{\Gamma}
\newcommand{\s}{\sigma}
\newcommand{\ps}{\slashed p}


\title{Homework: Quantum Field Theory \#3}
\author{Yingsheng Huang}
\begin{document}
\maketitle
%\tableofcontents

{\bf11.1}\quad

(a). $(\gamma^5)^2=\mathds{1}$
\begin{align*}
  (\g^5)^2&=\g^0\g^1\g^2\g^3\g^0\g^1\g^2\g^3&\\
  &=(-1)^3\g^0\g^0\g^1\g^2\g^3\g^1\g^2\g^3\\
  &=(-1)^3\g^0\g^0(-1)^2\g^1\g^1\g^2\g^3\g^2\g^3\\
  &=(-1)^3\g^0\g^0(-1)^2\g^1\g^1(-1)\g^2\g^2\g^3\g^3\\
  &=\mathds{1}
\end{align*}


(b). $\g_{\mu}\slashed p\g^{\mu}=-2\slashed p$
\begin{align*}
  \g_{\mu}\ps\g^{\mu}&=g_{\mu\a}\g^{\a}\g^{\nu}p_{\nu}\g^{\mu}&\\
  &=(2g^{\a\nu}-\g^{\nu}\g^{\a})\g^{\mu}g_{\mu\a}p_{\nu}\\
  &=2\g^{\nu}p_{\nu}-\g^{\nu}\g_{\mu}\g^{\mu}p_{\nu}\\
  &=-2\ps
\end{align*}

(c). $\g_{\mu}\ps\slashed q \ps\g^{\mu}=-2\ps\slashed q\ps$
\begin{align*}
  \g_{\mu}\ps\slashed q \ps\g^{\mu}&=\g_{\mu}\g^{\nu}p_{\nu}\g^{\a}q_{\a}\g^{\b}p_{\b}\g^{\mu}&\\
  &=g_{\mu\tau}\g^{\tau}\g^{\nu}p_{\nu}\g^{\a}q_{\a}\g^{\b}p_{\b}\g^{\mu}\\
  &=(2g^{\nu\tau}-\g^{\nu}\g^{\tau})g_{\mu\tau}\g^{\a}\g^{\b}\g^{\mu}p_{\nu}q_{\a}p_{\b}\\
  &=2\g^{\a}\g^{\b}\g^{\nu}p_{\nu}q_{\a}p_{\b}-\g^{\nu}\g^{\tau}\g^{\a}\g^{\b}\g^{\mu}g_{\mu\tau}p_{\nu}q_{\a}p_{\b}\\
  &=2\g^{\a}\g^{\b}\g^{\nu}p_{\nu}q_{\a}p_{\b}-\g^{\nu}(2g^{\a\tau}-\g^{\a}\g^{\tau})\g^{\b}\g^{\mu}g_{\mu\tau}p_{\nu}q_{\a}p_{\b}\\
  &=2\g^{\a}\g^{\b}\g^{\nu}p_{\nu}q_{\a}p_{\b}-2\g^{\nu}\g^{\b}\g^{\a}p_{\nu}q_{\a}p_{\b}+\g^{\nu}\g^{\a}\g^{\tau}\g^{\b}\g^{\mu}g_{\mu\tau}p_{\nu}q_{\a}p_{\b}\\
  &=2\g^{\a}\g^{\b}\g^{\nu}p_{\nu}q_{\a}p_{\b}-2\g^{\nu}\g^{\b}\g^{\a}p_{\nu}q_{\a}p_{\b}+\g^{\nu}\g^{\a}(2g^{\tau\b}-\g^{\b}\g^{\tau})\g^{\mu}g_{\mu\tau}p_{\nu}q_{\a}p_{\b}\\
  &=2\g^{\a}\g^{\b}\g^{\nu}p_{\nu}q_{\a}p_{\b}-2\g^{\nu}\g^{\b}\g^{\a}p_{\nu}q_{\a}p_{\b}+2\g^{\nu}\g^{\a}\g^{\b}p_{\nu}q_{\a}p_{\b}-4\g^{\nu}\g^{\a}\g^{\b}p_{\nu}q_{\a}p_{\b}\\
  &=-2\ps\slashed q\ps
\end{align*}

(d). $\Bqty{\g^5,\g^{\mu}}=0$
$$\g^5\g^0=(-1)^3\g^0\g^0\g^1\g^2\g^3=-\g^0\g^5$$
$$\g^5\g^1=(-1)^3\g^1\g^0\g^1\g^2\g^3=-\g^1\g^5$$
$$\g^5\g^2=(-1)^3\g^2\g^0\g^1\g^2\g^3=-\g^2\g^5$$
$$\g^5\g^3=(-1)^3\g^3\g^0\g^1\g^2\g^3=-\g^3\g^5$$
$$\Longrightarrow\Bqty{\g^5,\g^{\mu}}=0$$
%\begin{align*}
%  \Bqty{\g^5,\g^{\mu}}&=\g^5\g^{\mu}+\g^{\mu}\g^5\\
%  &=\g^0\g^1\g^2\g^3\gm+\gm\g^0\g^1\g^2\g^3\\
%  &=2g^{\mu3}\g^0\g^1\g^2-\g^0\g^1\g^2\gm\g^3+2g^{\mu0}\g^1\g^2\g^3-\g^0\gm\g^1\g^2\g^3\\
%  &=2g^{\mu3}\g^0\g^1\g^2-2\g^0\g^1g^{\mu2}\g^3+\g^0\g^1\gm\g^2\g^3+2g^{\mu0}\g^1\g^2\g^3-\g^0g^{\mu1}\g^2\g^3+\g^0\g^1\gm\g^2\g^3
%\end{align*}

(e). $\Tr[\g^{\a}\gm\g^{\b}\gn]=4(g^{\a\mu}g^{\b\nu}-g^{\a\b}g^{\mu\nu}+g^{\a\nu}g^{\mu\b})$
\begin{align*}
  \Tr[\gm\gn]&=\Tr[2g^{\mu\nu}\cdot\mathds{1}-\gn\gm]\\
  &=8g^{\mu\nu}-\Tr[\gm\gn]
\\
&=4g^{\mu\nu}
\end{align*}
\begin{align*}
  \Tr[\g^{\a}\gm\g^{\b}\gn]&=\Tr[(2g^{\a\mu}-\gm\g^{\a})\g^{\b}\gn]\\
  &=\Tr[2g^{\a\mu}\g^{\b}\gn-2g^{\a\b}\gm\gn+2g^{\a\nu}\gm\g^{\b}-\gm\g^{\b}\gn\g^{\a}]\\&=g^{\a\mu}\Tr[\g^{\b}\gn]-g^{\g\b}\Tr[\gm\gn]+g^{\a\nu}\Tr[\gm\g^{\b}]\\
  &=4(g^{\a\mu}g^{\b\nu}-g^{\g\b}g^{\mu\nu}+g^{\a\nu}g^{\mu\b})\\
\end{align*}

{\bf2.}\quad
Spinor identity:

(a). Show that $\sum_su_s(p)\bar u_s(p)=\ps+m$ and $\sum_sv_s(p)\bar v_s(p)=\ps-m$.
\begin{align*}
  \sum_su_s(p)\bar u_s(p)&=\sum_s\pmqty{\sqrt{p\cdot\sigma}\xi^s\\\sqrt{p\cdot\bar\sigma}\xi^s}\pmqty{\xi^{s\dagger}\sqrt{p\cdot\bar\sigma}&\xi^{s\dagger}\sqrt{p\cdot\sigma}}\\
  &=\sum_s\pmqty{\sqrt{p\cdot\sigma}\xi^s\xi^{s\dagger}\sqrt{p\cdot\bar\sigma}&\sqrt{p\cdot\sigma}\xi^s\xi^{s\dagger}\sqrt{p\cdot\sigma}\\\sqrt{p\cdot\bar\sigma}\xi^s\xi^{s\dagger}\sqrt{p\cdot\bar\sigma}&\sqrt{p\cdot\bar\sigma}\xi^s\xi^{s\dagger}\sqrt{p\cdot\sigma}}\\
  \intertext{(Use $\sum_s\xi^s\xi^{s\dagger}=\mathds{1}$)}
  &=\pmqty{m&p\cdot\sigma\\p\cdot\bar\sigma&m}\\
  &=\ps+m
\end{align*}
Similarly, $\sum_sv_s(p)\bar v_s(p)=\ps-m$.

(b). Show that $\bar u_{\s}(p)\gm u_{\s'}(p)=2\delta_{\s\s'}p^{\mu}$.

\begin{align*}
  &\bar u_{\s}(p)\gm u_{\s'}(p)=2\delta_{\s\s'}p^{\mu}\\
  \Longrightarrow &p_{\mu}\bar u_{\s}(p)\gm u_{\s'}(p)=2\delta_{\s\s'}p_{\mu}p^{\mu}\\
  \intertext{From the dirac equation for $\bar u$, we have} &\bar u_{\s}\gm p_{\mu}=m\bar u_{\s}
\end{align*}
So
\begin{align*}
  &p_{\mu}\bar u_{\s}(p)\gm u_{\s'}(p)=2\delta_{\s\s'}p_{\mu}p^{\mu}\\
  \Longrightarrow &m\bar u_{\s}u_{\s'}=2\delta_{\s\s'}p_{\mu}p^{\mu}\\
  \Longrightarrow &2m^2\xi_{\s}^{\dagger}\xi_{\s'}=2\delta_{\s\s'}m^2\\
  \Longrightarrow&2\delta_{\s\s'}m^2=2\delta_{\s\s'}m^2
\end{align*}
More strict prove can be done by involving Gordon identity, which is shown in the last of problem {\bf11.4}.

We can also compute only the third component of $p^{i}$, then do a coordinate transformation to the actual $p^{\mu}$ with all 4 component.
\begin{align*}
  \bar u_{\s}(p)\gm u_{\s'}(p)&=\pmqty{\xi^{\s\dagger}\sqrt{p\cdot\bar\sigma}&\xi^{\s\dagger}\sqrt{p\cdot\sigma}}\gm\pmqty{\sqrt{p\cdot\sigma}\xi^{\s'}\\\sqrt{p\cdot\bar\sigma}\xi^{\s'}}\\
  &=\pmqty{\xi^{\s\dagger}\sqrt{p\cdot\bar\sigma}&\xi^{\s\dagger}\sqrt{p\cdot\sigma}}\pmqty{0&\s^{\mu}\\\bar\s^{\mu}&0}\pmqty{\sqrt{p\cdot\sigma}\xi^{\s'}\\\sqrt{p\cdot\bar\sigma}\xi^{\s'}}\\
  &=\pmqty{\xi^{\s\dagger}\sqrt{p\cdot\sigma}\bar\s^{\mu}&\xi^{\s\dagger}\sqrt{p\cdot\bar\sigma}\s^{\mu}}\pmqty{\sqrt{p\cdot\sigma}\xi^{\s'}\\\sqrt{p\cdot\bar\sigma}\xi^{\s'}}\\
  &=\xi^{\s\dagger}\sqrt{p\cdot\sigma}\bar\s^{\mu}\sqrt{p\cdot\sigma}\xi^{\s'}+\xi^{\s\dagger}\sqrt{p\cdot\bar\sigma}\s^{\mu}\sqrt{p\cdot\bar\sigma}\xi^{\s'}\\
  \intertext{Insert $p_\mu=\pmqty{E,0,0,-p^3}$ condition}
  &=\xi^{\s\dagger}\pmqty{\sqrt{E-p_3}\\&\sqrt{E+p_3}}\bar\s^{\mu}\pmqty{\sqrt{E-p_3}\\&\sqrt{E+p_3}}\xi^{\s'}\\&\;\;\;\;\;+\xi^{\s\dagger}\pmqty{\sqrt{E+p_3}\\&\sqrt{E-p_3}}\s^{\mu}\pmqty{\sqrt{E+p_3}\\&\sqrt{E-p_3}}\xi^{\s'}\\
  \intertext{if only $p_3$ component exists, all component of $\mu$ but $3$ are zero.}
%  &=\xi^{\s\dagger}E\bar\s^{\mu}\xi^{\s'}+\xi^{\s\dagger}E\s^{\mu}\xi^{\s'}\\
  &=\xi^{\s\dagger}\pmqty{\sqrt{E-p_3}\\&\sqrt{E+p_3}}\pmqty{-1\\&1}\pmqty{\sqrt{E-p_3}\\&\sqrt{E+p_3}}\xi^{\s'}\\&\;\;\;\;\;+\xi^{\s\dagger}\pmqty{\sqrt{E+p_3}\\&\sqrt{E-p_3}}\pmqty{1\\&-1}\pmqty{\sqrt{E+p_3}\\&\sqrt{E-p_3}}\xi^{\s'}\\
  &=\xi^{\s\dagger}\pmqty{2p_3\\&2p_3}\xi^{\s'}\\
  &=2p_3\xi^{\s\dagger}\xi^{\s'}\\
  &=2p_3\delta_{\s\s'}
\end{align*}
Choose a different coordinate system and we have a set of new $p_{\mu}$, with gives $\bar u_{\s}(p)\gm u_{\s'}(p)=2\delta_{\s\s'}p^{\mu}$.




{\bf{3.2}}\quad Derive the \emph{Gordon identity}
\begin{align}\label{40}
  \bar{u}(p')\gamma^{\mu}u(p)=\bar{u}(p')\bqty{\frac{p'^{\mu}+p^{\mu}}{2m}+\frac{i\sigma^{\mu\nu}q_{\nu}}{2m}}u(p)
\end{align}
where $q=(p'-p)$.

From the standard covariant form of Dirac equation
\begin{align*}
  (i\gamma^{\mu}\partial_{\mu}-m)\psi(x)=0
\end{align*}
and can be written as
\begin{align}
  \gamma^{\mu}p_{\mu}u(p)=m u(p)
\end{align}
From previous definition
\begin{align*}
  \bar{u}(p)\equiv u^{\dagger}(p)\gamma^0
\end{align*}
and
\begin{align*}
  u^{\dagger}(p)p_{\mu}^{\dagger}(\gamma^{\mu})^{\dagger}=m u^{\dagger}(p)
\end{align*}
So we have
\begin{align*}
  \bar{u}(p)\gamma^0p_{\mu}^{\dagger}(\gamma^{\mu})^{\dagger}\gamma^0=m \bar{u}(p)
\end{align*}
Then
\begin{align*}
  \bar{u}(p')\gamma^{\mu}u(p)&=\frac{\bar{u}(p')\gamma^0p'_{\mu'}^{\dagger}(\gamma^{\mu'})^{\dagger}\gamma^0}{m}\gamma^{\mu}\frac{\gamma^{\mu''}p_{\mu''}u(p)}{m}\\
  &=\bar{u}(p')\frac{\gamma^0p'_{\mu'}^{\dagger}(\gamma^{\mu'})^{\dagger}\gamma^0\gamma^{\mu}\gamma^{\mu''}p_{\mu''}}{m^2}u(p)
\end{align*}
Note that $p_{\mu}$ and $\gamma$ commute, and
\begin{align*}
  \gamma^0(\gamma^{\mu})^{\dagger}\gamma^0&=\Pmqty{0&1\\1&0}\Pmqty{0&\sigma^{\mu}\\-\sigma^{\mu}&0}^{\dagger}\Pmqty{0&1\\1&0}\\
  &=\Pmqty{0&\sigma^{\mu}\\-\sigma^{\mu}&0}\\&=
  \gamma^{\mu}
\end{align*}
which means
\begin{align*}
  \bar{u}(p)\gamma^{\mu}p_{\mu}=m \bar{u}(p)
\end{align*}
and
\begin{align*}
  \bar{u}(p')\gamma^{\mu}u(p)&=\bar{u}(p')\frac{\gamma^{\nu}p'_{\nu}\gamma^{\mu}\gamma^{\nu}p_{\nu}}{m^2}u(p)
\end{align*}

Now we observe
\begin{align*}
  i\sigma^{\mu\nu}q_{\nu}&=-\frac{1}{2}[\gamma^{\mu},\gamma^{\nu}](p'_{\nu}-p_{\nu})\\
  &=-\frac{1}{2}(\gamma^{\mu}\gamma^{\nu}p'_{\nu}-\gamma^{\nu}\gamma^{\mu}p'_{\nu}-\gamma^{\mu}\gamma^{\nu}p_{\nu}+\gamma^{\nu}\gamma^{\mu}p_{\nu})
\end{align*}
and
\begin{align*}
  \gamma^{\mu}\gamma^{\nu}=-\gamma^{\nu}\gamma^{\mu}+2g^{\mu\nu}
\end{align*}
We have
\begin{align*}
  i\sigma^{\mu\nu}q_{\nu}&=-\frac{1}{2}(2\gamma^{\mu}\gamma^{\nu}p'_{\nu}-2g^{\mu\nu}p'_{\nu}-2\gamma^{\mu}\gamma^{\nu}p_{\nu}+2g^{\mu\nu}p_{\nu})\\&=(p'^{\mu}-p^{\mu})-\gamma^{\mu}\gamma^{\nu}(p'_{\nu}-p_{\nu})
\end{align*}
With this \eqref{40} becomes
\begin{align*}
  \bar{u}(p')\gamma^{\mu}u(p)&=\bar{u}(p')\bqty{\frac{p'^{\mu}+p^{\mu}}{2m}+\frac{(p'^{\mu}-p^{\mu})-\gamma^{\mu}\gamma^{\nu}(p'_{\nu}-p_{\nu})}{2m}}u(p)\\
  &=\bar{u}(p')\bqty{\frac{p'^{\mu}}{m}-\frac{\gamma^{\mu}\gamma^{\nu}(p'_{\nu}-p_{\nu})}{2m}}u(p)\\
  &=\bar{u}(p')\bqty{\frac{p'^{\mu}}{m}-\frac{\gamma^{\mu}\gamma^{\nu}(p'_{\nu}-p_{\nu})}{2m}}u(p)
  %\\  &=\bar{u}(p')\bqty{\frac{\gamma^{\mu}\gamma^{\nu}(p'_{\nu}-p_{\nu})}{4 m}}u(p)
\end{align*}
We know that
\begin{align*}
  \bar{u}(p')\frac{\gamma^{\nu}p'_{\nu}\gamma^{\mu}\gamma^{\nu}p_{\nu}}{m^2}u(p)&=\frac{1}{2}\Bqty{\bar{u}(p')\frac{-\gamma^{\nu}p'_{\nu}\gamma^{\nu}\gamma^{\mu}p_{\nu}+2\gamma^{\nu}p'_{\nu}g^{\mu\nu}p_{\nu}-\gamma^{\mu}p'_{\nu}\gamma^{\nu}\gamma^{\nu}p_{\nu}+2p'_{\nu}g^{\mu\nu}\gamma^{\nu}p_{\nu}}{m^2}u(p)}\\
  &=\frac{1}{2}\Bqty{\bar{u}(p')\frac{-m\gamma^{\nu}\gamma^{\mu}p_{\nu}+2\gamma^{\nu}p'_{\nu}g^{\mu\nu}p_{\nu}-\gamma^{\mu}p'_{\nu}\gamma^{\nu}m+2p'_{\nu}g^{\mu\nu}\gamma^{\nu}p_{\nu}}{m^2}u(p)}\\
  &=\bar{u}(p')\bqty{\frac{p'^{\mu}+p^{\mu}}{m}-\frac{\gamma^{\nu}\gamma^{\mu}p_{\nu}+\gamma^{\mu}p'_{\nu}\gamma^{\nu}}{2m}}u(p)\\
  &=\bar{u}(p')\bqty{\frac{p'^{\mu}}{m}-\frac{-\gamma^{\mu}\gamma^{\nu}p'_{\nu}+\gamma^{\mu}p'_{\nu}\gamma^{\nu}}{2m}}u(p)\\
  &=\bar{u}(p')\bqty{\frac{p'^{\mu}}{m}-\frac{\gamma^{\mu}\gamma^{\nu}(p'_{\nu}-p_{\nu})}{2m}}u(p)
\end{align*}
And it consists with the former one.

From Gorden identity $\bar{u}(p')\gamma^{\mu}u(p)=\bar{u}(p')\bqty{\frac{p'^{\mu}+p^{\mu}}{2m}+\frac{i\sigma^{\mu\nu}q_{\nu}}{2m}}u(p)$, we can derive ($p'=p$)
\begin{align*}
  \bar{u}(p)\gamma^{\mu}u(p)&=\bar u(p)\frac{p^{\mu}}{m}u(p)\\
  &=2\delta_{\s\s'}p^{\mu}
\end{align*}






\end{document}
