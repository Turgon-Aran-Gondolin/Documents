%!mode::"Tex:UTF-8"
\PassOptionsToPackage{unicode}{hyperref}
\PassOptionsToPackage{naturalnames}{hyperref}
\documentclass{article}
\usepackage{fullpage}
\usepackage{parskip}
\usepackage{physics}
\usepackage{amsmath}
\usepackage{amssymb}
\usepackage{xcolor}
\usepackage[colorlinks,linkcolor=blue,citecolor=green]{hyperref}
\usepackage{array}
\usepackage{longtable}
\usepackage{multirow}
\usepackage{comment}
\usepackage{graphicx}
\usepackage{cite}
\usepackage{amsfonts}
%\usepackage{bm}
%\usepackage{fourier}
%\usepackage{slashbox}
%\usepackage{intent}

\newcommand{\ac}[1]{a_{\vb{#1}}}
\newcommand{\ad}[1]{a_{\vb{#1}}^{\dagger}}
\newcommand{\bc}[1]{b_{\vb{#1}}}
\newcommand{\bd}[1]{b_{\vb{#1}}^{\dagger}}
\newcommand{\omp}[1]{\omega_{\vb{#1}}}
\newcommand{\intphead}[1]{\int\frac{\dd^3#1}{(2\pi)^3}}
\newcommand{\del}[2]{\delta_{#1#2}}
\newcommand{\id}{\int\dd^4x}
\newcommand{\lag}{\mathcal{L}}
\newcommand{\bm}[1]{\boldsymbol{#1}}
\newcommand{\trh}{\Lambda_{\frac{1}{2}}}
\newcommand{\gm}{\gamma^{\mu}}
\newcommand{\gn}{\gamma^{\nu}}
\newcommand{\gs}{\gamma^{\sigma}}
\newcommand{\gr}{\gamma^{\rho}}
\newcommand{\gnr}{g^{\nu\rho}}
\newcommand{\gmr}{g^{\mu\rho}}
\newcommand{\gms}{g^{\mu\sigma}}
\newcommand{\gns}{g^{\nu\sigma}}
\newcommand{\Jrsmn}{(\mathcal{J}^{\rho\sigma})^{\mu}_{\nu}}
\newcommand{\Srs}{S^{\rho\sigma}}
\newcommand{\bP}{\vb{P}}
\newcommand{\bA}{\vb{A}}
\newcommand{\ba}{\boldsymbol{\alpha}}
\newcommand{\po}{\vb{p}_1}
\newcommand{\ps}{\vb{p}_2}
\newcommand{\apo}{\abs{\vb{p}_1}}
\newcommand{\aps}{\abs{\vb{p}_2}}

\title{Homework: Quantum Field Theory \#3}
\author{Yingsheng Huang}
\begin{document}
\maketitle
{\bf1.}\quad We know that
\begin{align}
  \gamma^0_W=\pmqty{0&1\\1&0},\;\;\;\;\gamma^i_W=\pmqty{0&\sigma^i\\-\sigma^i&0}\\
  \gamma^0_D=\pmqty{1&0\\0&-1},\;\;\;\;\gamma^i_D=\pmqty{0&\sigma^i\\-\sigma^i&0}
\end{align}
and it must have the unitary transformation relation
\begin{align}
  \gamma^{\mu}_W=U\gamma^{\mu}_D U^{\dagger}
\end{align}
Now we start with $\gamma^0$. Form linear algebra, we know how to diagonalize a unitary matrix. It's easy to find
\begin{align*}
  U=\frac{1}{\sqrt{2}}\left(
\begin{array}{cccc}
 1 & 0 & -1 & 0 \\
 0 & 1 & 0 & -1 \\
 1 & 0 & 1 & 0 \\
 0 & 1 & 0 & 1 \\
\end{array}
\right)
\end{align*}
and after verifying, it consists with the other three matrices.

{\bf2.}\quad Verify $[\gamma^{\mu},S^{\rho\sigma}]=(\mathcal{J}^{\rho\sigma})^{\mu}_{\nu}\gamma^{\nu}$.

From the definition of $(\mathcal{J}^{\rho\sigma})_{\mu\nu}$, we have
\begin{align}
  (\mathcal{J}^{\mu\nu})_{\alpha\beta}=i(\delta^{\mu}_{\alpha}\delta^{\nu}_{\beta}-\delta^{\mu}_{\beta}\delta^{\nu}_{\alpha})
\end{align}
which means
\begin{align*}
  (\mathcal{J}^{\rho\sigma})^{\mu}_{\nu}=i(g^{\rho\mu}\delta^{\sigma}_{\nu}-g^{\sigma\mu}\delta^{\rho}_{\nu})
\end{align*}
With $S^{\mu\nu}=\frac{i}{4}[\gm,\gn]$ and $\rho\neq\sigma$ (otherwise the entire term vanished and it's a trival situation) which means $\gr\gs=-\gs\gr$ (Of course we can compute the $g$ metrice but it would take time and this shall do the trick.)
\begin{align*}
  [\gamma^{\mu},S^{\rho\sigma}]&=\frac{i}{4}[\gm,[\gr,\gs]]\\
  &=\frac{i}{4}(\gm\gr\gs-\gm\gs\gr+\gs\gr\gm-\gr\gs\gm)\\
  &=\frac{i}{4}(2\gmr\gs-\gr\gm\gs-2\gms\gr+\gs\gm\gr+2\gmr\gs-\gs\gm\gr-2\gms\gr+\gr\gm\gs)\\
  &=i(\gmr\gs-\gms\gr)\\&=i(\gmr\delta^{\sigma}_{\nu}-\gms\delta^{\rho}_{\nu})\gn\\&=\Jrsmn\gn
\end{align*}

{\bf3.}\quad Derive the Schr\"odinger-Pauli equation.

For electron in an EM field, Dirac equation can be written as
\begin{align}
  [i\pdv{t}+e\phi-\ba\cdot(\bP+e\bA)-m\beta]\psi=0
\end{align}
where $\bP=-i\nabla$.
Set
\begin{align*}
  \psi=\pmqty{\varphi\\\chi}e^{-imt}
\end{align*}
so that the certain part of electron rest mass can be removed. Then we have
\begin{align*}
  i\pdv{t}\varphi&=\bm{\sigma}\cdot(\bP+e\bA)\chi-e\phi\varphi\\
  i\pdv{t}\chi&=\bm{\sigma}\cdot(\bP+e\bA)\varphi-e\phi\chi-2m\chi\\
\end{align*}
At nonrelativistic limit, we have
\begin{align}\label{2}
  \chi\approx\frac{1}{2m}\bm{\sigma}\cdot(\bP+e\bA)\varphi
\end{align}
where $\chi/\varphi\ll1$. Then
\begin{align*}
  i\pdv{t}\varphi&=\frac{1}{2m}\bqty{\bm{\sigma}\cdot(\bP+e\bA)}^2\varphi-e\phi\varphi
\end{align*}
Using
\begin{align*}
  \bqty{\bm{\sigma}\cdot(\bP+e\bA)}^2&=(\bP+e\bA)^2+i\bm{\sigma}\cdot\bqty{(\bP+e\bA)\times(\bP+e\bA)}\\
  &=(\bP+e\bA)^2+ie\bm{\sigma}\cdot\bqty{\bP\times\bA+\bA\times\bP}\\
  &=(\bP+e\bA)^2+e\bm{\sigma}\cdot(\nabla\times\bA)\\
  &=(\bP+e\bA)^2+e\bm{\sigma}\cdot\bm{B}
\end{align*}
Then \eqref{2} becomes
\begin{align}{\label{1}}
  i\pdv{t}\varphi&=\bqty{\frac{1}{2m}(\bP+e\bA)^2+\frac{e}{2m}\bm{\sigma}\cdot\bm{B}-e\phi}\varphi\\
  &=\bqty{\frac{1}{2m}(\bP+e\bA)^2-\bm{\mu}\cdot\bm{B}-e\phi}\varphi
\end{align}
where $\bm{\mu}=-\frac{e\hbar}{2mc}\bm{\sigma}$ (in our previous calculation $\hbar=c=1$) is the intrinsic magnetic moment of electron. And \eqref{1} is the standard form of the Schr\"odinger-Pauli equation.


\end{document}
