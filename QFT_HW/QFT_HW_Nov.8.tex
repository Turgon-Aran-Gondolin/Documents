%!mode::"Tex:UTF-8"
\PassOptionsToPackage{unicode}{hyperref}
\PassOptionsToPackage{naturalnames}{hyperref}
\documentclass{article}
\usepackage{fullpage}
\usepackage{parskip}
\usepackage{physics}
\usepackage{amsmath}
\usepackage{amssymb}
\usepackage{xcolor}
\usepackage[colorlinks,linkcolor=blue,citecolor=green]{hyperref}
\usepackage{array}
\usepackage{longtable}
\usepackage{multirow}
\usepackage{comment}
\usepackage{graphicx}
\usepackage{cite}
\usepackage{amsfonts}
\usepackage{bm}
\usepackage{slashed}
\usepackage{dsfont}
\usepackage{mathtools}
\usepackage[compat=1.1.0]{tikz-feynman}
\usepackage{simplewick}
%\usepackage{fourier}
%\usepackage{slashbox}
%\usepackage{intent}
\usepackage{mathrsfs}



\newcommand{\ac}[1]{a_{\vb{#1}}}
\newcommand{\ad}[1]{a_{\vb{#1}}^{\dagger}}
\newcommand{\bc}[1]{b_{\vb{#1}}}
\newcommand{\bd}[1]{b_{\vb{#1}}^{\dagger}}
\newcommand{\omp}[1]{\omega_{\vb{#1}}}
\newcommand{\intphead}[1]{\int\frac{\dd^3#1}{(2\pi)^3}}
\newcommand{\del}[2]{\delta_{#1#2}}
\newcommand{\id}{\int\dd^4x}
\newcommand{\lag}{\mathcal{L}}
\renewcommand{\bm}[1]{\boldsymbol{#1}}
\newcommand{\trh}{\Lambda_{\frac{1}{2}}}
\newcommand{\gm}{\gamma^{\mu}}
\newcommand{\gn}{\gamma^{\nu}}
\newcommand{\gs}{\gamma^{\sigma}}
\newcommand{\gr}{\gamma^{\rho}}
\newcommand{\gnr}{g^{\nu\rho}}
\newcommand{\gmr}{g^{\mu\rho}}
\newcommand{\gms}{g^{\mu\sigma}}
\newcommand{\gns}{g^{\nu\sigma}}
\newcommand{\vbp}{\vb{p}}
\newcommand{\g}{\gamma}
\renewcommand{\a}{\alpha}
\renewcommand{\b}{\beta}
\renewcommand{\t}{\theta}
\newcommand{\p}{\phi}
\newcommand{\vp}{\varphi}
\newcommand{\s}{\sigma}
\renewcommand{\G}{\Gamma}
\newcommand{\pars}{\slashed\partial}
\newcommand{\ps}{\slashed p}




\newcommand{\Tscr}{\mathscr{T}}
\newcommand{\Cscr}{\mathscr{C}}


\title{Homework: Quantum Field Theory \#5}
\author{Yingsheng Huang}
\begin{document}
\maketitle
{\bf1.}\quad
Complete table at P\&S P71.

For the first three colomn, they're already done in class so I simply give the results. (We use the shorthand in P\&S: $(-1)^{\mu}=1$ for $\mu=0$ and $(-1)^{\mu}=-1$ for $\mu=1,2,3$.\footnote{And I write $(CPT)^{-1}$ as $CPT^{-1}$ for short.})
$$P\bar\psi\psi P^{-1}=+\bar\psi\psi(t,-\vb{x})$$
$$T\bar\psi\psi T^{-1}=+\bar\psi\psi(-t,\vb{x})$$
$$C\bar\psi\psi C^{-1}=+\bar\psi\psi(t,\vb{x})$$
$$CPT\bar\psi\psi CPT^{-1}=+\bar\psi\psi(-t,-\vb{x})$$
$$P\bar\psi\g^5\psi P^{-1}=-\bar\psi\g^5\psi(t,-\vb{x})$$
$$T\bar\psi\g^5\psi T^{-1}=-\bar\psi\g^5\psi(-t,\vb{x})$$
$$C\bar\psi\g^5\psi C^{-1}=+\bar\psi\g^5\psi(t,\vb{x})$$
$$CPT\bar\psi\g^5\psi CPT^{-1}=+\bar\psi\g^5\psi(-t,-\vb{x})$$
$$P\bar\psi\gm\psi P^{-1}=(-1)^{\mu}\bar\psi\gm\psi(t,-\vb{x})$$
$$T\bar\psi\gm\psi T^{-1}=(-1)^{\mu}\bar\psi\gm\psi(-t,\vb{x})$$
$$C\bar\psi\gm\psi C^{-1}=-\bar\psi\gm\psi(t,\vb{x})$$
$$CPT\bar\psi\gm\psi CPT^{-1}=-\bar\psi\gm\psi(-t,-\vb{x})$$
Now we calculate the rest.

Given
$$P\psi P^{-1}=\eta\g^0\psi(t,-\vb{x})$$
$$P\bar\psi P^{-1}=\eta^*\bar\psi(t,-\vb{x}) \g^0$$
we have
\begin{align*}
  P\bar\psi\gm\g^5\psi P^{-1}&=\abs{\eta}^2\bar\psi\g^0\gm\g^5\g^0\psi\\
  &=-(-1)^{\mu}\bar\psi\gm\g^5\psi
\end{align*}
and
\begin{align*}
  P\bar\psi\s^{\mu\nu}\psi P^{-1}&=\frac{i}{2}\bar\psi\g^0[\gm,\gn]\g^0\psi\\
  &=\frac{i}{2}(-1)^{\mu}(-1)^{\nu}\bar\psi[\g^{\mu},\gn]\psi\\
  &=(-1)^{\mu}(-1)^{\nu}\bar\psi\s^{\mu\nu}\psi
\end{align*}
and similarly
\begin{align*}
  P\bar\psi\partial_{\mu}\psi P^{-1}&=(-1)^{\mu}\bar\psi\partial_{\mu}\psi
\end{align*}
Define
$$\Tscr\equiv i\g^1\g^3$$
and
$$T\psi T^{-1}=\Tscr\psi$$
$$T\bar\psi T^{-1}=\bar\psi\Tscr^{-1}$$
$$\Tscr(\gm)^*\Tscr^{-1}=\g_{\mu}=(-1)^{\mu}\gm$$
$$\Tscr(\g^5)^*\Tscr^{-1}=\g^5$$
$$\Tscr=\Tscr^{-1}=\Tscr^{\dagger}$$
we have
\begin{align*}
  T\bar\psi\gm\g^5\psi T^{-1}&=\bar\psi\Tscr^{-1}(\gm\g^5)^*\Tscr\psi\\
  &=\bar\psi\Tscr^{-1}\gm^*\Tscr^{-1}\Tscr\g^5^*\Tscr\psi\\
  &=\bar\psi\g_{\mu}\g^5\psi\\
  &=(-1)^{\mu}\bar\psi\gm\g^5\psi
\end{align*}
and
\begin{align*}
  T\bar\psi\s^{\mu\nu}\psi T^{-1}&=-\frac{i}{2}T\bar\psi[\gm,\gn]\psi T^{-1}\\
  &=-\frac{i}{2}\bar\psi\Tscr[\gm,\gn]^*\Tscr^{-1}\psi\\
  &=-(-1)^{\mu}(-1)^{\nu}\bar\psi\s^{\mu\nu}\psi
\end{align*}
and
\begin{align*}
  T\bar\psi\partial_{\mu}\psi T^{-1}&=-(-1)^{\mu}\bar\psi\partial_{\mu}\psi
\end{align*}
Define
$$\Cscr\equiv i\g^2\g^0$$
and
$$C\psi C^{-1}=\Cscr\bar\psi^T$$
$$C\bar\psi C^{-1}=\psi^T\Cscr$$
$$\Cscr(\gm)^T\Cscr^{-1}=-\gm$$
$$\Cscr(\g^5)^T\Cscr^{-1}=\g^5$$
$$\Cscr^{\dagger}=\Cscr^{-1}=-\Cscr=\Cscr^T$$
thus
\begin{align*}
  &(\Cscr(\gm)^T\Cscr^{-1})^{\dagger}=-(\gm)^{\dagger}\\
  =&(\Cscr^{-1})^{\dagger}(\gm)^*\Cscr^{\dagger}\\
  =&-\Cscr^{\dagger}(\gm)^*\Cscr^{-1}\\
  =&\Cscr(\gm)^*\Cscr^{-1}=-(\gm)^{\dagger}
\end{align*}
and
\begin{align*}
  \Cscr\g^5\Cscr^{-1}=\g^5
\end{align*}
Then we have
\begin{align*}
  C\bar\psi\gm\g^5\psi C^{-1}&=\psi^T\Cscr\gm\g^5\Cscr\bar\psi^T\\
  &=\psi^T\gm^T\g^5^T\bar\psi^T\\
  &=-(\bar\psi\g^5\gm\psi)^T\\
  &=\bar\psi\gm\g^5\psi
\end{align*}
and
\begin{align*}
  C\bar\psi\s^{\mu\nu}\psi C^{-1}&=\frac{i}{2}\psi^T\Cscr[\gm,\gn]\Cscr\bar\psi^T\\
  &=-\frac{i}{2}\psi^T[\gm^T,\gn^T]\bar\psi^T\\
  &=\frac{i}{2}(\bar\psi[\gn,\gm]\psi)^T\\
  &=-\bar\psi\s^{\mu\nu}\psi
\end{align*}
and
\begin{align*}
  C\bar\psi\partial_{\mu}\psi C^{-1}&=\bar\psi\partial_{\mu}\psi
\end{align*}

CPT is to multiply all those coefficients and too trival to list here.

{\bf2.}\quad
Calculate the Dirac propagator.
$$\mel{0}{\psi_a(x)\bar\psi_b(y)}{0}=\intphead{p}\frac{1}{2E_{\vbp}}\sum_su^s_a(p)\bar u^s_b(p)e^{-ip\cdot(x-y)}=(i\pars_x+m)_{ab}\intphead{p}\frac{1}{2E_{\vbp}}e^{-ip\cdot(x-y)}$$
$$\mel{0}{\bar\psi_b(y)\psi_a(x)}{0}=\intphead{p}\frac{1}{2E_{\vbp}}\sum_sv^s_a(p)\bar v^s_b(p)e^{-ip\cdot(x-y)}=-(i\pars_x+m)_{ab}\intphead{p}\frac{1}{2E_{\vbp}}e^{ip\cdot(x-y)}$$
The definition of Dirac propagator
$$S_F(x-y)=\int\frac{\dd^4p}{(2\pi)^4}\frac{i(\ps+m)}{p^2-m^2+i\epsilon}e^{-ip\cdot(x-y)}$$
the two poles (if $\epsilon=0$) are located in $p^0=\omega_p\equiv\sqrt{\vbp^2+m^2}$ and $p^0=-\omega_p$.

For $x^0-y^0>0$, we have
\begin{align*}
  \int\frac{\dd p^0}{2\pi}\frac{i(\ps+m)}{p^2-m^2+i\epsilon}e^{-ip^0(x^0-y^0)}&=\int\frac{\dd p^0}{2\pi}\frac{i(\ps+m)}{p_0^2-\omega_p^2+i\epsilon}e^{-ip^0(x^0-y^0)}\\
  &=\int\frac{\dd p^0}{2\pi}\frac{i(\ps+m)}{(p_0-\omega_p+i\epsilon)(p_0+\omega_p-i\epsilon)}e^{-ip^0(x^0-y^0)}\\
  &=\frac{1}{2(\omega_p-i\epsilon)}\int\frac{\dd p^0}{2\pi}i(\ps+m)(\frac{1}{p_0-\omega_p+i\epsilon}-\frac{1}{p_0+\omega_p-i\epsilon})e^{-ip^0(x^0-y^0)}\\
  &=\frac{i\pars_x+m}{2(\omega_p-i\epsilon)}\int\frac{\dd p^0}{2\pi}i(\frac{1}{p_0-\omega_p+i\epsilon}-\frac{1}{p_0+\omega_p-i\epsilon})e^{-ip^0(x^0-y^0)}\\
  \intertext{use residue theorem, and the contour is closed below (only one singularity on the right)}
  &=\frac{i\pars_x+m}{2(\omega_p-i\epsilon)}e^{-ip^0(x^0-y^0)}\\
  &=\frac{i\pars_x+m}{2\omega_p}e^{-ip^0(x^0-y^0)}\\
  \intertext{wihch means}
  S_F(x-y)&=\mel{0}{\psi_a(x)\bar\psi_b(y)}{0}
\end{align*}
Similarly, when $x^0-y^0<0$, the contour is closed above and
$$S_F(x-y)=-\frac{i\pars_x+m}{2\omega_p}e^{ip^0(x^0-y^0)}=-\mel{0}{\bar\psi_b(y)\psi_a(x)}{0}$$

\appendix
\subsection*{Appendix}
\begin{center}
  \begin{tabular}{|c|ccccc|}
    \hline
    &$\g^0$&$\g^1$&$\g^2$&$\g^3$&$\g^5$\\\hline
    T&1&-1&1&-1&1\\%\hline
    -1&1&-1&-1&-1&1\\
    *&1&1&-1&1&1\\%\hline
    \dagger&1&-1&-1&-1&1\\\hline
  \end{tabular}
\end{center}


\end{document}
