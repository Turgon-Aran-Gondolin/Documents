%!mode::"Tex:UTF-8"
\PassOptionsToPackage{unicode}{hyperref}
\PassOptionsToPackage{naturalnames}{hyperref}
\documentclass{article}
\usepackage{fullpage}
\usepackage{parskip}
\usepackage{physics}
\usepackage{amsmath}
\usepackage{amssymb}
\usepackage{xcolor}
\usepackage[colorlinks,linkcolor=blue,citecolor=green]{hyperref}
\usepackage{array}
\usepackage{longtable}
\usepackage{multirow}
\usepackage{comment}
\usepackage{graphicx}
\usepackage{cite}
\usepackage{amsfonts}
\usepackage{slashed}
\usepackage{mathtools}
\usepackage[compat=1.1.0]{tikz-feynman}
\usepackage{simplewick}
%\usepackage{bm}
\usepackage{dsfont}
%\usepackage{fourier}% -*- program: Value -*-
%\usepackage{slashbox}
%\usepackage{intent}


\newcommand{\ac}[1]{a_{\vb{#1}}}
\newcommand{\ad}[1]{a_{\vb{#1}}^{\dagger}}
\newcommand{\bc}[1]{b_{\vb{#1}}}
\newcommand{\bd}[1]{b_{\vb{#1}}^{\dagger}}
\newcommand{\omp}[1]{\omega_{\vb{#1}}}
\newcommand{\intphead}[1]{\int\frac{\dd^3#1}{(2\pi)^3}}
\newcommand{\del}[2]{\delta_{#1#2}}
\newcommand{\id}{\int\dd^4x}
\newcommand{\lag}{\mathcal{L}}
\newcommand{\bm}[1]{\boldsymbol{#1}}
\newcommand{\trh}{\Lambda_{\frac{1}{2}}}
\newcommand{\gm}{\gamma^{\mu}}
\newcommand{\gn}{\gamma^{\nu}}
\newcommand{\gs}{\gamma^{\sigma}}
\newcommand{\gr}{\gamma^{\rho}}
\newcommand{\gmn}{g^{\mu\nu}}
\newcommand{\gnm}{g^{\nu\mu}}
\newcommand{\gmm}{g^{\mu\mu}}
\newcommand{\gnn}{g^{\nu\nu}}
\newcommand{\gnr}{g^{\nu\rho}}
\newcommand{\gmr}{g^{\mu\rho}}
\newcommand{\gms}{g^{\mu\sigma}}
\newcommand{\gns}{g^{\nu\sigma}}
\newcommand{\g}{\gamma}
\renewcommand{\a}{\alpha}
\renewcommand{\b}{\beta}
\renewcommand{\t}{\theta}
\newcommand{\p}{\phi}
\newcommand{\vp}{\varphi}
\newcommand{\s}{\sigma}
\renewcommand{\G}{\Gamma}



\title{Homework: Quantum Field Theory}
\author{Yingsheng Huang}
\begin{document}
\maketitle
%\tableofcontents

test

{\bf2.1}\quad(a). The first part of this question was already asked previously, so I'll skip it. The tensor form of Maxwell equations is
\begin{align}
 \begin{cases}
   &\epsilon^{\mu\nu\rho\sigma}\partial_{\nu}F_{\rho\sigma}=0\\
   &\partial_{\nu}F^{\mu\nu}=-\mu_0J^{\mu}
 \end{cases}
\end{align}
where the first equation can be written as
\begin{equation*}
  \partial_{\rho}F_{\mu\nu}+\partial_{\mu}F_{\nu\rho}+\partial_{\nu}F_{\rho\mu}=0
\end{equation*}
Identifying $E^i=-F^{0i}$ and $\epsilon^{ijk}B^k=-F^{ij}$, we can have (where we already know $J^{\mu}={\rho c,J^x,J^y,J^z}$)
\begin{align*}
&\partial{\nu}F^{0\nu}=-\mu_0J^0=-c\mu_0\rho\\
=-&\frac{1}{c}(\div{\vb{E}})
\end{align*}
which leads to
\begin{align}
\div{\vb{E}}=\frac{\rho}{\varepsilon_0}
\end{align}
And that's the first one of the standard form Maxwell equations.

The fourth one (often known as the Ampere's Law) can be derived from this one as well. The spatial part of this equation is
\begin{align*}
\partial_{\nu}F^{i\nu}=-\mu_0J^i
\end{align*}
We have
\begin{align*}
  &\partial_{\nu}F^{i\nu}=-\mu_0J^i\\
  =&\partial_0\frac{E^i}{c}-\varepsilon^{ijk}\partial_jB^k-\varepsilon^{ikj}\partial_kB^j\\
  =&\partial_t\frac{E^i}{c^2}-\curl{\vb{B}}
\end{align*}
Then
\begin{align}
 \curl{\vb{B}}=\mu_0\vb{J}+\mu_0\varepsilon_0\pdv{\vb{E}}{t}
\end{align}

For the rest two equations, we need the first equation in our tensor form Maxwell equations. Given
\begin{equation*}
  \partial_{\rho}F_{\mu\nu}+\partial_{\mu}F_{\nu\rho}+\partial_{\nu}F_{\rho\mu}=0
\end{equation*}
we have
\begin{align*}
&\partial_iF^{0j}+\partial_0F^{ij}+\partial_jF^{i0}=0\\
=&-\partial_iE^j-\varepsilon^{ijk}\partial_0B^k+\partial_jE^i\\
\end{align*}
Then we have
\begin{align}
\curl{\vb{E}}=-\pdv{\vb{B}}{t}
\end{align}
This is the second one of the Maxwell equations (known as the Faraday's Law). To obtain our last equation, we have (let $\rho=1,\mu=2,\nu=3$)
\begin{align*}
&\partial_{1}F_{23}+\partial_{2}F_{31}+\partial_{3}F_{12}=0\\
=&\varepsilon^{231}\partial_1B^1+\varepsilon^{312}\partial_2B^2+\varepsilon^{123}\partial_3B^3\\
=&\div{\vb{B}}
\end{align*}
So we have
\begin{align}
\div{\vb{B}}=0
\end{align}
Now we have all the four Maxwell equation in the standard form. Notice that the question is about a EM field with {\bf{no sources}}, so any terms contains the current $J^{\mu}$ in the equations we get should be eliminated. The rest should be easy so I'll skip it again.

\makebox{\phantom{2.1}}\quad(b). From the definition in the book, we have
\begin{align}
T^{\mu\nu}=\pdv{\mathcal{L}}{(\partial_{\mu}\phi)}\partial_{\nu'}\phi g^{\nu'\nu}-\mathcal{L}g^{\mu\nu}
\end{align}
And write down the Lagrangian density
\begin{align}
\mathcal{L}=-\frac{1}{4}F_{\mu\nu}F^{\mu\nu}
\end{align}
we have the energy-momentum tensor
\begin{align}
T^{\mu\nu}=-F^{\mu\lambda}\partial_{\nu}A^{\lambda}+\frac{1}{4}g^{\mu\nu}F_{\lambda\sigma}F^{\lambda\sigma}
\end{align}
Adding an additional term $K^{\lambda\mu\nu}=F^{\mu\lambda}A^{\nu}$, we have
\begin{align*}
\hat{T}^{\mu\nu}&=T^{\mu\nu}+\partial_{\lambda}K^{\lambda\mu\nu}\\
&=\frac{1}{4}g^{\mu\nu}F_{\lambda\sigma}F^{\lambda\sigma}-F^{\mu\lambda}\partial_{\nu}A^{\lambda}+F^{\mu\lambda}\partial_{\lambda}A^{\nu}+(\partial_{\lambda}F^{\mu\lambda})A^{\nu}
\end{align*}
The last term vanish when equation of motion is applied. Now we have
\begin{align}
  \hat{T}^{\mu\nu}&=\frac{1}{4}g^{\mu\nu}F_{\lambda\sigma}F^{\lambda\sigma}+F^{\mu\lambda}F_{\lambda}^{\nu}
\end{align}
and it's symmatric.

The energy is the $\hat{T}^{00}$ term
\begin{align*}
  \mathcal{E}&=\hat{T}^{00}\\
  &=\frac{1}{4}g^{00}F_{\lambda\sigma}F^{\lambda\sigma}+F^{0\lambda}F_{\lambda}^{0}\\
  &=\frac{1}{4}(F_{0\sigma}F^{0\sigma}+F_{\lambda0}F^{\lambda0}+F_{ij}F^{ij})+F^{0\lambda}F^0_{\lambda}\\
  &=\frac{1}{4}F_{ij}F^{ij}+\frac{1}{2}F_{0i}F^{0i}-F^{0i}F^{0i'}g^{ii'}\\
  &=\frac{1}{4}g_{ii'}g_{jj'}F^{ij}F^{ij}+\frac{1}{2}g_{00}g_{ii'}F^{0i}F^{0i}-F^{0i}F^{0i'}g^{ii'}\\
  &=\frac{1}{4}F^{ij}F^{ij}-\frac{1}{2}F^{0i}F^{0i}+F^{0i}F^{0i'}\\
  &=\frac{1}{2}\abs{\vb{B}}^2+\frac{1}{2}\abs{\vb{E}}^2\\
\end{align*}
which is exactly the standard electromagnetic energy. The momentum density can be derived in the same way, which is given by the $\hat{T}^{0i}$ term
\begin{align*}
   \vb{S}&=\hat{T}^{0i}\\
   &=\frac{1}{4}g^{0i}F_{\lambda\sigma}F^{\lambda\sigma}+F^{0\lambda}F_{\lambda}^{i}\\
   &=F^{0j'}F^{ji}g_{jj'}\\
   &=-(-E^j)(-\epsilon^{jik}B^k)\\
   &=\epsilon{ijk}E^jB^k
\end{align*}
So
\begin{align}
  \vb{S}=\vb{E}\times\vb{B}
\end{align}
{\bf2.2}\quad(a) The action S is
\begin{align}
S=\int\dd^4x (\partial_{\mu}\phi^*\partial^{\mu}\phi-m^2\phi*\phi)
\end{align}
And we can write down the Lagrangian as
\begin{align}
\mathcal{L}=\partial_{\mu}\phi^*\partial^{\mu}\phi-m^2\phi^*\phi
\end{align}
To write down the Hamiltionian we have
\begin{align}
  \pi=\pdv{\mathcal{L}}{\dot{\phi}}=\dot{\phi^*}, \;\;\pi^* =\pdv{\mathcal{L}}{\dot{\phi^*}}=\dot{\phi}
\end{align}
and Hamiltionian density is
\begin{align}
  \nonumber\mathcal{H}&=\pi\dot{\phi}+\pi^*\dot{\phi^*}-\mathcal{L}\\
  \nonumber&=\pi\pi^*+\pi^*\pi-\pi\pi^*+\nabla{\phi^*}\cdot\nabla{\phi}+m^2\phi^*\phi\\
  &=\pi^*\pi+\nabla{\phi^*}\cdot\nabla{\phi}+m^2\phi^*\phi
\end{align}
The commutation relations is
\begin{align}
  [\phi(\vb{x}),\pi(\vb{y})]=[\phi^*(\vb{x}),\pi^*(\vb{y})]=i\delta^3(\vb{x}-\vb{y})
\end{align}
And the equation of motion can be derived by Euler-Lagarange equation
\begin{align*}
  &\partial_{\mu}\pdv{\mathcal{L}}{(\partial_{\mu}\phi)}-\pdv{\mathcal{L}}{\phi}=0\\
  =&\square\phi^*+m^2\phi^*
\end{align*}
and
\begin{align*}
  &\partial_{\mu}\pdv{\mathcal{L}}{(\partial_{\mu}\phi^*)}-\pdv{\mathcal{L}}{\phi^*}=0\\
  =&\square\phi+m^2\phi
\end{align*}
Now we have the typical Klein-Gordon equation
\begin{align}
  &(\square+m^2)\phi=0\\
  &(\square+m^2)\phi^*=0
\end{align}
\makebox{\phantom{2.1}}\quad(b).
Firstly we write down $\phi$, $\phi^*$, $\pi$, $\pi^*$, $\nabla \phi$ and $\nabla\phi^*$ explicitly
\begin{align}
  \phi^*(\vb{x})&=\int\frac{\dd^3p}{(2\pi)^3}\frac{1}{\sqrt{2\omega_{\vb{p}}}}(a_{\vb{p}}^{\dagger}e^{ip\cdot x}+b_{\vb{p}}e^{-ip\cdot x})
  =\int\frac{\dd^3p}{(2\pi)^3}\frac{1}{\sqrt{2\omega_{\vb{p}}}}(a_{\vb{p}}^{\dagger}+b_{-\vb{p}})e^{ip\cdot x}\\
  \phi(\vb{x})&=\int\frac{\dd^3p}{(2\pi)^3}\frac{1}{\sqrt{2\omega_{\vb{p}}}}(a_{\vb{p}}e^{-ip\cdot x}+b_{\vb{p}}^{\dagger}e^{ip\cdot x})
  =\int\frac{\dd^3p}{(2\pi)^3}\frac{1}{\sqrt{2\omega_{\vb{p}}}}(a_{-\vb{p}}+b_{\vb{p}}^{\dagger})e^{ip\cdot x}\\
  \pi^*(\vb{x})&=\int\frac{\dd^3p}{(2\pi)^3}(-i)\sqrt{\frac{\omega_{\vb{p}}}{2}}(a_{\vb{p}}e^{-ip\cdot x}-b_{\vb{p}}^{\dagger}e^{ip\cdot x})
  =\int\frac{\dd^3p}{(2\pi)^3}(-i)\sqrt{\frac{\omega_{\vb{p}}}{2}}(a_{-\vb{p}}-b_{\vb{p}}^{\dagger})e^{ip\cdot x}\\
  \pi(\vb{x})&=\int\frac{\dd^3p}{(2\pi)^3}i\sqrt{\frac{\omega_{\vb{p}}}{2}}(a_{\vb{p}}^{\dagger}e^{ip\cdot x}-b_{\vb{p}}e^{-ip\cdot x})
  =\int\frac{\dd^3p}{(2\pi)^3}i\sqrt{\frac{\omega_{\vb{p}}}{2}}(a_{\vb{p}}^{\dagger}-b_{-\vb{p}})e^{ip\cdot x}\\
  \nabla\phi^*(\vb{x})&=\int\frac{\dd^3p}{(2\pi)^3}(-i)\frac{\vb{p}}{\sqrt{2\omega_{\vb{p}}}}(a_{\vb{p}}^{\dagger}e^{ip\cdot x}-b_{\vb{p}}e^{-ip\cdot x})
  =\int\frac{\dd^3p}{(2\pi)^3}(-i)\frac{\vb{p}}{\sqrt{2\omega_{\vb{p}}}}(a_{\vb{p}}^{\dagger}+b_{-\vb{p}})e^{ip\cdot x}\\
  \nabla\phi(\vb{x})&=\int\frac{\dd^3p}{(2\pi)^3}i\frac{\vb{p}}{\sqrt{2\omega_{\vb{p}}}}(a_{\vb{p}}e^{-ip\cdot x}-b_{\vb{p}}^{\dagger}e^{ip\cdot x})
  =\int\frac{\dd^3p}{(2\pi)^3}(-i)\frac{\vb{p}}{\sqrt{2\omega_{\vb{p}}}}(a_{-\vb{p}}+b_{\vb{p}}^{\dagger})e^{ip\cdot x}
\end{align}
The Hamiltionian is
\begin{align}
  \nonumber H&=\int\dd^3x(\pi^*\pi+\nabla{\phi^*}\cdot\nabla{\phi}+m^2\phi^*\phi)\\
  \nonumber&=\int\frac{\dd^3p}{(2\pi)^3}\frac{1}{2\omega_{\vb{p}}}[\omega_{\vb{p}}^2(\ac{-p}\ad{-p}-\ac{-p}\bc{p}-\bd{p}\ad{-p}+\bd{p}\bc{p})+\abs{\vb{p}}^2(\ad{p}\ac{p}+\ad{p}\bd{-p}+\bc{-p}\ac{p}+\bc{-p}\bd{-p})\\\nonumber&\phantom{=}+m^2(\ad{p}\ac{p}+\ad{p}\bd{-p}+\bc{-p}\ac{p}+\bc{-p}\bd{-p})]\\
  \nonumber&=\int\frac{\dd^3p}{(2\pi)^3}\frac{1}{2\omega_{\vb{p}}}[\omega_{\vb{p}}^2(\ac{p}\ad{p}+\bd{p}\bc{p})+(\abs{\vb{p}}^2+m^2)(\ad{p}\ac{p}+\bc{p}\bd{p})-(\omega_{\vb{p}}^2-\abs{\vb{p}}^2-m^2)(\ac{-p}\bc{p}+\bd{p}\ad{-p})]\\
  \nonumber&=\int\frac{\dd^3p}{(2\pi)^3}\frac{\omp{p}}{2}(\ac{p}\ad{p}+\bd{p}\bc{p}+\ad{p}\ac{p}+\bc{p}\bd{p})\\
  &=\int\frac{\dd^3p}{(2\pi)^3}\omp{p}(\ad{p}\ac{p}+\bd{p}\bc{p}+\frac{1}{2}[\ac{p},\ad{p}]+\frac{1}{2}[\bc{p},\bd{p}])
\end{align}
Thus we can see it contains two sets of particle with mass m.

\makebox{\phantom{2.1}}\quad(c).
The conserved charge is
\begin{align}
  Q=\int\dd^3x\frac{i}{2}(\phi^*\pi^*-\pi\phi)
\end{align}
Expand it and we can get
\begin{align}
  \nonumber Q&=\int\dd^3x\frac{i}{2}(\phi^*\pi^*-\pi\phi)\\
  \nonumber &=\frac{i}{2}\int\frac{\dd^3p}{(2\pi)^3}(-\frac{i}{2})[(\ad{p}+\bc{-p})(\ac{p}-\bd{-p})+(\ad{p}-\bc{-p})(\ac{p}+\bd{-p})]\\
  \nonumber&=\intphead{p}\frac{1}{4}(\ad{p}\ac{p}-\ad{p}\bd{-p}+\bc{-p}\ac{p}-\bc{p}\bd{p}+\ad{p}\ac{p}+\ad{p}\bd{-p}-\bc{-p}\ac{p}-\bc{p}\bd{p})\\
  &=\intphead{p}\frac{1}{2}(\ad{p}\ac{p}-\bc{p}\bd{p})
\end{align}
Apparently, $a$ and $b$ stand for different charges of two particles.

\makebox{\phantom{2.1}}\quad(d). Assuming we have two complex Klein-Gordon field, written as $\phi_a(x)$, where $a=1, 2$. There're now 4 conserved charges, one is the same as $Q$ in (c), the other three is given by
\begin{align}
  Q^i=\int\dd^3x\frac{i}{2}(\phi_a^*(\sigma^i)_{ab}\pi_b^*-\pi_a(\sigma^i)_{ab}\phi_b)
\end{align}
where $\sigma ^i$ is the Pauli matrice. We can prove it using Noether's theorem.

First, the new Lagrangian would be
\begin{align}
  \lag=\sum\limits_m(\partial_{\mu}\phi^*_m\partial^{\mu}\phi_m-m^2\phi^*_m\phi_m)
\end{align}


We know that the Lagrangian is invariant under $U(N)$ and $SU(N)$ group, then the four generator can be given. The generator of $U(1)$ group is corresponding to the conserved charge $Q$ given above in (c). The generators of $SU(2)$ provides the transformations for the rest of conserved charges
\begin{align*}
  \phi_a&\rightarrow \phi_a-i\alpha (J^k)_{ab}\phi_b\\
  \phi_b^*&\rightarrow\phi_b^*+i\alpha\phi_a^*(J^k)_{ab}
\end{align*}
and by Noether's theorem we can have
\begin{align*}
  j^{\mu}&=\sum\limits_m\pdv{\lag}{(\partial_{\mu}\phi_m)}\Delta\phi_m\\
  &=-\partial^{\mu}\phi^*_mi (J^k)_{mn}\phi_n+ i\phi_n^*(J^k)_{nm}\partial_{\mu}\phi_m
\end{align*}
The conserved charge is
\begin{align*}
  Q^k&=\int\dd^3xj^0\\
  &=i\int\dd^3x(\phi_a^*(\sigma^k)_{ab}\pi^*_b-\pi_a  (\sigma^k)_{ab}\phi_b)
\end{align*}
Now we have all four conserved charges.

The commutation relation then is
\begin{align*}
  [Q^i,Q^j]&=-\frac{1}{4}\int\dd^3x\dd^3y[\phi_a^*(\sigma^i)_{ab}\pi_b^*-\pi_a(\sigma^i)_{ab}\phi_b,\phi_c^*(\sigma^j)_{cd}\pi_d^*-\pi_c(\sigma^j)_{cd}\phi_d]\\
  &=-\frac{1}{4}\int\dd^3x\dd^3y([\phi_a^*(\sigma^i)_{ab}\pi_b^*,\phi_c^*(\sigma^j)_{cd}\pi_d^*]-[\phi_a^*(\sigma^i)_{ab}\pi_b^*,\pi_c(\sigma^j)_{cd}\phi_d]\\&\phantom{=}-[\pi_a(\sigma^i)_{ab}\phi_b,\phi_c^*(\sigma^j)_{cd}\pi_d^*]+[\pi_a(\sigma^i)_{ab}\phi_b,\pi_c(\sigma^j)_{cd}\phi_d])
\end{align*}
We view it by each and every terms. Note that $(\sigma ^i)_{ab}$ is a number, and here only $[\phi^*,\pi^*]$ and $[\phi,\pi]$ are non-zero, so we have
\begin{align*}
  &[\phi_a^*(\sigma^i)_{ab}\pi_b^*,\phi_c^*(\sigma^j)_{cd}\pi_d^*]=\phi_a^*(\sigma^i)_{ab}[\pi_b^*,\phi_c^*](\sigma^j)_{cd}\pi_d^*+\phi^*_c(\sigma^j)_{cd}[\phi_a^*,\pi_d^*](\sigma^i)_{ab}\pi_b^*
\end{align*}
and
\begin{align*}
  &[\pi_a(\sigma^i)_{ab}\phi_b,\pi_c(\sigma^j)_{cd}\phi_d]=\pi_c(\sigma^j)_{cd}[\pi_a,\phi_d](\sigma^i)_{ab}\phi_b+\pi_a(\sigma^i)_{ab}[\phi_b,\pi_c](\sigma^j)_{cd}\phi_d
\end{align*}
Then
\begin{align}
  \nonumber [Q^i,Q^j]=&-\frac{1}{4}\int\dd^3x\dd^3y(\phi_a^*(\sigma^i)_{ab}[\pi_b^*,\phi_c^*](\sigma^j)_{cd}\pi_d^*+\phi^*_c(\sigma^j)_{cd}[\phi_a^*,\pi_d^*](\sigma^i)_{ab}\pi_b^*\\\nonumber&+\pi_c(\sigma^j)_{cd}[\pi_a,\phi_d](\sigma^i)_{ab}\phi_b+\pi_a(\sigma^i)_{ab}[\phi_b,\pi_c](\sigma^j)_{cd}\phi_d)\\
  \nonumber=&\frac{1}{4}\int\dd^3x\dd^3yi\delta^3(x-y)\{\phi_a^*(\sigma^i)_{ab}\del{c}{b}(\sigma^j)_{cd}\pi_d^*-\phi^*_c(\sigma^j)_{cd}\del{a}{d}(\sigma^i)_{ab}\pi_b^*\\\nonumber&+\pi_c(\sigma^j)_{cd}\del{a}{d}(\sigma^i)_{ab}\phi_b-\pi_a(\sigma^i)_{ab}\del{b}{c}(\sigma^j)_{cd}\phi_d)\}
\end{align}
From the commutation relations of angular momentum we have
\begin{align*}
  (\sigma^i)_{ab}(\sigma^j)_{bd}-(\sigma^j)_{ab}(\sigma^i)_{bd}=2\epsilon^{ijk}(\sigma^k)_{ad}
\end{align*}
And
\begin{align}
  \nonumber [Q^i,Q^j]=&\frac{1}{4}\int\dd^3xi\{\phi_a^*(\sigma^i)_{ab}(\sigma^j)_{bd}\pi_d^*-\phi^*_c(\sigma^i)_{ab}(\sigma^j)_{ca}\pi_b^*+\pi_c(\sigma^j)_{ca}(\sigma^i)_{ab}\phi_b-\pi_a(\sigma^j)_{bd}(\sigma^i)_{ab}\phi_d)\}\\
  \nonumber=&\frac{1}{4}\int\dd^3xi\{\phi_a^*(\sigma^i)_{ab}(\sigma^j)_{bd}\pi_d^*-\phi^*_a(\sigma^j)_{ab}(\sigma^i)_{bd}\pi_d^*+\pi_a(\sigma^j)_{ab}(\sigma^i)_{bd}\phi_d-\pi_a(\sigma^i)_{ab}(\sigma^j)_{bd}\phi_d)\}\\
  \nonumber=&\frac{i}{2}\int\dd^3x\{\phi^*_a\epsilon^{ijk}(\sigma^k)_{ab}\pi_b^*-\pi_a\epsilon^{ijk}(\sigma^k)_{ab}\phi_b\}\\
  =&i\epsilon^{ijk}Q^k
\end{align}
Obviously $[Q^i,Q^i]=0$ so I'll skip the discussion about it.
If we have n identical complex scalar fields, then $Q$ should fit the commutation relations for SU(N) group (or just to say SU(N) or SO(2N) group Lie algebra). There will be $n(2n-1)$ conserved charges.

{\bf2.3}\quad The correlation function
\begin{align}
  \mel{0}{\phi(x)\phi(y)}{0}=D(x-y)=\intphead{p}\frac{1}{2E_{\vb{p}}}e^{-ip\cdot(x-y)}
\end{align}
If $(x-y)$ is spacelike, we have $(x-y)^2=-r^2$. Then the correlation function is (I'll skip some details but the process is easy)
\begin{align}
  D(x-y)&=\frac{1}{(2\pi)^2}\int^{\infty}_{0}\dd p\frac{p^2}{2E_{\vb{p}}}\frac{e^{ipr}-e^{-ipr}}{ipr}\\
  &=\frac{1}{8\pi^2r}\int^{\infty}_0\dd p\frac{psin(pr)}{\sqrt{p^2+m^2}}
\end{align}
Using the known relation (can be found in mathworld)
\begin{align}
  K_0(z)=\int_0^{\infty}\frac{cos(zt)}{\sqrt{t^2+1}}\dd t
\end{align}
we can derive
\begin{align*}
  K_0'(mr)=-\frac{1}{m}\int^{\infty}_{0}\dd p\frac{psin(pr)}{\sqrt{p^2+m^2}}
\end{align*}
So
\begin{align}
  D(x-y)=-\frac{m}{8\pi^2r}K_0'(mr)
\end{align}
And from another known relation of $K_1$ we can derive
\begin{align*}
  K_1(z)&=\frac{\Gamma(\frac{1}{2})\cdot2z}{\sqrt{\pi}}\int^{\infty}_0\frac{cos(t)\dd t}{(t^2+z^2)^{\frac{3}{2}}}\\
  &=-2\int^{\infty}_0\dv{z}\frac{cos(t)\dd t}{(t^2+z^2)^{\frac{1}{2}}}\\
  &=-2\dv{z}K_0(z)
\end{align*}
Then we have our final result
\begin{align}
  D(x-y)=\frac{m}{4\pi^2r}K_1(mr)
\end{align}

{\bf3.1}\quad(a). First we have the commutation relations in the book
\begin{align}
  [J^{\mu\nu},J^{\rho\sigma}]=i(g^{\nu\rho}J^{\mu\sigma}-g^{\mu\rho}J^{\nu\sigma}-g^{\nu\sigma}J^{\mu\rho}+g^{\mu\sigma}J^{\nu\rho})
\end{align}
Define the generators of rotations and boosts as
\begin{align}
  L^i=\frac{1}{2}\epsilon^{ijk}J^{jk}, \;\;K^i=J^{0i}
\end{align}
The commutation relations of the combination
\begin{align}
  \bm{J}_+=\frac{1}{2}(\bm{L}+i\bm{K}),\;\;\bm{J}_-=\frac{1}{2}(\bm{L}-i\bm{K})
\end{align}
is
\begin{align*}
  [\bm{J}_+,\bm{J}_-]=\frac{1}{4}\qty{[\bm{L},\bm{L}]-i[\bm{L},\bm{K}]+i[\bm{K},\bm{L}]-[\bm{K},\bm{K}]}
\end{align*}
Apparently $[\bm{L},\bm{L}]=0, [\bm{K},\bm{K}]=0$, and $[\bm{L},\bm{K}]=-[\bm{K},\bm{L}]$. Since (in Minkowski spacetime)
\begin{align*}
  [L^i,K^i]&=\frac{1}{2}\epsilon^{ijk}[J^{jk},J^{0i}]\\
  &=\frac{1}{2}\epsilon^{ijk}i\qty{g^{ji}J^{k0}-g^{ki}J^{j0}}\\
  &=0
\end{align*}
So $[J_+,J_-]=0$.
Now we evaluate the separate commutation relations between $J_+$ and $J_-$.
\begin{align*}
  [J_+^i,J_+^j]&=\frac{1}{4}\qty{[L^i,L^j]+i[L^i,K^j]+i[K^i,L^j]-[K^i,K^j]}
\end{align*}
Let's evaluate those terms one by one:

The first term
\begin{align*}
  [L^i,L^j]&=\frac{1}{4}\epsilon^{ijk}\epsilon^{jk'i'}[J^{jk},J^{k'i'}]\\
  &=\frac{i}{4}\epsilon^{ijk}\epsilon^{jk'i'}(g^{kk'}J^{ji}-g^{ki'}J^{jk'})\\
  &=-\frac{i}{4}\epsilon^{ijk}(\epsilon^{jk'i'}\delta^{kk'}J^{ji}-\epsilon^{jk'i'}\delta^{ki'}J^{jk'})\\
  &=-\frac{i}{4}\epsilon^{ijk}(\epsilon^{jki}J^{ji}-\epsilon^{jik}J^{ji})\\
  &=i\epsilon^{ijk} L^k
\end{align*}
where $i',k'=i,k$.
The second term
\begin{align*}
  [L^i,K^j']&=\frac{1}{2}\epsilon^{ijk}[J^{jk},J^{0j'}]\\
  &=\frac{i}{2}\epsilon^{ijk}[g^{jj'}J^{k0}-g^{kj'}J^{j0}]\\
  &=\frac{i}{2}\epsilon^{ij'k}J^{0k}+\frac{i}{2}\epsilon^{ij'j}J^{0j}\\
  &=i\epsilon^{ij'k} K^{k}
\end{align*}
The third term is similar
\begin{align*}
  [K^i,L^j]=i\epsilon^{ijk}K^k
\end{align*}
The forth term is
\begin{align*}
  [K^i,K^j]&=[J^{0i},J^{0j}]\\
  &=-i(g^{00}J^{ij}+g^{ij}J^{00})\\
  &=-iJ^{ij}\\
  &=-i\epsilon^{ijk}L^k
\end{align*}
And we have
\begin{align*}
  [J^i_+,J^j_+]&=\frac{1}{4}\qty{i\epsilon^{ijk}L^k-\epsilon^{ijk}K^{k}-\epsilon^{ijk}K^{k}+i\epsilon^{ijk}L^k}\\
  &=i\epsilon^{ijk} J^k_+
\end{align*}
With similar process we can have the commutation relation of $J_-$ which is almost the same
\begin{align*}
  [J^i_-,J^j_-]=i\epsilon^{ijk}J^k_-
\end{align*}

\makebox{\phantom{\bf{3.1}}}\quad(b). For $(\frac{1}{2},0)$ representation, we have
\begin{align*}
  \bm{J}_+=\frac{1}{2}\vb{\bm{\sigma}},\;\;\bm{J}_-=0
\end{align*}
from which we can derive
\begin{align*}
  \bm{L}=\frac{1}{2}\vb{\bm{\sigma}},\;\;\bm{K}=-\frac{i}{2}\vb{\bm{\sigma}}
\end{align*}
We know that the transformation law is
\begin{align*}
  \phi_{(\frac{1}{2},0)}&\rightarrow(1-i\bm{\theta}\cdot \bm{L}-i\bm{\beta}\cdot \bm{K})\phi_{(\frac{1}{2},0)}\\
  &=(1-i\bm{\theta}\cdot \frac{\bm{\sigma}}{2}-\bm{\beta}\cdot \frac{\bm{\sigma}}{2})\phi_{(\frac{1}{2},0)}
\end{align*}
And for $(0,\frac{1}{2})$ representation, we have
\begin{align*}
  \bm{J}_+=0,\;\;\bm{J}_-=\frac{1}{2}\vb{\bm{\sigma}}
\end{align*}
Similarly,the transformation law is
\begin{align*}
  \phi_{(0,\frac{1}{2})}&\rightarrow(1-i\bm{\theta}\cdot \bm{L}-i\bm{\beta}\cdot \bm{K})\phi_{(0,\frac{1}{2})}\\
    &=(1-i\bm{\theta}\cdot \frac{\bm{\sigma}}{2}+\bm{\beta}\cdot \frac{\bm{\sigma}}{2})\phi_{(0,\frac{1}{2})}
\end{align*}
From these we can conclude
\begin{align*}
  \phi_{(\frac{1}{2},0)}=\psi_L,\;\;\phi_{(0,\frac{1}{2})}=\psi_R
\end{align*}

\makebox{\phantom{\bf{3.1}}}\quad(c). The identity $\bm{\sigma}^T=-\sigma^2\boldsymbol{\sigma}\sigma^2$ allows us to rewrite the $\psi_L$ transformation in the unitarily equivalent form
\begin{align}
  \psi'\rightarrow\psi'(1+i\bm{\theta}\cdot \frac{\bm{\sigma}}{2}+\bm{\beta}\cdot\frac{\bm{\sigma}}{2})
\end{align}
where $\psi'=\psi_L^T\sigma^2$.

The parametrized matrix can be
\begin{align*}
  \begin{pmatrix}
    V^0+V^3&V^1-iV^2\\
    V^1+iV^2&V^0-V^3
  \end{pmatrix}
  =V^{\mu}\sigma_{\mu}=V^0+V^i\sigma_i
\end{align*}
where $\sigma_0=I$, $\sigma_i$ is the standard Pauli matrix in i direction.

Using this, we have our now transformation law (higher order discarded)
\begin{align*}
  V^{\mu}\sigma_{\mu}&\rightarrow(1-i\bm{\theta}\cdot \frac{\bm{\sigma}}{2}+\bm{\beta}\cdot \frac{\bm{\sigma}}{2})(V^0+V^i\sigma_i)(1+i\bm{\theta}\cdot \frac{\bm{\sigma}}{2}+\bm{\beta}\cdot\frac{\bm{\sigma}}{2})\\
  &=V^0(1+\bm{\beta}\cdot\bm{\sigma})+V^i(\sigma_i+\frac{1}{2}\beta^j\qty{\sigma^i,\sigma^j}+\frac{i}{2}\theta^j[\sigma_i,\sigma_j])\\
  &=V^0(1+\beta^i\sigma_i)+V^i(\sigma_i-\epsilon^{ijk}\theta^j\sigma_k+\beta^i)
\end{align*}
So
\begin{align*}
  &V^0\rightarrow V^0+V^i\beta^i\\
  &V^i\sigma_i\rightarrow V^i(\sigma_i-\epsilon^{ijk}\theta^j\sigma_k)+V^0\beta^i\sigma_i\\
  \text{which means}\;\;\;\;\;\;&V^i\rightarrow V^i+\epsilon^{ijk}\theta^jV^k+\beta^iV^0
\end{align*}
This is exactly how a 4-vector transform.

{\bf{3.2}}\quad Derive the \emph{Gordon identity}
\begin{align}\label{40}
  \bar{u}(p')\gamma^{\mu}u(p)=\bar{u}(p')\bqty{\frac{p'^{\mu}+p^{\mu}}{2m}+\frac{i\sigma^{\mu\nu}q_{\nu}}{2m}}u(p)
\end{align}
where $q=(p'-p)$.

From the standard covariant form of Dirac equation
\begin{align*}
  (i\gamma^{\mu}\partial_{\mu}-m)\psi(x)=0
\end{align*}
and can be written as
\begin{align}
  \gamma^{\mu}p_{\mu}u(p)=m u(p)
\end{align}
From previous definition
\begin{align*}
  \bar{u}(p)\equiv u^{\dagger}(p)\gamma^0
\end{align*}
and
\begin{align*}
  u^{\dagger}(p)p_{\mu}^{\dagger}(\gamma^{\mu})^{\dagger}=m u^{\dagger}(p)
\end{align*}
So we have
\begin{align*}
  \bar{u}(p)\gamma^0p_{\mu}^{\dagger}(\gamma^{\mu})^{\dagger}\gamma^0=m \bar{u}(p)
\end{align*}
Then
\begin{align*}
  \bar{u}(p')\gamma^{\mu}u(p)&=\frac{\bar{u}(p')\gamma^0p'_{\mu'}^{\dagger}(\gamma^{\mu'})^{\dagger}\gamma^0}{m}\gamma^{\mu}\frac{\gamma^{\mu''}p_{\mu''}u(p)}{m}\\
  &=\bar{u}(p')\frac{\gamma^0p'_{\mu'}^{\dagger}(\gamma^{\mu'})^{\dagger}\gamma^0\gamma^{\mu}\gamma^{\mu''}p_{\mu''}}{m^2}u(p)
\end{align*}
Note that $p_{\mu}$ and $\gamma$ commute, and
\begin{align*}
  \gamma^0(\gamma^{\mu})^{\dagger}\gamma^0&=\Pmqty{0&1\\1&0}\Pmqty{0&\sigma^{\mu}\\-\sigma^{\mu}&0}^{\dagger}\Pmqty{0&1\\1&0}\\
  &=\Pmqty{0&\sigma^{\mu}\\-\sigma^{\mu}&0}\\&=
  \gamma^{\mu}
\end{align*}
which means
\begin{align*}
  \bar{u}(p)\gamma^{\mu}p_{\mu}=m \bar{u}(p)
\end{align*}
and
\begin{align*}
  \bar{u}(p')\gamma^{\mu}u(p)&=\bar{u}(p')\frac{\gamma^{\nu}p'_{\nu}\gamma^{\mu}\gamma^{\nu}p_{\nu}}{m^2}u(p)
\end{align*}

Now we observe
\begin{align*}
  i\sigma^{\mu\nu}q_{\nu}&=-\frac{1}{2}[\gamma^{\mu},\gamma^{\nu}](p'_{\nu}-p_{\nu})\\
  &=-\frac{1}{2}(\gamma^{\mu}\gamma^{\nu}p'_{\nu}-\gamma^{\nu}\gamma^{\mu}p'_{\nu}-\gamma^{\mu}\gamma^{\nu}p_{\nu}+\gamma^{\nu}\gamma^{\mu}p_{\nu})
\end{align*}
and
\begin{align*}
  \gamma^{\mu}\gamma^{\nu}=-\gamma^{\nu}\gamma^{\mu}+2g^{\mu\nu}
\end{align*}
We have
\begin{align*}
  i\sigma^{\mu\nu}q_{\nu}&=-\frac{1}{2}(2\gamma^{\mu}\gamma^{\nu}p'_{\nu}-2g^{\mu\nu}p'_{\nu}-2\gamma^{\mu}\gamma^{\nu}p_{\nu}+2g^{\mu\nu}p_{\nu})\\&=(p'^{\mu}-p^{\mu})-\gamma^{\mu}\gamma^{\nu}(p'_{\nu}-p_{\nu})
\end{align*}
With this \eqref{40} becomes
\begin{align*}
  \bar{u}(p')\gamma^{\mu}u(p)&=\bar{u}(p')\bqty{\frac{p'^{\mu}+p^{\mu}}{2m}+\frac{(p'^{\mu}-p^{\mu})-\gamma^{\mu}\gamma^{\nu}(p'_{\nu}-p_{\nu})}{2m}}u(p)\\
  &=\bar{u}(p')\bqty{\frac{p'^{\mu}}{m}-\frac{\gamma^{\mu}\gamma^{\nu}(p'_{\nu}-p_{\nu})}{2m}}u(p)\\
  &=\bar{u}(p')\bqty{\frac{p'^{\mu}}{m}-\frac{\gamma^{\mu}\gamma^{\nu}(p'_{\nu}-p_{\nu})}{2m}}u(p)
  %\\  &=\bar{u}(p')\bqty{\frac{\gamma^{\mu}\gamma^{\nu}(p'_{\nu}-p_{\nu})}{4 m}}u(p)
\end{align*}
We know that
\begin{align*}
  \bar{u}(p')\frac{\gamma^{\nu}p'_{\nu}\gamma^{\mu}\gamma^{\nu}p_{\nu}}{m^2}u(p)&=\frac{1}{2}\Bqty{\bar{u}(p')\frac{-\gamma^{\nu}p'_{\nu}\gamma^{\nu}\gamma^{\mu}p_{\nu}+2\gamma^{\nu}p'_{\nu}g^{\mu\nu}p_{\nu}-\gamma^{\mu}p'_{\nu}\gamma^{\nu}\gamma^{\nu}p_{\nu}+2p'_{\nu}g^{\mu\nu}\gamma^{\nu}p_{\nu}}{m^2}u(p)}\\
  &=\frac{1}{2}\Bqty{\bar{u}(p')\frac{-m\gamma^{\nu}\gamma^{\mu}p_{\nu}+2\gamma^{\nu}p'_{\nu}g^{\mu\nu}p_{\nu}-\gamma^{\mu}p'_{\nu}\gamma^{\nu}m+2p'_{\nu}g^{\mu\nu}\gamma^{\nu}p_{\nu}}{m^2}u(p)}\\
  &=\bar{u}(p')\bqty{\frac{p'^{\mu}+p^{\mu}}{m}-\frac{\gamma^{\nu}\gamma^{\mu}p_{\nu}+\gamma^{\mu}p'_{\nu}\gamma^{\nu}}{2m}}u(p)\\
  &=\bar{u}(p')\bqty{\frac{p'^{\mu}}{m}-\frac{-\gamma^{\mu}\gamma^{\nu}p'_{\nu}+\gamma^{\mu}p'_{\nu}\gamma^{\nu}}{2m}}u(p)\\
  &=\bar{u}(p')\bqty{\frac{p'^{\mu}}{m}-\frac{\gamma^{\mu}\gamma^{\nu}(p'_{\nu}-p_{\nu})}{2m}}u(p)
\end{align*}
And it consists with the former one.


{\bf3.4\quad Majorana fermions}

(a). Equation for $\chi$ as a massive field
\begin{align}
  i\bar{\sigma}\cdot\partial{\chi}-im\sigma^2\chi^*=0
\end{align}
The transformation laws would be
\begin{align*}
  \chi'(x)=\trh\chi(\Lambda^{-1}x)
\end{align*}
where $\trh=e^{-\frac{i}{2}(\theta-i\beta)\cdot\sigma}$.

Now the transformation
\begin{align*}
  i\bar{\sigma}\cdot\partial{\chi}-im\sigma^2\chi^*&\rightarrow i\bar{\sigma}^{\mu}\cdot(\Lambda^{-1})^{\nu}_{\mu}\partial_{\nu}\trh{\chi}(\Lambda^{-1}x)-im\sigma^2\trh\chi^*(\Lambda^{-1}x)
\end{align*}
Using the relation
\begin{align*}
  \trh^{-1}\gamma^{\mu}\trh=\Lambda^{\mu}_{\nu}\gamma^{\nu}
\end{align*}
and the transformation relation becomes
\begin{align*}
  i\bar{\sigma}\cdot\partial{\chi}-im\sigma^2\chi^*&\rightarrow i\trh\Lambda^{\mu}_{\rho}\bar{\sigma}^{\rho}\cdot(\Lambda^{-1})^{\nu}_{\mu}\partial_{\nu}{\chi}(\Lambda^{-1}x)-im\sigma^2\trh\chi^*(\Lambda^{-1}x)\\
  &=\trh[i\bar{\sigma}^{\nu}\cdot\partial_{\nu}{\chi}-im\sigma^2\chi^*](\Lambda^{-1}x)\\
  &=0
\end{align*}
We can draw the conclusion that our equation for $\chi$ is relativistically invariant.

Then we must prove that this equation implies the Klein-Gordon equation.
We can rewrite our equation and its complex-conjugate form as
\begin{align*}
  \bar{\sigma}^{\mu}\cdot\partial_{\mu}{\chi}=m\sigma^2\chi^*\\
  \bar{\sigma}^{*\mu}\cdot\partial_{\mu}{\chi^*}=m\sigma^2\chi
\end{align*}
Solve this set of equations and we have
\begin{align*}
  \chi^*=\frac{\bar{\sigma}^{\mu}\cdot\partial_{\mu}{\chi}}{m\sigma^2}
\end{align*}
which leads to
\begin{align*}
  \bar{\sigma}^{*\mu}\cdot\partial_{\mu}\frac{\bar{\sigma}^{\mu}\cdot\partial_{\mu}{\chi}}{m\sigma^2}=m\sigma^2\chi
\end{align*}
and it means
\begin{align*}
  -\partial_{\mu}\partial^{\mu}\chi=m^2\chi
\end{align*}
which is exactly the Klein-Gordon equation we want.

(b). The classical action is
\begin{align}
  S=\int\dd^4x\bqty{\chi^{\dagger}i\bar{\sigma}^{\mu}\partial_{\mu}\chi+\frac{im}{2}(\chi^T\sigma^2\chi-\chi^{\dagger}\sigma^2\chi^*)}
\end{align}
(where $\chi^{\dagger}=(\chi^*)^T$ and recall that for Grassmann numbers $(\alpha\beta)^*=\beta^*\alpha^*$) and its complex-conjugate is

(First
\begin{align*}
  \chi^{\dagger}i\bar{\sigma}^{\mu}\partial_{\mu}\chi=\chi^*_ai\bar{\sigma^{\mu}}_{ab}\partial_{\mu}\chi_b
\end{align*}
and
\begin{align*}
  (\chi^*_ai\bar{\sigma^{\mu}}_{ab}\partial_{\mu}\chi_b)^*=\partial_{\mu}\chi_b^*i\bar{\sigma^{\mu}}_{ba}\chi_a
\end{align*}
)
\begin{align*}
  S^*&=-\int\dd^4x\bqty{\partial_{\mu}\chi^{\dagger}i(\bar{\sigma}^{\mu})^{\dagger}\chi+\frac{im}{2}(\chi^{\dagger}(\sigma^2)^{\dagger}\chi^*-\chi^T(\sigma^2)^{\dagger}\chi)}\\
  &=\int\dd^4x\bqty{\chi^{\dagger}i\bar{\sigma}^{\mu}\partial_{\mu}\chi+\frac{im}{2}(\chi^T\sigma^2\chi-\chi^{\dagger}\sigma^2\chi^*)-\partial_{\mu}(\chi^{\dagger}i\bar{\sigma}^{\mu}\chi)}\\
  &=S
\end{align*}

To derive the equation of motion out of this action
\begin{align*}
  \delta S&=\delta\int\dd^4x\bqty{\chi^*_ai\bar{\sigma^{\mu}}_{ab}\partial_{\mu}\chi_b+\frac{im}{2}(\chi_a\sigma^2_{ab}\chi_b-\chi^*_a\sigma^2_{ab}\chi^*_b)}=0
\end{align*}
and
\begin{align*}
  \pdv{S}{\chi_1^*}=i\bar{\sigma^{\mu}}_{12}\partial_{\mu}\chi_2-\frac{im}{2}\sigma^2_{12}\chi_2^*\\
  \pdv{S}{\chi_2^*}=i\bar{\sigma^{\mu}}_{21}\partial_{\mu}\chi_1-\frac{im}{2}\sigma^2_{21}\chi_1^*
\end{align*}
and
\begin{align*}
  \pdv{S}{(\partial_{\mu}\chi_a)}=\chi^*_bi\bar{\sigma^{\mu}}_{ba}
\end{align*}
and
\begin{align*}
  \pdv{S}{\chi_a}=\frac{im}{2}\sigma^2_{ab}\chi{b}
\end{align*}
Combine these and we have the Majorana equation above.

(c). We already knew that
\begin{align*}
  \psi_L(x)=\chi_1(x),\;\;\;\;\psi_R(x)=i\sigma^2\chi^*_2(x)
\end{align*}
The Dirac Lagrangian is
\begin{align}
  \lag=\bar{\psi}(i\gamma^{\mu}\partial_{\mu}-m)\psi
\end{align}
where $\bar{\psi}\equiv\psi^{\dagger}\gamma^0$. It can be written as
\begin{align*}
  \lag&=\Pmqty{\psi_L^{\dagger}&\psi_R^{\dagger}}\Pmqty{0&1\\1&0}\pqty{i\Pmqty{0&\sigma^{\mu}\\\bar{\sigma}^{\mu}&0}\partial_{\mu}-m}\Pmqty{\psi_L\\\psi_R}\\
  &=\Pmqty{\psi_R^{\dagger}&\psi_L^{\dagger}}\pqty{i\Pmqty{0&\sigma^{\mu}\\\bar{\sigma}^{\mu}&0}\partial_{\mu}-m\Pmqty{1&0\\0&1}}\Pmqty{\psi_L\\\psi_R}\\
  &=i\psi_L^{\dagger}\bar{\sigma}^{\mu}\partial_{\mu}\psi_L+i\psi_R^{\dagger}\sigma^{\mu}\partial_{\mu}\psi_R-m\psi_R^{\dagger}\psi_L-m\psi_L^{\dagger}\psi_R\\
  &=i\chi_1^{\dagger}\bar{\sigma}^{\mu}\partial_{\mu}\chi_1+\sigma^2\chi_2^T\sigma^{\mu}\partial_{\mu}(i\sigma^2\chi^*_2)+mi\sigma^2\chi^T_2\chi_1-m\chi_1^{\dagger}i\sigma^2\chi^*_2\\
  &=i\chi_1^{\dagger}\bar{\sigma}^{\mu}\partial_{\mu}\chi_1+i\chi_2^T\bar{\sigma}^{\mu T}\partial_{\mu}\chi^*_2+im(\chi^T_2\sigma^2\chi_1-\chi_1^{\dagger}\sigma^2\chi^*_2)
\end{align*}
here $\sigma^2\sigma^{\mu}=\bar{\sigma}^{\mu T}\sigma^2$ was used. Since
\begin{align*}
  i\chi_2^T\bar{\sigma}^{\mu T}\partial_{\mu}\chi^*_2&=(i\chi_2^T\bar{\sigma}^{\mu T}\partial_{\mu}\chi_2^*)^T\\
  &=-i\partial_{\mu}\chi_2^{\dagger}\bar{\sigma}^{\mu}\chi_2
\end{align*}
we have (discard the boundary term)
\begin{align*}
  \lag&=i\chi_1^{\dagger}\bar{\sigma}^{\mu}\partial_{\mu}\chi_1-i\chi_2^{\dagger}\bar{\sigma}^{\mu}\partial_{\mu}\chi_2+im(\chi^T_2\sigma^2\chi_1-\chi_1^{\dagger}\sigma^2\chi^*_2)
\end{align*}
Now we can rewrite the Lagrangian as
\begin{align*}
  \lag=i\chi^{\dagger}_i\bar{\sigma}^{\mu}\partial_{\mu}\chi_i+\frac{im_{ij}}{2}(\chi^T_i\sigma^2\chi_j-\chi_i^{\dagger}\sigma^2\chi^*_j)
\end{align*}
where the mass matrix is given by
\begin{align*}
  m=\Pmqty{0&m\\m&0}
\end{align*}
(d). The divergence of the first current
\begin{align*}
  \partial_{\mu}J^{\mu}&=\partial_{\mu}(\chi^{\dagger}\bar{\sigma}^{\mu}\chi)\\
  &=\partial_{\mu}\chi^{\dagger}\bar{\sigma}^{\mu}\chi+\chi^{\dagger}\bar{\sigma}^{\mu}\partial_{\mu}\chi\\
  &=\chi^T m\sigma^2\chi+\chi^{\dagger}m\sigma^2\chi^*
\end{align*}
$\partial_{\mu}J^{\mu}=0$ only when $m=0$.

The divergence of the second current
\begin{align*}
  \partial_{\mu} J^{\mu}&=\partial_{\mu}(\chi_1^{\dagger}\bar{\sigma}^{\mu}\chi_1-\chi_2^{\dagger}\bar{\sigma}^{\mu}\chi_2)\\
  &=\partial_{\mu}\chi_1^{\dagger}\bar{\sigma}^{\mu}\chi_1+\chi_1^{\dagger}\bar{\sigma}^{\mu}\partial_{\mu}\chi_1-\partial_{\mu}\chi_2^{\dagger}\bar{\sigma}^{\mu}\chi_2-\chi_2^{\dagger}\bar{\sigma}^{\mu}\partial_{\mu}\chi_2\\
  &=m\chi_2^T\sigma^2\chi_1+\chi_1^{\dagger}m\sigma^2\chi_2^*-m\chi_1^T\sigma^2\chi_2-\chi_2^{\dagger}m\sigma^2\chi_1^*\\
  &=0
\end{align*}
since $\chi^T_2\sigma^2\chi_1=\chi^T_1\sigma^2\chi_2$.

The action for N free massive 2-component fermion fields with O(N) symmetry would be
\begin{align*}
  S=\int\dd^4x \bqty{i\chi^{\dagger}_i\bar{\sigma}^{\mu}\partial_{\mu}\chi_i+\frac{im}{2}(\chi^T_i\sigma^2\chi_i-\chi_i^{\dagger}\sigma^2\chi^*_i)}
\end{align*}

(e). Quantize the Majorana theory.

$$H=\int\dd^3x[\chi^{\dagger}i\vb{\sigma}\cdot\nabla\chi+\frac{im}{2}(\chi^{\dagger}\s^2\chi^*-\chi^T\s^2\chi)]$$

The basic idea is to give a canonical anti-commutation relation and use it to derive the other commutation relations and write the field operators as creation and annihilation operators.


{\bf3.6\quad Fierz Transformations.} Let $u_i$, $i=1,...,4$, be four 4-component Dirac spinors. In the text, we proved the Fierz rearrangement formulae (3.78) and (3.79). The first of these formulae can be written in 4-component notation as
\begin{align}
  \bar{u}_1\gamma^{\mu}(\frac{1+\gamma^5}{2})u_2\bar{u}_3\gamma_{\mu}(\frac{1+\gamma^5}{2})u_4=-\bar{u}_1\gamma^{\mu}(\frac{1+\gamma^5}{2})u_4\bar u_3\gamma_{\mu}(\frac{1+\gamma^5}{2})u_2
\end{align}
In fact, there are similar rearrangement formulae for any product
\begin{align}
  (\bar{u}_1\Gamma^Au_2)(\bar{u}_3\Gamma^Bu_4)
\end{align}
where $\Gamma^A, \,\Gamma^B$ are any of the 16 combinations of Dirac matrices listed in Section 3.4.

(a). To begin, normalize the 16 matrices $\Gamma^A$ to the convention
\begin{align}
  \tr[\Gamma^A\Gamma^B]=4\delta^{AB}
\end{align}
This gives $\Gamma^A=\Bqty{1,\gamma^0,i\gamma^j,...}$; write all 16 elements of this set.

First for the scalar $\mathds{1}$ and the vector $\gamma_{\mu}\rightarrow\{\gamma^0,\,i\gamma^i\}$ (in 4-dimension) the normalization convention is automatically satisfied.

For the tensor $\sigma^{\mu\nu}=\frac{i}{2}[\gamma^{\mu},\gamma^{\nu}]$,
\begin{align*}
  \tr[[\gamma^{\mu},\gamma^{\nu}][\gamma^{\mu},\gamma^{\nu}]]&=\tr[\gm\gn\gm\gn-\gn\gm\gm\gn-\gm\gn\gn\gm+\gn\gm\gn\gm]\\
  &=\tr[\gm\gn\gm\gn-(2\gmn-\gm\gn)\gm\gn-\gm\gn(2\gmn-\gm\gn)+(2\gmn-\gm\gn)(2\gmn-\gm\gn)]\\
  &=\tr\left[\right.\gm\gn\gm\gn-2\gmn\gm\gn+\gm\gn\gm\gn-2\gm\gn\gmn+\gm\gn\gm\gn\\
  &\;\;\;+4\gmn\gmn-2\gmn\gm\gn-2\gm\gn\gmn+\gm\gn\gm\gn\left.\right]\\
  &=\tr[4\gm\gn\gm\gn-8\gmn\gm\gn+4\gmn\gmn]\\
  &=16(\gmn\gmn-\gmm\gnn+\gmn\gnm)-32\gmn\gmn+4\gmn\gmn\\
  &=-16\gmm\gnn+4\gmn\gmn
\end{align*}
and use the relation $\gm\gn=-\gn\gm$ for $\mu\neq\nu$ (otherwise the whole equation vanished)
\begin{align*}
  \tr[\sigma^{\mu\nu}\sigma^{\mu\nu}]&=-\frac{1}{4}\tr[\gm\gn\gm\gn-\gn\gm\gm\gn-\gm\gn\gn\gm+\gn\gm\gn\gm]\\
  &=4\gmm\gnn\\
  &=\begin{cases}
  +4,\;\;\text{if $\mu=\nu$ or $\mu\neq0\&\nu\neq0$}\\
  -4,\;\;\text{if $\mu\text{ or }\nu=1$}
  \end{cases}
\end{align*}
So the tensor $\sigma^{\mu\nu}$ must be normalized like:
\begin{align*}
  \sigma^{0i},\;\;\;-\sigma^{ij}
\end{align*}
where $i,j$ stands for $1,\,2,\,3$.

For the pseudovector $\gamma^{\mu}\gamma^5$,
\begin{align*}
  \tr[\gm\gamma^5\gm\gamma^5]&=\tr[\gm\gamma^5\gamma^5\gm]\\
  &=\tr[\gm\gm \mathds{1}]\\
  &=\tr[\gmm \mathds{1}]\\
  &=\begin{cases}
  +4,\;\text{if $\mu=0$}\\
  -4,\;\text{if $\mu=1,2,3$}
  \end{cases}
\end{align*}
which gives the following normalization
\begin{align*}
  \gamma^0\gamma^5,\;\;\;i\gamma^i\gamma^5
\end{align*}
where $i=1,2,3$.

For the pseudoscalar $\gamma^5$,
\begin{align*}
  \tr[\gamma^5\gamma^5]=\tr[\mathds{1}]=4
\end{align*}
so the normalization must be $\gamma^5$.

(b). Write the general Fierz identity as an equation
\begin{align}
  (\bar u_1\Gamma^Au_2)(\bar u_3\Gamma^Bu_4)=\sum_{C,D}C^{AB}_{\;\;\;\;\;CD}(\bar u_1\Gamma^Cu_4)(\bar u_3\Gamma^Du_2),
\end{align}
with unknown coefficients $C^{AB}_{\;\;\;\;\;CD}$. Using the compleleness of the 16 $\Gamma^A$ matrices, show that
\begin{align}
  C^{AB}_{\;\;\;\;\;CD}=\frac{1}{16}\tr[\Gamma^C\Gamma^A\Gamma^D\Gamma^B].
\end{align}

We can write
\begin{align*}
  (\bar u_1\Gamma^Au_2)(\bar u_3\Gamma^Bu_4)=\bar u^a_1u_2^b\bar u^c_3 u^d_4M_{abcd}
\end{align*}
where
\begin{align*}
  M_{abcd}=\Gamma^A_{ab}\Gamma^B_{cd}
\end{align*}
Define the 16 matrices $G_{cb}$ by $(G_{cb})_{ad}=G_{abcd}$, we then find
\begin{align*}
  M_{abcd}&=\sum_{r,s}(G_{rs})_{ad}\delta^r_c\delta^s_d\\
  &=\sum_{r,s}(G_{rs})_{ad}(H^{rs})_{cb}
\end{align*}
where $(H^{rs})_{cb}=\delta^r_c\delta^s_d$. Now according to the completeness of the 16 matrices basis, we can expand $M$ and $H$ into
\begin{align*}
  \Gamma^A_{ab}\Gamma^B_{cd}=\sum_{C,D}C^{AB}_{CD}\Gamma^C_{ad}\Gamma^D_{cb}
\end{align*}
then
\begin{align*}
  (\bar u_1\Gamma^Au_2)(\bar u_3\Gamma^Bu_4)=\sum_{C,D}C^{AB}_{\;\;\;\;\;CD}(\bar u_1\Gamma^Cu_4)(\bar u_3\Gamma^Du_2)
\end{align*}
(Actually our derivation is completely based on this relation, especially when we construct the two matrices, we constructed them based on this relation. )

Now we have
\begin{align*}
  \Gamma^A_{ab}\Gamma^B_{cd}&=C^{AB}_{CD}\Gamma^C_{ad}\Gamma^D_{cb}\\
  \Gamma^A_{ab}\Gamma^B_{cd}\Gamma^C_{da}\Gamma^D_{bc}&=C^{AB}_{CD}\Gamma^C_{ad}\Gamma^D_{cb}\Gamma^C_{da}\Gamma^D_{bc}\\
  \Gamma^C_{da}\Gamma^A_{ab}\Gamma^D_{bc}\Gamma^B_{cd}&=C^{AB}_{CD}\Gamma^C_{ad}\Gamma^C_{da}\Gamma^D_{cb}\Gamma^D_{bc}
\end{align*}
which means
\begin{align*}
  \tr[CADB]=C^{AB}_{CD}\tr[C^2]\tr[D^2]\\
  C^{AB}_{CD}=\frac{1}{16}\tr[CADB]
\end{align*}

(c). Work out explicitly the Fierz transformation laws for the products $(\bar u_1u_2)(\bar u_3u_4)$ and $(\bar u_1\gamma^{\mu}u_2)(\bar u_3\gamma_{\mu}u_4)$.

For the first one, we have $\Gamma^A=\Gamma^B=1$, so
\begin{align*}
  (\bar u_1u_2)(\bar u_3u_4)=\frac{1}{4}\sum_C(\bar u_1\Gamma^Cu_4)(\bar u_3\Gamma^Cu_2)
\end{align*}

For the second one
\begin{align*}
  (\bar u_1\gamma^{\mu}u_2)(\bar u_3\gamma_{\mu}u_4)
\end{align*}
if $\mu=0$, it's exactly the same as before. And if $\mu=i=1,2,3$, only use the given relation, we have $\Gamma^A=\Gamma^B=1$ again, and
\begin{align*}
  (\bar u_1\gamma^{\mu}u_2)(\bar u_3\gamma_{\mu}u_4)=\sum_C\eta_C(\bar u_1\Gamma^Cu_4)(\bar u_3\Gamma^Cu_2)
\end{align*}
where $\eta_C=1$ for a scalar $\Gamma^C$, $\eta_C=-1$ for a pseudoscalar one, $\eta_C=-\frac{1}{2}$ for a vector one and $\frac{1}{2}$ for a pseudovector one.


{\bf3.7}\quad(a). Consider the C, P, T transformation, given thje antisymmetric tensor $\bar\psi\sigma^{\mu\nu}\psi$, we have
\begin{align*}
  \bar\psi\sigma^{\mu\nu}\psi\rightarrow P\bar\psi\sigma^{\mu\nu}\psi P\\
  \bar\psi\sigma^{\mu\nu}\psi\rightarrow T\bar\psi\sigma^{\mu\nu}\psi T\\
  \bar\psi\sigma^{\mu\nu}\psi\rightarrow C\bar\psi\sigma^{\mu\nu}\psi C
\end{align*}
For parity
\begin{align*}
  P\bar\psi\sigma^{\mu\nu}\psi P&=P\bar\psi PP\sigma^{\mu\nu}\psi P\\
  &=\eta_a^*\eta_a\bar\psi\gamma^0\sigma^{\mu\nu}\gamma^0\psi \\
  &=\frac{i}{2}\bar\psi\gamma^0[\gm,\gn]\gamma^0\psi\\
  &=\frac{i}{2}\bar\psi(\gamma^0\gm\gn\gamma^0-\gamma^0\gn\gm\gamma^0)\psi\\
  &=\frac{i}{2}\bar\psi((2g^{0\mu}-\gm\gamma^0)(2g^{\nu0}-\gamma^0\gn)-(2g^{0\nu}-\gn\gamma^0)(2g^{\mu0}-\gamma^0\gm))\psi\\
  &=\frac{i}{2}\bar\psi(4g^{0\mu}g^{\nu0}-2g^{0\mu}\gamma^0\gn-2\gm\gamma^0g^{\nu0}+\gm\gamma^0\gamma^0\gn-4g^{0\nu}g^{\mu0}+2g^{\mu0}\gn\gamma^0+2g^{0\nu}\gamma^0\gm-\gn\gamma^0\gamma^0\gm)\psi\\
  &=\frac{i}{2}\bar\psi(2g^{0\mu}[\gn,\gamma^0]+[\gm,\gn]+2g^{0\nu}[\gamma^0,\gm])\psi
\end{align*}
Apparently, we have
\begin{align*}
  P\bar\psi\sigma^{\mu\nu}\psi P&=(-1)^{\mu}(-1)^{\nu}\bar\psi\sigma^{\mu\nu}\psi
\end{align*}
For time reveral
\begin{align*}
  T\bar\psi\sigma^{\mu\nu}\psi T&=T\bar\psi TT\sigma^{\mu\nu}\psi T\\
  &=\bar\psi (-\gamma^1\gamma^3) \sigma^{\mu\nu}^*(\gamma^1\gamma^3)\psi\\
  &=-\frac{i}{2}\bar\psi (-\gamma^1\gamma^3) [\gm^*,\gn^*](\gamma^1\gamma^3)\psi\\
  &=-\frac{i}{2}\bar\psi (\gamma^1\gamma^3\gn^*\gm^*\gamma^1\gamma^3-\gamma^1\gamma^3\gm^*\gn^*\gamma^1\gamma^3)\psi
\end{align*}
If $\mu$ or $\nu$ equals to $2$, and the other is not, then
$$[\gm^*,\gn^*]=-[\gm,\gn]$$
Under this circumstance, if the other equals to 1 or 3, then
\begin{align*}
  T\bar\psi\sigma^{\mu\nu}\psi T&=-\frac{i}{2}\bar\psi (\gamma^1\gamma^3) [\gm,\gn](\gamma^1\gamma^3)\psi\\
  &=-\frac{i}{2}\bar\psi (\gamma^1\gamma^3) (\gm\gn-\gn\gm)(\gamma^1\gamma^3)\psi\\
  &=-\frac{i}{2}\bar\psi (\gamma^1\gamma^3) (2\gm\gn)(\gamma^1\gamma^3)\psi\\
  &=-i\bar\psi\gm\gn\psi\\
  &=-\frac{i}{2}\bar\psi [\gm,\gn]\psi\\
  &=-\bar\psi\s^{\mu\nu}\psi
\end{align*}
($$\begin{array}{rclc}
  \g^1\g^3&\g^2\g^1&\g^1\g^3&=\g^2\g^1\\
  &\g^1\g^2&&=\g^1\g^2\\
  &\g^2\g^3&&=\g^2\g^3\\
  &\g^3\g^2&&=\g^3\g^2\\
\end{array}$$
)
if the other equals to 0, then
\begin{align*}
  T\bar\psi\sigma^{\mu\nu}\psi T&=-\frac{i}{2}\bar\psi (\gamma^1\gamma^3) (2\gm\gn)(\gamma^1\gamma^3)\psi\\
  &=\bar\psi\s^{\mu\nu}\psi
\end{align*}
If none of $\mu$ and $\nu$ equals to 2
$$[\gm^*,\gn^*]=[\gm,\gn]$$
Then if $\mu$ and $\nu$ equals to 1 or 3, and the other equals to 0
\begin{align*}
  T\bar\psi\sigma^{\mu\nu}\psi T&=-\frac{i}{2}\bar\psi (-\gamma^1\gamma^3) [\gm^*,\gn^*](\gamma^1\gamma^3)\psi\\
  &= \frac{i}{2}\bar\psi (\gamma^1\gamma^3) [\gm,\gn](\gamma^1\gamma^3)\psi\\
  &=\frac{i}{2}\bar\psi (\gamma^1\gamma^3) (\gm\gn-\gn\gm)(\gamma^1\gamma^3)\psi\\
  &=\frac{i}{2}\bar\psi (\gamma^1\gamma^3) (2\gm\gn)(\gamma^1\gamma^3)\psi\\\intertext{after twice minus sign insert}
  &=\bar\psi\s^{\mu\nu}\psi
\end{align*}
and if $\mu$ and $\nu$ equal to 1 and 3 (or vice versa)
\begin{align*}
  T\bar\psi\sigma^{\mu\nu}\psi T&=-\frac{i}{2}\bar\psi (\gamma^1\gamma^3) (2\gm\gn)(\gamma^1\gamma^3)\psi\\
  &=\bar\psi\s^{\mu\nu}\psi
\end{align*}
So
\begin{align*}
  T\bar\psi\sigma^{\mu\nu}\psi T=-(-1)^{\mu}(-1)^{\nu}\bar\psi\sigma^{\mu\nu}\psi
\end{align*}
Similarly, we have
\begin{align*}
  C\bar\psi\s^{\mu\nu}\psi C=-\bar\psi\s^{\mu\nu}\psi
\end{align*}

(b). For complex Klein-Gordon field, we have
\begin{align}
  \phi^*(\vb{x})&=\int\frac{\dd^3p}{(2\pi)^3}\frac{1}{\sqrt{2\omega_{\vb{p}}}}(a_{\vb{p}}^{\dagger}e^{ip\cdot x}+b_{\vb{p}}e^{-ip\cdot x})
  =\int\frac{\dd^3p}{(2\pi)^3}\frac{1}{\sqrt{2\omega_{\vb{p}}}}(a_{\vb{p}}^{\dagger}+b_{-\vb{p}})e^{ip\cdot x}\\
  \phi(\vb{x})&=\int\frac{\dd^3p}{(2\pi)^3}\frac{1}{\sqrt{2\omega_{\vb{p}}}}(a_{\vb{p}}e^{-ip\cdot x}+b_{\vb{p}}^{\dagger}e^{ip\cdot x})
  =\int\frac{\dd^3p}{(2\pi)^3}\frac{1}{\sqrt{2\omega_{\vb{p}}}}(a_{-\vb{p}}+b_{\vb{p}}^{\dagger})e^{ip\cdot x}\\
%  \pi^*(\vb{x})&=\int\frac{\dd^3p}{(2\pi)^3}(-i)\sqrt{\frac{\omega_{\vb{p}}}{2}}(a_{\vb{p}}e^{-ip\cdot x}-b_{\vb{p}}^{\dagger}e^{ip\cdot x})
%  =\int\frac{\dd^3p}{(2\pi)^3}(-i)\sqrt{\frac{\omega_{\vb{p}}}{2}}(a_{-\vb{p}}-b_{\vb{p}}^{\dagger})e^{ip\cdot x}\\
%  \pi(\vb{x})&=\int\frac{\dd^3p}{(2\pi)^3}i\sqrt{\frac{\omega_{\vb{p}}}{2}}(a_{\vb{p}}^{\dagger}e^{ip\cdot x}-b_{\vb{p}}e^{-ip\cdot x})
%  =\int\frac{\dd^3p}{(2\pi)^3}i\sqrt{\frac{\omega_{\vb{p}}}{2}}(a_{\vb{p}}^{\dagger}-b_{-\vb{p}})e^{ip\cdot x}\\
%  \nabla\phi^*(\vb{x})&=\int\frac{\dd^3p}{(2\pi)^3}(-i)\frac{\vb{p}}{\sqrt{2\omega_{\vb{p}}}}(a_{\vb{p}}^{\dagger}e^{ip\cdot x}-b_{\vb{p}}e^{-ip\cdot x})
%  =\int\frac{\dd^3p}{(2\pi)^3}(-i)\frac{\vb{p}}{\sqrt{2\omega_{\vb{p}}}}(a_{\vb{p}}^{\dagger}+b_{-\vb{p}})e^{ip\cdot x}\\
%  \nabla\phi(\vb{x})&=\int\frac{\dd^3p}{(2\pi)^3}i\frac{\vb{p}}{\sqrt{2\omega_{\vb{p}}}}(a_{\vb{p}}e^{-ip\cdot x}-b_{\vb{p}}^{\dagger}e^{ip\cdot x})
%  =\int\frac{\dd^3p}{(2\pi)^3}(-i)\frac{\vb{p}}{\sqrt{2\omega_{\vb{p}}}}(a_{-\vb{p}}+b_{\vb{p}}^{\dagger})e^{ip\cdot x}
\end{align}
Use $P$ operator to act on $\phi$
\begin{align*}
  P\phi(t,\vb{x})P=\phi(t,-\vb{x})
\end{align*}
and
\begin{align*}
  &P\phi(t,\vb{x})P=\int\frac{\dd^3p}{(2\pi)^3}\frac{1}{\sqrt{2\omega_{\vb{p}}}}(a_{-\vb{p}}+b_{\vb{p}}^{\dagger})e^{i(Et+\vb{p}\cdot\vb{x})}=\int\frac{\dd^3p}{(2\pi)^3}\frac{1}{\sqrt{2\omega_{\vb{p}}}}(a_{\vb{p}}+b_{-\vb{p}}^{\dagger})e^{ip\cdot x}\\
  =&P\int\frac{\dd^3p}{(2\pi)^3}\frac{1}{\sqrt{2\omega_{\vb{p}}}}(a_{-\vb{p}}+b_{\vb{p}}^{\dagger})e^{ip\cdot x}P\\
  =&\int\frac{\dd^3p}{(2\pi)^3}\frac{1}{\sqrt{2\omega_{\vb{p}}}}P(a_{-\vb{p}}+b_{\vb{p}}^{\dagger})Pe^{ip\cdot x}\\
  \Longrightarrow&Pa_{\vb{p}}P=a_{-\vb{p}},\;\;\;\;Pb_{\vb{p}}^{\dagger}P=b_{-\vb{p}}^{\dagger}\\
  \intertext{Similarly, act on $\phi^*$, we have}&Pa_{\vb{p}}^{\dagger}P=a_{-\vb{p}}^{\dagger},\;\;\;\;Pb_{\vb{p}}P=b_{-\vb{p}}
\end{align*}
And for $T$
$$T\phi(t,\vb{x})T=\phi(-t,\vb{x})$$
and
\begin{align*}
  &T\phi(t,\vb{x})T=\int\frac{\dd^3p}{(2\pi)^3}\frac{1}{\sqrt{2\omega_{\vb{p}}}}(a_{-\vb{p}}+b_{\vb{p}}^{\dagger})e^{i(-Et-\vb{p}\cdot\vb{x})}=\int\frac{\dd^3p}{(2\pi)^3}\frac{1}{\sqrt{2\omega_{\vb{p}}}}(a_{\vb{p}}+b_{-\vb{p}}^{\dagger})e^{-ip\cdot x}\\
  =&T\int\frac{\dd^3p}{(2\pi)^3}\frac{1}{\sqrt{2\omega_{\vb{p}}}}(a_{-\vb{p}}+b_{\vb{p}}^{\dagger})e^{ip\cdot x}T\\
  =&\int\frac{\dd^3p}{(2\pi)^3}\frac{1}{\sqrt{2\omega_{\vb{p}}}}T(a_{-\vb{p}}+b_{\vb{p}}^{\dagger})Te^{ip\cdot x}\\
  \Longrightarrow&Ta_{\vb{p}}T=a_{-\vb{p}},\;\;\;\;Tb_{\vb{p}}^{\dagger}T=b_{-\vb{p}}^{\dagger}\\
  \intertext{Similarly, act on $\phi^*$, we have}&Ta_{\vb{p}}^{\dagger}T=a_{-\vb{p}}^{\dagger},\;\;\;\;Tb_{\vb{p}}T=b_{-\vb{p}}
\end{align*}
Last the $C$ operator
$$C\phi(t,\vb{x})C=\phi^*(t,\vb{x})$$
and
\begin{align*}
  &C\phi(t,\vb{x})C=\int\frac{\dd^3p}{(2\pi)^3}\frac{1}{\sqrt{2\omega_{\vb{p}}}}(a_{\vb{p}}^{\dagger}+b_{-\vb{p}})e^{ip\cdot x}%=\int\frac{\dd^3p}{(2\pi)^3}\frac{1}{\sqrt{2\omega_{\vb{p}}}}(a_{\vb{p}}+b_{-\vb{p}}^{\dagger})e^{-ip\cdot x}\\
  \\=&C\int\frac{\dd^3p}{(2\pi)^3}\frac{1}{\sqrt{2\omega_{\vb{p}}}}(a_{-\vb{p}}+b_{\vb{p}}^{\dagger})e^{ip\cdot x}C\\
  =&\int\frac{\dd^3p}{(2\pi)^3}\frac{1}{\sqrt{2\omega_{\vb{p}}}}C(a_{-\vb{p}}+b_{\vb{p}}^{\dagger})Ce^{ip\cdot x}\\
  \Longrightarrow&Ca_{\vb{p}}C=a_{-\vb{p}}^{\dagger},\;\;\;\;Cb_{\vb{p}}^{\dagger}C=b_{-\vb{p}}\\
  \intertext{Similarly, act on $\phi^*$, we have}&Ca_{\vb{p}}^{\dagger}C=a_{-\vb{p}},\;\;\;\;Cb_{\vb{p}}C=b_{-\vb{p}}^{\dagger}
\end{align*}

For the current
$$J^{\mu}=i(\phi^*\partial^{\mu}\phi-\partial^{\mu}\phi^*\phi)$$
we can calculate its transformation under C, P, T.
\begin{align*}
  PJ^{\mu}P&=iP(\phi^*\partial^{\mu}\phi-\partial^{\mu}\phi^*\phi)P\\
  &=i(\phi^*(t,-\vb{x})\partial^{\mu}\phi(t,-\vb{x})-\partial^{\mu}\phi^*(t,-\vb{x})\phi(t,-\vb{x}))\\
  &=(-1)^{\mu}i(\phi^*\partial^{\mu}\phi-\partial^{\mu}\phi^*\phi)
\end{align*}
where we use the shorthand as the textbook $(-1)^{\mu}\equiv1,\mu=0$ and $(-1)^{\mu}\equiv -1,\mu=1,2,3$.
\begin{align*}
  TJ^{\mu}T&=iT(\phi^*\partial^{\mu}\phi-\partial^{\mu}\phi^*\phi)T\\
  &=i(\phi^*(-t,\vb{x})\partial^{\mu}\phi(-t,\vb{x})-\partial^{\mu}\phi^*(-t,\vb{x})\phi(-t,\vb{x}))\\
  &=-(-1)^{\mu}i(\phi^*\partial^{\mu}\phi-\partial^{\mu}\phi^*\phi)
\end{align*}
and
\begin{align*}
  CJ^{\mu}C&=iC(\phi^*\partial^{\mu}\phi-\partial^{\mu}\phi^*\phi)C\\
  &=i(\phi(t,\vb{x})\partial^{\mu}\phi^*(t,\vb{x})-\partial^{\mu}\phi(t,\vb{x})\phi^*(t,\vb{x}))\\
  &=i(\phi\partial^{\mu}\phi^*-\partial^{\mu}\phi\phi^*)\\
  &=-J^{\mu}
\end{align*}

(c). See Weinberg 5.8.

Define $A$ built from $\phi$ and $B$ built from $\psi$. To make them Lorentz invariant, we can always assume $A\propto\phi^*A'\phi$ and $B\propto \bar\psi B'\psi$. Assuming the eigenvalues of CPT is 1, then
$$CPT\; A(x) \; (CPT)^{-1}=A(-x)$$
$$CPT\; B(x) \; (CPT)^{-1}=B^{\dagger}(-x)=B(-x)$$
So it's invariant under CPT transformation.


{\bf4.3\quad
Linear sigma model.}

The Hamiltionian of N scalar fields is
$$H=\int\dd^3x(\frac{1}{2}(\Pi^i)^2+\frac{1}{2}(\nabla\Phi^i)^2+V(\Phi^2))$$
where $(\Phi^i)^2=\bm{\Phi\cdot\Phi}$, and
$$V(\Phi^2)=\frac{1}{2}m^2(\Phi^i)^2+\frac{\lambda}{4}((\Phi^i)^2)^2$$
is a function symmetric under rotations of $\Phi$.

(a). Show that the propagator is
$$\contraction{}{\Phi}{^i(x)}{\Phi}\Phi^i(x)\Phi^j(y)=\delta^{ij}D_F(x-y)$$
The interaction part of Hamiltionian is
$$\mathcal{H}_{int}=\frac{\lambda}{4}((\Phi^i)^2)^2=\frac{\lambda}{4}\bqty{\sum_{i,j}{\Phi^i}^2{\Phi^j}^2}=\frac{\lambda}{4}\bqty{\sum_i{\Phi^i}^4+2\sum_{i<j}{\Phi^i}^2{\Phi^j}^2}$$
it's a N scalar field $\phi^4$ theory. The tree level is
$$\mel{0}{T\Bqty{(\frac{-i\lambda}{4})\Phi^i(x)\Phi^j(y)\bqty{\int\dd^3 z\sum_k{\Phi^k}^4+2\sum_{k<l}{\Phi^k}^2{\Phi^l}^2}}}{0}$$
%Divide $\Phi$ into $\Phi^i$, the original $\Phi\Phi$ becomes $\sum_{i,j}\Phi^i\Phi^j$.
and it involves two type of vertex: one with different field species $\Phi^i$ and $\Phi^j$, which has the form of $\frac{-i\lambda}{4}4!=-6i\lambda$; one with only one field species, which has the form of $\frac{-i\lambda}{4}2!2!2!=-2i\lambda$.

The propagator is $\contraction{}{\Phi}{^i(x)}{\Phi}\Phi^i(x)\Phi^j(y)$. It would be ($x^0>y^0$)
\begin{align*}
  [\Phi^i_+(x),\Phi^j_-(y)]&=\int\frac{\dd^3p}{(2\pi)^3}\frac{\dd^3k}{(2\pi)^3}[{a^i_{\vb{p}}}^{\dagger},a^j_{\vb{k}}]e^{i(k\cdot y-p\cdot x)}\\
  &=\int\frac{\dd^3p}{(2\pi)^3}\frac{\dd^3k}{(2\pi)^3}(2\pi)^3\delta^3(\vb{p-k})e^{i(k\cdot y-p\cdot x)}\\
  &=\int\frac{\dd^3p}{(2\pi)^3}e^{ip\cdot(y-x)}\delta^{ij}\\
  &=\delta^{ij}D_F(x-y)|_{x^0>y^0}
\end{align*}
Similarly, when $y^0>x^0$, the result becomes $\delta^{ij}D_F(x-y)$, which shows the propagator is exactly as showed above.


Give the following vertex
%\begin{tikzpicture}
%  \begin{feynman}
\begin{align*}
\begin{gathered}
    {\feynmandiagram[layered layout, vertical=f1 to p2]{
    f1 [particle=k] -- f2 [small,dot] -- f4 [particle=l] ,
    p2 [particle=i] -- f2 -- p3 [particle=j],
    };}
\end{gathered}
  =-2i\lambda(\delta^{kl}\delta^{ij}+\delta^{ki}\delta^{lj}+\delta^{kj}\delta^{li})
\end{align*}
%  \end{feynman}
%\end{tikzpicture}
%\begin{center}
%\begin{tikzpicture}
%  \begin{feynman}
%    \vertex (f1);
%    \vertex [above left=of f1] (a) {k};
%    \vertex [above right=of f1] (c) {l};
%    \vertex [below left=of f1] (b) {i};
%    \vertex [below right=of f1] (d) {j};
%
%    \diagram* {
%    (a) -- (f1) -- (c),
%    (b) -- (f1) -- (d),
%    };
%  \end{feynman}
%\end{tikzpicture}
%\end{center}
To calculate the differential cross section $\dv{\sigma}{\Omega}$ in centre-of-mass frame, we have
$$\dv{\s}{\Omega}=\frac{\abs{\mathcal{M}}^2}{64\pi^2E_{CM}^2}$$
And
\begin{align*}
  \mathcal{M}(\Phi^k+\Phi^i\rightarrow\Phi^l+\Phi^j)=-2i\lambda(\delta^{kl}\delta^{ij}+\delta^{ki}\delta^{lj}+\delta^{kj}\delta^{li})
\end{align*}
From this we have $\mathcal{M}$ for scenario with different species of $\Phi$, which is showed above.

So the differential cross section of two different $\Phi$ is
$$ \dv{\s}{\Omega}=\frac{\abs{-2\lambda}^2}{64\pi^2E_{CM}^2}=\frac{\lambda^2}{16\pi^2E_{CM}^2}$$
and of one kind of $\Phi$ it's
$$ \dv{\s}{\Omega}=\frac{\abs{-6\lambda}^2}{64\pi^2E_{CM}^2}=\frac{9\lambda^2}{16\pi^2E_{CM}^2}$$
For $\Phi^1+\Phi^1\rightarrow\Phi^2+\Phi^2$, do the angular integration and obtain $2\pi$ (there's a factor 2 due to the exchange of two incoming particles)
$$\s_{tot} =\frac{\lambda^2}{8\pi E_{CM}^2}$$
for $\Phi^1+\Phi^2\rightarrow\Phi^1+\Phi^2$
$$\s_{tot} = \frac{\lambda^2}{4\pi E_{CM}^2}$$
and for $\Phi^1+\Phi^1\rightarrow\Phi^1+\Phi^1$
$$\s_{tot} = \frac{9\lambda^2}{4\pi E_{CM}^2}$$

(b). Now consider the situation $m^2<0$: $m^2=-\mu^2$, the classical potential becomes
$$V(\Phi^2)=-\frac{1}{2}\mu^2(\Phi^i)^2+\frac{\lambda}{4}((\Phi^i)^2)^2$$
and it has a minimum
$$\boldsymbol{\Phi}^2=\frac{\mu^2}{\lambda}$$
Shift this field according to
$$\Phi^i(x)=\pi^i(x),\;i=1,2,\dots,N-1$$
$$\Phi^N(x)=v+\s(x)$$
rewrite the Lagrangian
$$\lag=\frac{1}{2}(\partial\s)^2-\mu^2\s^2+\frac{1}{2}(\partial\bm{\pi})^2-\lambda v\s(\s^2+\bm{\pi}^2)-\frac{1}{4}\lambda(\s^2+\bm{\pi}^2)^2$$
The first three terms stands for the free part of this field and the rest is the interaction terms. It's clear then that $\sigma$ is a massive field with mass $\mu$ and $\pi^i$ is a massless field ($i=1,2,\dots,N-1$).





{\bf Additional}\quad(a). Given the Lagrangian
\begin{align}
  \mathcal{L}=\frac{1}{2}(\partial_{\mu}\phi)^2-\frac{\lambda}{4!}\phi^4
\end{align}
and the definition of energy-momentum tensor
\begin{align}
  T^{\mu}_{\nu}\equiv\pdv{\mathcal{L}}{(\partial_{\mu}\phi)}\partial_{\nu}\phi-\mathcal{L}\delta^{\mu}_{\nu}
\end{align}
Then we have
\begin{align*}
  T_{\mu\nu}=&\pdv{\mathcal{L}}{(\partial_{\mu'}\phi)}\partial_{\nu}\phi g_{\mu\mu'}-\mathcal{L}g_{\mu\nu}\\
  =&\partial_{\mu}\phi\partial_{\nu}\phi-\frac{1}{2}\partial_{\mu}\phi\partial_{\nu}\phi+\frac{\lambda}{4!}\phi^4g_{\mu\nu}\\
  =&\frac{1}{2}\partial_{\mu}\phi\partial_{\nu}\phi+\frac{\lambda}{4!}\phi^4g_{\mu\nu}
\end{align*}




\makebox{\phantom{\bf Additional}}\quad(b). Consider dilation transformation $\delta x^{\mu}=\alpha  x^{\mu}$, $\delta \phi=-\alpha\phi$, derive the Noether current $J_{\mu}^D$. If we add $\Delta\mathcal{L}=-\frac{1}{2}m^2\phi^2$, will this dilation symmatry still hold?

According to Neother's theorem, we can derive conservation law from Lagrangian symmatry. We knew that given the least action principle and transformation relations, we have
\begin{align}\label{deltaS}
  \delta S=\int\dd^4x\partial_{\mu}[(\mathcal{L}g^{\mu}_{\rho}-\pdv{\mathcal{L}}{(\partial_{\mu}\phi)}\partial_{\rho}\phi)\delta x^{\rho}+\pdv{\mathcal{L}}{(\partial_{\mu}\phi)}\delta\phi]=0
\end{align}
which means
\begin{align*}
  \delta S=\id \partial_{\mu}[-T^{\mu}_{\rho}\delta x^{\rho}+\pdv{\mathcal{L}}{(\partial_{\mu}\phi)}\delta\phi]=0
\end{align*}
Then we have the corresponding current
\begin{align*}
  J^{\mu}=-T^{\mu}_{\rho}\delta x^{\rho}+\pdv{\mathcal{L}}{(\partial_{\mu}\phi)}\delta\phi=-T^{\mu}_{\rho}\delta x^{\rho}+\partial^{\mu}\phi\delta\phi
\end{align*}
By defining $j^{\mu}=-J^{\mu}/\alpha$, we have
\begin{align*}
  j^{\mu}=T^{\mu}_{\rho}x^{\rho}+\partial^{\mu}\phi\,\phi
\end{align*}
and (given $\partial_{\mu}T^{\mu}_{\rho}=0$)
\begin{align*}
  \partial_{\mu}j^{\mu}&=T^{\mu}_{\rho}\partial_{\mu}x^{\rho}+\phi\square\phi+\partial_{\mu}\phi\partial^{\mu}\phi\\
  &=T^{\mu}_{\mu}+\phi\square\phi+\partial_{\mu}\phi\partial^{\mu}\phi
\end{align*}
From part (a), we know that
\begin{align*}
  T^{\mu}_{\mu}=(1-\frac{d}{2})\partial_{\mu}\phi\partial^{\mu}\phi+\frac{\lambda d}{4!}\phi^4
\end{align*}
where $d$ is the dimension of spacetime. From the equaton of motion, we know that
\begin{align*}
  \square\phi+\frac{\lambda}{3!}\phi^3=0
\end{align*}
Now (assuming it's a 4-d situation)
\begin{align*}
  \partial_{\mu}j^{\mu}=0
\end{align*}

If we add $\Delta\mathcal{L}=-\frac{1}{2}m^2\phi^2$, the Lagrangian becomes
\begin{align}
  \lag=\frac{1}{2}\partial_{\mu}\phi\partial^{\mu}\phi-\frac{1}{2}m^2\phi^2-\frac{\lambda}{4!}\phi^4
\end{align}
From our previous discussion, the energy-momentum tensor now should be
\begin{align*}
    T^{\mu}_{\mu}=(1-\frac{d}{2})\partial_{\mu}\phi\partial^{\mu}\phi+\frac{\lambda d}{4!}\phi^4+\frac{d}{2}m^2\phi^2
\end{align*}
Then (assuming it's 4-d again)
\begin{align*}
  \partial_{\mu}j^{\mu}&=(1-2)\partial_{\mu}\phi\partial^{\mu}\phi+\frac{4\lambda }{4!}\phi^4+2m^2\phi^2-\frac{\lambda}{3!}\phi^4+\partial_{\mu}\phi\partial^{\mu}\phi\\
  &=2m^2\phi^2
\end{align*}
The current doesn't conserve so the symmatry won't hold.


Another approach: Using the transformation relations given before, we have
\begin{align*}
  \delta S&=\int\dd^4x\partial_{\mu}[(\mathcal{L}g^{\mu}_{\rho}-\pdv{\mathcal{L}}{(\partial_{\mu}\phi)}\partial_{\rho}\phi)\alpha x^{\rho}-\pdv{\mathcal{L}}{(\partial_{\mu}\phi)}\alpha\phi]\\
  &=\alpha\int\dd^4x\partial_{\mu}[(\mathcal{L}g^{\mu}_{\rho}-\pdv{\mathcal{L}}{(\partial_{\mu}\phi)}\partial_{\rho}\phi)x^{\rho}-\pdv{\mathcal{L}}{(\partial_{\mu}\phi)}\phi]\\
  &=\alpha\int\dd^4x\partial_{\mu}[\frac{1}{2}\partial_{\mu'}\phi\partial^{\mu'}\phi  x^{\mu}-\frac{\lambda}{4!}\phi^4x^{\mu}-\partial^{\mu}\phi\partial_{\rho}\phi x^{\rho}-(\partial^{\mu}\phi)\phi]\\
  &=0
\end{align*}
Then the integrand goes to zero
\begin{align*}
  \partial_{\mu}[\frac{1}{2}\partial_{\mu'}\phi\partial^{\mu'}\phi  x^{\mu}-\frac{\lambda}{4!}\phi^4x^{\mu}-\partial^{\mu}\phi\partial_{\rho}\phi x^{\rho}-(\partial^{\mu}\phi)\phi]=0
\end{align*}
where $J^{\mu}=\frac{1}{2}\partial_{\mu'}\phi\partial^{\mu'}\phi  x^{\mu}-\frac{\lambda}{4!}\phi^4x^{\mu}-\partial^{\mu}\phi\partial_{\rho}\phi x^{\rho}-(\partial^{\mu}\phi)\phi$ is exactly the Noether current we want.





The derivation of formula \eqref{deltaS} :
Consider a general transformation
\begin{align*}
  f(x)\rightarrow f'(x')
\end{align*}
and define
\begin{align*}
  \delta f\equiv f'(x')-f(x)=f'(x+\delta x^{\mu})-f(x)\cong f'(x)-f(x)+\delta x^{\mu}\partial_{\mu}f+\mathcal{O}(\delta x^2)
\end{align*}
and also define
\begin{align*}
  \delta_0 f=f'(x)-f(x)
\end{align*}
then
\begin{align*}
  \delta f=\delta_0 f+\delta x^{\mu}\partial_{\mu}f
\end{align*}
Now we deal with this problem from the least action principle first
\begin{align*}
  &\delta S=0\\
  =&\id\delta\lag+\int\delta(\dd^4x)\lag
\end{align*}
And
\begin{align*}
  \delta\lag&=\delta_0\lag+\delta x^{\mu}\partial_{\mu}\lag\\
  &=\pdv{\lag}{\phi}\delta_0\phi+\pdv{\lag}{(\partial_{\mu}\phi)}\delta_0(\partial_{\mu}\phi)+\delta x^{\mu}\partial_{\mu}\lag\\
  &=\delta x^{\mu}\partial_{\mu}\lag+(\pdv{\lag}{\phi}-\partial_{\mu}\pdv{\lag}{(\partial_{\mu}\phi)})\delta_0\phi+\partial_{\mu}(\pdv{\lag}{(\partial_{\mu}\phi)}\delta_0\phi)
\end{align*}
The other part
\begin{align*}
  \delta(\dd^4x)&=\pdv{\delta x^{\mu}}{x^{\mu}}\dd^4x=\partial_{\mu}(\delta x^{\mu})\lag
\end{align*}
which can be derived by
\begin{align*}
  \dd^4x'=\abs{\pdv{x'^{\mu}}{x^{\nu}}}\dd^4x=\abs{\pdv{(x^{\mu}+\delta x^{\mu})}{x^{\nu}}}\dd^4x=(1+\pdv{\delta x^{\mu}}{x^{\mu}})\dd^4x
\end{align*}
And the whole part of $\delta S$ becomes (applying Euler-Lagarange equation)
\begin{align*}
  \delta S&=\id \delta x^{\mu}\partial_{\mu}\lag+(\pdv{\lag}{\phi}-\partial_{\mu}\pdv{\lag}{(\partial_{\mu}\phi)})\delta_0\phi+\partial_{\mu}(\pdv{\lag}{(\partial_{\mu}\phi)}\delta_0\phi)+ \partial_{\mu}(\delta x^{\mu})\lag\\
  &=\id \partial_{\mu}(\delta x^{\mu}\lag)+(\pdv{\lag}{\phi}-\partial_{\mu}\pdv{\lag}{(\partial_{\mu}\phi)})\delta_0\phi+\partial_{\mu}(\pdv{\lag}{(\partial_{\mu}\phi)}\delta_0\phi)\\
  &=\id \partial_{\mu}(\delta x^{\mu}\lag)+\partial_{\mu}(\pdv{\lag}{(\partial_{\mu}\phi)}\delta_0\phi)
\end{align*}
Note that $\delta_0 \phi=\delta \phi-\delta x^{\mu}\partial_{\mu}\phi$, and
\begin{align*}
  \delta S&=\id \partial_{\mu}(\delta x^{\mu}\lag)+\partial_{\mu}(\pdv{\lag}{(\partial_{\mu}\phi)}\delta \phi-\pdv{\lag}{(\partial_{\mu}\phi)}\delta x^{\rho}\partial_{\rho}\phi)\\
  &=\id \partial_{\mu}\{\delta x^{\mu}\lag+\pdv{\lag}{(\partial_{\mu}\phi)}\delta \phi-\pdv{\lag}{(\partial_{\mu}\phi)}\delta x^{\rho}\partial_{\rho}\phi\}\\
  &=\id \partial_{\mu}\{(\lag g^{\mu}_{\rho}-\pdv{\lag}{(\partial_{\mu}\phi)}\partial_{\rho}\phi)\delta x^{\rho}+\pdv{\lag}{(\partial_{\mu}\phi)}\delta \phi\}
\end{align*}

\end{document}
