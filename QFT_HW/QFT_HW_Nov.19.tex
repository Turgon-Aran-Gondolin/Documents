%!mode::"Tex:UTF-8"
\PassOptionsToPackage{unicode}{hyperref}
\PassOptionsToPackage{naturalnames}{hyperref}
\documentclass{article}
\usepackage{geometry}
%\usepackage{fullpage}
\usepackage{parskip}
\usepackage{physics}
\usepackage{amsmath}
\usepackage{amssymb}
\usepackage{xcolor}
\usepackage[colorlinks,linkcolor=blue,citecolor=green]{hyperref}
\usepackage{array}
\usepackage{longtable}
\usepackage{multirow}
\usepackage{comment}
\usepackage{graphicx}
\usepackage{cite}
\usepackage{amsfonts}
\usepackage{bm}
\usepackage{slashed}
\usepackage{dsfont}
\usepackage{mathtools}
\usepackage[compat=1.1.0]{tikz-feynman}
\usepackage{simplewick}
%\usepackage{fourier}
%\usepackage{slashbox}
%\usepackage{intent}
\usepackage{mathrsfs}

\geometry{left=1cm,right=1cm,top=1.5cm,bottom=2cm}

\newcommand{\ac}[1]{a_{\vb{#1}}}
\newcommand{\ad}[1]{a_{\vb{#1}}^{\dagger}}
\newcommand{\bc}[1]{b_{\vb{#1}}}
\newcommand{\bd}[1]{b_{\vb{#1}}^{\dagger}}
\newcommand{\omp}[1]{\omega_{\vb{#1}}}
\newcommand{\intphead}[1]{\int\frac{\dd^3#1}{(2\pi)^3}}
\newcommand{\del}[2]{\delta_{#1#2}}
\newcommand{\id}{\int\dd^4x}
\newcommand{\lag}{\mathcal{L}}
\renewcommand{\bm}[1]{\boldsymbol{#1}}
\newcommand{\trh}{\Lambda_{\frac{1}{2}}}
\newcommand{\gm}{\gamma^{\mu}}
\newcommand{\gn}{\gamma^{\nu}}
\newcommand{\gs}{\gamma^{\sigma}}
\newcommand{\gr}{\gamma^{\rho}}
\newcommand{\gnr}{g^{\nu\rho}}
\newcommand{\gmr}{g^{\mu\rho}}
\newcommand{\gms}{g^{\mu\sigma}}
\newcommand{\gns}{g^{\nu\sigma}}
\newcommand{\vbp}{\vb{p}}
\newcommand{\vbk}{\vb{k}}
\newcommand{\g}{\gamma}
\renewcommand{\a}{\alpha}
\renewcommand{\b}{\beta}
\renewcommand{\t}{\theta}
\newcommand{\la}{\lambda}
\newcommand{\p}{\phi}
\newcommand{\vp}{\varphi}
\newcommand{\s}{\sigma}
\renewcommand{\G}{\Gamma}
\newcommand{\pars}{\slashed\partial}
\newcommand{\ps}{\slashed p}




\newcommand{\Tscr}{\mathscr{T}}
\newcommand{\Cscr}{\mathscr{C}}


\title{Homework: Quantum Field Theory \#7}
\author{Yingsheng Huang}
\begin{document}
\maketitle
{\bf1.}\quad
LIPS for N-body:
$$\dd\Phi_{N_{\beta}}=\prod_{f=1}^{N_{\beta}}\frac{\dd^3p_f}{(2\pi)^32E_{\vb{p}}}\delta^4(\sum_{N_{\alpha}}p_{\alpha}-\sum_{N_{\beta}} p_{\beta})$$
For two-body scenario:

i). $m_1=m_2=0.\,(p_1^{\mu}+p_2^{\mu}=\Pmqty{M,\vb{0}},\;\text{center-of-mass frame})$
\begin{align*}
  \dd\Phi&=\frac{\dd^3p_1\dd^3p_2}{(2\pi)^62E_{\vb{p_1}}2E_{\vb{p_2}}}\delta^4(\sum_{N_{\alpha}}p_{\alpha}-\sum_{N_{\beta}} p_{\beta})\\
  &=\frac{p_1^2\dd p_1\dd\Omega}{(2\pi)^62E_12E_2}\delta(M-E_1-E_2)\\
  \intertext{and $\delta(M-E_1-E_2)=\abs{\dv{(M-E_1-E_2)}{p_1}}^{-1}_{p_1=p_0}\delta(p_1-p_0)=\frac{E_1E_2}{p_0(E_1+E_2)}\delta(p_1-p_0)$ where $p_0$ is the solution of $f(p_1)=M-E_1-E_2=0$}
  &=\frac{p_0^2\dd\Omega}{(2\pi)^62E_12E_2}\frac{E_1E_2}{p_0(E_1+E_2)}\\
  &=\frac{p_0}{(2\pi)^34M}\dd\Omega\\
  \intertext{note that $p_0=\sqrt{\frac{M^2}{4}-m^2}=\frac{M}{2}$}
  &=\frac{1}{8(2\pi)^6}\dd\Omega
\end{align*}

ii). $m_1=m_2=m$, $M>2m$.

We only need to apply the result of the last one (before we apply the massless condition) and we can get
$$\dd\Phi=\frac{\sqrt{M^2-4m^2}}{(2\pi)^68M}\dd\Omega$$

\end{document}
