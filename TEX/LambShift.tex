\documentclass{article}
\usepackage{mathrsfs}
\usepackage{amsmath}
\usepackage{geometry}
\usepackage{graphicx}
%%\usepackage{amsmath}
\usepackage{fouridx}%%���±�
\usepackage{graphicx}
\usepackage{braket}
\usepackage{bm}
\usepackage[toc,page,titletoc,title]{appendix}
\geometry{left=2.5cm,right=2.5cm,top=2.5cm,bottom=2.5cm}%%����ҳ�߾�
\linespread{1.2}
\begin{document}
\title{Numerical results for Lamb Shift}
\author{Sun Qing-Feng\footnote{qfsun@mail.ustc.edu.cn}}
\date{September 2015}
\maketitle
\section{Introduction}
Lamb shift \cite{Lamb1}, the energy splitting between the $2S_{1/2}$ and $2P_{1/2}$ states, is one of the most important experiments in history. As is known, the fine structure, witch
is of $\mathcal{O}{(\alpha^4})$, does not split states of the same $j$ with different $l$:
\begin{equation}
\Delta E_{\text{fine structure}}=\frac{(Z\alpha)^2}{n}\left(\frac{1}{j+\frac{1}{2}}-\frac{3}{4n}\right)E_n.
\end{equation}
For $n=2$ and $Z=1$, the fine structure splitting of the $2P_{1/2}$ (or $2S_{1/2}$) and $2P_{3/2}$ states:
\begin{equation}
\Delta E(2P_{3/2})-\Delta E(2P_{1/2})\approx 4.5\times 10^{-5} \text{ eV}=10948.8 \text{ MHz}.
\end{equation}
But in Lamb's experiments, the energy of $2S_{1/2}$ and $2P_{1/2}$ is not exactly the same.
In this note, I will give a numeric result of the Lamb shift based on the results of  \cite{Pineda}.
\section{Details of numeric calculations of Lamb shift}
\subsection{Contribution from the Bethe's log in $\Delta E(2S_{1/2})$}
The main contribution of Lamb shift comes form the Bethe's log\footnote{In this note, I adopt the atom unite (a.u.) for convenience, details of a.u. are listed in Appendix.}:
\begin{equation}
\Delta E_1=\frac{2\alpha^3}{3\pi}\sum_{m\neq n}{\left|\braket{n|\bm{v}|m}\right|^2}\left(E_m-E_n\right)\ln{\frac{E_r}{\left|E_m-E_n\right|}}.
\end{equation}
It is easy to check:
\begin{equation}\label{Bethe's log}
\Delta E_1=\frac{2\alpha^3}{3\pi}\sum_{m\neq n}{\left|\braket{n|\bm{x}|m}\right|}^2\left(E_m-E_n\right)^3\ln{\frac{E_r}{\left|E_m-E_n\right|}}.
\end{equation}
Here, we have used the elementary commutation relation:
$$
\left[\bm{x},\bm{v}\right]=i, \qquad \qquad \left[\bm{x},\hat{h}_0\right]=i\bm{v}.
$$
$$
\hat{h}_0=\frac{\bm{v}^2}{2}-\frac{Z}{r},
$$
and the $E_r$ above represents the energy of electron at rest:
$$
E_r=m_e c^2=\frac{1}{\alpha^2} \text{ a.u.},
$$
and the energy level for the bound states:
$$
E_n=-\frac{1}{2}\frac{Z^2}{n^2} \text{ a.u.}.
$$
It is easy to check $E_n\sim \alpha^2E_r$ and we expect the Lamb shift $\Delta E(2S_{1/2})-\Delta E(2P_{1/2}) \sim \alpha^5E_r$ in our results.
For the discrete states:
$$
\psi_{nlm}(r,\theta,\phi)=Y_{lm}(\theta,\phi)R_{nl}(r).
$$
where:
\begin{equation}
R_{nl}(r)=\frac{1}{(2n+1)!}\sqrt{\frac{(n+l)!}{(n-l-1)!2n}}\left(\frac{2Z}{n}\right)^{\frac{3}{2}}e^{-\frac{Z r}{n}}\left(\frac{2Z r}{n}\right)^{l}F\left(-(n-l-1),2l+2,\frac{2Z r}{n}\right),
\end{equation}
and
$$
\int_{0}^{+\infty}{d r}{r^2R_{ml}(r)R_{nl}(r)}=\delta_{mn}.
$$
When we take $n=2$ in (\ref{Bethe's log}), the transition matrix elements:
$$
\int_{0}^{+\infty}{d r}{r^3R_{20}(r)R_{m1}(r)}=\frac{256\sqrt{2}}{Z}\left(\frac{m-2}{m+2}\right)^m\frac{\sqrt{m^7(m^2-1)}}{(m^2-4)^3}.
$$
After integration over the azimuthal angle, $\theta$ and $\phi$, we have
$$
\left|\braket{n|\bm{x}|m}\right|^2=\frac{131072}{Z^2}\left(\frac{m-2}{m+2}\right)^{2m}\frac{m^7(m^2-1)}{(m^2-4)^6}.
$$
when $m\gg 2$,
$$
\left|\braket{n|\bm{x}|m}\right|^2\sim \frac{1}{m^3 Z^2}.
$$
We then get the energy shift from the discrete states:
\begin{equation}
\Delta E_{1}^{dis}=44.07 \text{ MHz}.
\end{equation}
For the continuous states, I adopt the "k-scale" normalization:
$$
\int_{0}^{+\infty}{d r}{r^2R_{kl}(r)R_{k'l}(r)}=\delta(k-k').
$$
The $k$ above is related to the energy of the continuous states:
$$
E(k)=\frac{k^2}{2} \text{ a.u.}.
$$
The transition matrix element:
$$
\left|\braket{n|\bm{x}|k}\right|^2=\frac{131072}{Z^3}\frac{e^{-4m'\cot^{-1}{\left(\frac{1}{2}m'\right)}}}{(1-e^{-2\pi m'})}\frac{{m'^{9}(m'^2+1)}}{(m'^2+4)^6}.
$$
where
$$
m'=\frac{Z}{k}.
$$
when $k\gg 1$:
$$
\left|\braket{n|\bm{x}|k}\right|^2\sim\frac{Z^5}{k^8}.
$$
We then get the energy shift from the continuous states:
\begin{equation}
\Delta E_{1}^{con}=1003.35 \text{ MHz}.
\end{equation}
The the total energy shift from the Bethe's log is
\begin{equation}
\Delta E_{1}=\Delta E_{1}^{dis}+\Delta E_{1}^{con}=1047.42\text{ MHz}.
\end{equation}
\subsection{Contribution from the vacuum polarization and Darwin terms in $\Delta E(2S_{1/2})$}
With a.u., the contribution from the vacuum polarization and Darwin terms can be expressed as:
$$
\Delta E_{2}=-\frac{4Z\alpha^3}{15}\left|\phi_{n}(\bm{0})\right|^2=-\frac{4Z^4 \alpha^3}{15\pi n^3},
$$
and
$$
\Delta E_{3}=\frac{4Z\alpha^3}{3}\left(\frac{5}{6}-\ln2\right)\left|\phi_{n}(\bm{0})\right|^2=\frac{4Z^4 \alpha^3}{3\pi n^3}\left(\frac{5}{6}-\ln2\right),
$$
where $\left|\phi_{n}(\bm{0})\right|^2$ comes from the zero-point wave function, and only the s-wave states have non-zero value:
$$
\left|\phi_{n}(\bm{0})\right|^2=\left|R_{n0}(0)Y_{00}\right|^2=\frac{Z^3}{\pi n^3}\frac{1}{a_0^3}.
$$
Then for $n=2$, and $l=0$, $\Delta E_{2}=-27.13 \text{ MHz}$ and $\Delta E_{3}=19.01 \text{ MHz}$.
\section{summary}
The total results of energy shift for the $2S_{1/2}$ state:
\begin{equation}
\Delta E(2S_{1/2})=\Delta E_{1}+\Delta E_{2}+\Delta E_{3}=1039.31\text{ MHz}.
\end{equation}
This result of $\Delta E(2S_{1/2})$ is exactly the same with that in textbook of Weinberg (see Vol. 1,  \emph{The Lamb Shift in Light Atoms}).
The present experimental value gives \cite{Lundeen}
$$
\Delta E(2S_{1/2})-\Delta E(2P_{1/2})=1057.845(9) \text{ MHz}.
$$
\begin{appendices}
\section*{Basic facts about atom units}
In this note, I adopt the Hartree atomic units, where the numerical values of the following four fundamental physical constants are all unity by definition:\\
Electron mass $m_e$;\\
Elementary charge $e$;\\
Reduced Planck's constant $\hbar = \frac{h}{2\pi}$;\\
Coulomb's constant $k_e=\frac{1}{4\pi\epsilon_0}$.
\par
From the convention above, we can give some derived units:
\begin{table}[!hbp]
\centering
\begin{tabular}{|c|c|c|c|}
\hline
\hline
Dimension & Name & Symbol & Expression \\
\hline
Length & Bohr radius & $a_0$ & $\frac{\hbar}{m_e c\alpha}$\\
\hline
Energy & Hartree energy &  $E_h$ & $\alpha^2 m_e c^2$\\
\hline
Velocity & & & $\alpha c$\\
\hline
\end{tabular}
\caption{Derived atomic units}
\end{table}
\\
Then the speed of light:
$$
c=\frac{1}{\alpha}.
$$
In traditional (SI) units, the Hamiltonian is:
$$
\hat{h}_0=-\frac{\hbar^2}{2m_e}{\nabla}^2-\frac{1}{4\pi\epsilon_0}\frac{Z e^2}{r}.
$$
In a.u., it can be simplified as
$$
\hat{h}_0=\frac{\bm{v}^2}{2}-\frac{Z}{r}.
$$
The energy spectrum for the discrete states:
$$
E_n=-\frac{Z^2}{2n^2}.
$$
\end{appendices}
\begin{thebibliography}{Lam}
\bibitem{Lamb1}W. E. Lamb, R. C. Retherford, Phys. Rev. 72(1947)241.
\bibitem{Pineda}Antonio Pineda, Joan Soto, Phys. Lett. B 420(1998)391.
\bibitem{Bethe}H. A. Bethe, Phys. Rev. 72(1947)339.
\bibitem{Lundeen}S. R. Lundeen, F. M. Pipkin, Phys. Rev. Lett. 46(1981)232.
\end{thebibliography}
\end{document}
